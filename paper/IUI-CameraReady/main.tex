\PassOptionsToPackage{table}{xcolor}
\documentclass[sigchi]{acmart}
% \documentclass{sigchi}
% \settopmatter{printacmref=false} % Removes citation information below abstract
% \renewcommand\footnotetextcopyrightpermission[1]{} % removes footnote with conference information in first column
% Load basic packages
\usepackage{booktabs} % For formal tables
\usepackage{balance}  % to better equalize the last page
\usepackage{graphics} % for EPS, load graphicx instead
\usepackage[T1]{fontenc}
\usepackage{booktabs}
\usepackage{textcomp}
\usepackage{xspace}
\usepackage{setspace}
\usepackage[textsize=tiny]{todonotes}
% Some optional stuff you might like/need.
\usepackage{microtype} % Improved Tracking and Kerning
% \usepackage[all]{hypcap}  % Fixes bug in hyperref caption linking
\usepackage{ccicons}  % Cite your images correctly!
% \usepackage[utf8]{inputenc} % for a UTF8 editor only
\usepackage{verbatim}
\usepackage{relsize}
\usepackage{etoolbox}
\usepackage{lipsum}   % for filler text
\usepackage{setspace} % for \onehalfspacing and \singlespacing macros
\usepackage[normalem]{ulem}
\usepackage{fixltx2e}
\usepackage{amsmath}
\usepackage{amssymb}
\usepackage{afterpage}
\usepackage[htt]{hyphenat}
\usepackage{microtype}                 % use micro-typography (slightly more compact, better to read)
\PassOptionsToPackage{warn}{textcomp}  % to address font issues with \textrightarrow
\usepackage{times}                     % we use Times as the main font
\renewcommand*\ttdefault{txtt}         % a nicer typewriter font
\usepackage{tabu}                      % only used for the table example
\usepackage{booktabs}                  % only used for the table example
% \usepackage[linesnumbered,ruled]{algorithm2e}
\usepackage{algorithm}
\usepackage[noend]{algpseudocode}
\usepackage[export]{adjustbox}
\usepackage{tikz}
\usetikzlibrary{shapes,arrows}
\usepackage{subfig}
\raggedbottom
\DeclareMathOperator*{\argmax}{arg\,max}
% llt: Define a global style for URLs, rather that the default one
\makeatletter %making the spacing between paragraphs less dramatic
\def\url@leostyle{%
  \@ifundefined{selectfont}{
    \def\UrlFont{\sf}
  }{
    \def\UrlFont{\small\bf\ttfamily}
  }}
\makeatother

\newenvironment{denselist}{
    \begin{list}{\small{$\bullet$}}%
    {\setlength{\itemsep}{0ex} \setlength{\topsep}{0ex}
    \setlength{\parsep}{0pt} \setlength{\itemindent}{0pt}
    \setlength{\leftmargin}{1.5em}
    \setlength{\partopsep}{0pt}}}%
    {\end{list}}

\newcommand{\squishlist}{
   \begin{list}{$\bullet$}
    { \setlength{\itemsep}{0pt}
      \setlength{\parsep}{2pt}
      \setlength{\topsep}{0pt}
      \setlength{\partopsep}{0pt}
      \leftmargin=25pt
\rightmargin=0pt
\labelsep=5pt
\labelwidth=10pt
\itemindent=0pt
\listparindent=0pt
\itemsep=\parsep
    }
}
\newcommand{\origcolor}[1]{\color[HTML]{7C0FFF}{#1}}
\newcommand{\diffcolor}[1]{\color[HTML]{FA7300}{#1}}
\newcommand{\squishend}{\end{list}}
\newcommand{\npar}{\par \noindent}
% use extensively to toggle between paper and TR
\newcommand{\eat}[1]{}
\newcommand{\papertext}[1]{#1}
% \newcommand{\tr}[1]{{\leavevmode\color{lightgray}{#1}}}
\newcommand{\tr}[1]{}
\newcommand{\boldpara}[1]{\textbf{\paragraph{#1}}}
% de-facto paragraph format
\newcommand{\stitle}[1]{\par\noindent\textbf{#1}}
\newcommand{\opt}[1]{{\leavevmode\color{purple}{#1}}} %optional if need cut (may go into TR)
\newcommand{\ccut}[1]{} %confirmed cut
\newcommand{\system}{\textsc{VisPilot}\xspace}
\newcommand{\shortsystem}{\textsc{VP}\xspace} %shorthand for system
\newcommand{\cluster}{\textsc{Cluster}\xspace}
\newcommand{\BFS}{\textsc{BFS}\xspace}
\newcommand{\fem}{\texttt{Female}\xspace}
\newcommand{\blk}{\texttt{African-American}\xspace}
\newcommand{\blkfem}{\texttt{African-American Female}\xspace}

\newcommand{\agp}[1]{\leavevmode\textcolor[HTML]{AFCD38}{Aditya: #1}}
\newcommand{\dor}[1]{\leavevmode\textcolor[HTML]{00E8D8}{Doris: #1}}
\newcommand{\hdev}[1]{\leavevmode\textcolor[HTML]{9B6191}{Himel: #1}}
\newcommand{\haz}[1]{\leavevmode\textcolor[HTML]{E59400}{Hazem: #1}}
\newcommand{\change}[1]{{\leavevmode\color{red}#1}}
\newcommand{\achange}[1]{{\leavevmode\color{blue}#1}}
\newcommand{\bchange}[1]{{\leavevmode\color{cyan}#1}}

\urlstyle{leo}
% \AtBeginEnvironment{quote}{\small}
\renewenvironment{quote}{%
  \vspace{-3pt}
   \list{}{%
     \leftmargin0.15cm
     \rightmargin\leftmargin
   }
   \item\relax
}
{\vspace{-3pt} \endlist}

% To make various LaTeX processors do the right thing with page size.
\def\pprw{8.5in}
\def\pprh{11in}
\special{papersize=\pprw,\pprh}
\setlength{\paperwidth}{\pprw}
\setlength{\paperheight}{\pprh}
\setlength{\pdfpagewidth}{\pprw}
\setlength{\pdfpageheight}{\pprh}

\newcommand*\OK{\ding{51}}
\renewenvironment{quote}{%
   \list{}{%
     \leftmargin0.15cm
     \rightmargin\leftmargin
   }
   \item\relax
}
{\endlist}
\copyrightyear{2019}
\acmYear{2019}
\setcopyright{acmcopyright}
\acmConference[IUI '19]{24th International Conference on Intelligent User Interfaces}{March 17--20, 2019}{Marina del Ray, CA, USA}
\acmBooktitle{24th International Conference on Intelligent User Interfaces (IUI '19), March 17--20, 2019, Marina del Ray, CA, USA}
\acmPrice{15.00}
\acmDOI{10.1145/3301275.3302307}
\acmISBN{978-1-4503-6272-6/19/03}
% Authors, replace the red X's with your assigned DOI string during the rightsreview eform process.

\settopmatter{printacmref=true}
\fancyhead{}

%%%%%%%%%%%%%%%%%%%%%%%%%%%%%%%%%%%%%%%%%%%%%%%%%%%%%%%%%%%%%%%%
%%%%%%%%%%%%%%%%%%%%%% START OF THE PAPER %%%%%%%%%%%%%%%%%%%%%%
%%%%%%%%%%%%%%%%%%%%%%%%%%%%%%%%%%%%%%%%%%%%%%%%%%%%%%%%%%%%%%%%%

\begin{document}
\author{Doris Jung-Lin Lee$^\dagger$, Himel Dev$^\dagger$, Huizi Hu$^\dagger$, Hazem Elmeleegy$^\star$, Aditya Parameswaran$^\dagger$}
\affiliation{
  \institution{$^\dagger$University of Illinois, Urbana-Champaign, $^\star$Google, Inc.}
}
\email{jlee782|hdev3|huizihu2|adityagp@illinois.edu, elmeleegy@google.com}


% \author{Doris Jung-Lin Lee}
% \affiliation{
%   \institution{University of Illinois, Urbana-Champaign}
% }
% \email{jlee782@illinois.edu}

% \author{Himel Dev}
% \affiliation{
%   \institution{University of Illinois, Urbana-Champaign}
% }
% \email{hdev3@illinois.edu}
% \author{Huizi Hu}
% \affiliation{
%   \institution{University of Illinois, Urbana-Champaign}
% }
% \email{huizihu2@illinois.edu}
% \author{Hazem Elmeleegy}
% \affiliation{%
%   \institution{Google, Inc.}
% }
% \email{elmeleegy@google.com}

% \author{Aditya Parameswaran}
% \affiliation{%
%   \institution{University of Illinois, Urbana-Champaign}
% }
% \email{adityagp@illinois.edu}
%\title{Avoiding the Drill-down Fallacy with Storyboard: Assisted and Accelerated Data Exploration Through Data Subsets}
\title{Avoiding Drill-down Fallacies with {\em VisPilot}: Assisted Exploration of Data Subsets}
\begin{abstract}
As datasets continue to grow in size and complexity,
exploring multi-dimensional datasets remain challenging for analysts.
A common operation during this exploration is drill-down---understanding
the behavior of data subsets by \change{progressively} adding filters.
While widely used, in the absence of careful attention towards confounding factors,
drill-downs could lead to inductive fallacies.
Specifically, an analyst may end up being \lq\lq deceived\rq\rq\ into thinking that a deviation in trend is attributable to a local change, when in fact \change{it is a more general phenomenon};
we term this the \change{{\em drill-down fallacy}}. \change{One way to avoid falling prey to drill-down fallacies}
is to exhaustively explore all potential drill-down paths,
which quickly becomes infeasible \change{on complex datasets with many attributes}.
We present \system, an accelerated visual data exploration tool that guides analysts \change{through the} key insights in a dataset, while avoiding drill-down fallacies. Our user study results show that \system helps analysts discover interesting visualizations, understand attribute importance, and predict unseen visualizations better than other \change{multidimensional data analysis} baselines.
%towards meaningful insights for a variety of tasks.
\end{abstract}
\maketitle
 \author{Doris Jung-Lin Lee, Himel Dev, Huizi Hu, Hazem Elmeleegy, Aditya Parameswaran}


\keywords{exploratory data analysis, visualization recommendation.}
% \begin{teaserfigure}
%   \centering
%   \includegraphics[width=\linewidth]{figures/US_Election_Example.pdf}
%   \caption{A set of visualizations from the 2016 Election polls. These visualizations show the percentage of votes for three candidates (Donald Trump, Hilary Clinton, and Others) in different demographic groups (based on race and gender).}
%   \label{fig:elections_example}
% \end{teaserfigure}

%The task of navigating through a large, multidimensional dataset is a common challenge in exploratory analysis. Due to limitations on the number of visualizations that an analyst can examine at one time, the narrow scope of drill-downs can often lead to inductive fallacies. %Not only is manual drill-down and roll-up on data subsets tedious and inefficient for the analyst, the massive space of data subsets, lack of interesting patterns in most data subsets, fallacies of spurious correlations, and pitfalls of statistical paradoxes calls for a systematic and effective way for analysts to make sense of and navigate through the large space of possible visualizations.
%In this paper, we present \system, an interactive visualization recommendation system provide safe guarantee during drill-down exploration by picking the proper visualization reference that leads to interesting and informative trends. Given a dataset and the x and y axes of interest, \system\ intelligently explores the lattice of equivalent visualizations across data subsets, and recommends interesting and informative visualizations. The recommended visualizations are then displayed in an interactive dashboard, where the visualizations are organized into a hierarchical layout. Our evaluation study shows that visualization dashboards generated by \system\ are interpretable and leads to higher performance in data analytic tasks compared to the competing baselines.

% \begin{figure*}[bht]
% \centering
% \includegraphics[width=\linewidth]{figures/US_Election_Example.pdf}
% \caption{A set of visualizations from the 2016 Election polls. These visualizations show the percentage of votes for three candidates (Donald Trump, Hilary Clinton, and Others) in different demographic groups (based on race and gender).}
% \label{fig:elections_example}
% \end{figure*}

%!TEX root = main.tex
\section{Introduction}
%Exploring multidiemnsional dataset is hard
\par Visual data exploration is the \emph{de facto} first step in understanding multi-dimensional datasets. This exploration enables analysts to identify trends and patterns, generate and verify hypotheses, and detect outliers and anomalies. However, as datasets grow in size and complexity, visual data exploration ends up becoming challenging. In particular, to identify patterns that merit further investigation, an analyst may need to explore different subsets of the data to determine when and where certain patterns occur. Manually generating and examining each visualization in this space of data subsets (which grows exponentially in number of attributes) presents a major bottleneck in exploration.
%Drill-Down for exploration
\begin{figure}[ht!]
% \includegraphics[width=\linewidth]{figures/elections_example_lattice.pdf}
\includegraphics[width=\linewidth]{figures/elections_example_lattice_teaser.pdf}
\caption{Example data subset lattice from the 2016 US election dataset illustrating the drill-down fallacy along the purple path as opposed to the informative orange path.}
\label{fig:elections_example}
\end{figure}
\par One way of navigating this combinatorial space is to perform drill-downs on the space of data subsets (hereafter referred to as \emph{lattice}). For example, a campaign manager who is interested in understanding the voting patterns across different demographics (say, race, gender, or social class) using the 2016 US election exit polls~\cite{exitpolls} may first generate a bar chart for the entire population, where the x-axis shows the election candidates and the y-axis the percentage of votes for each of these candidates. In Figure~\ref{fig:elections_example}, the visualization at the top of the lattice corresponds to this overall population. They may then drill down to specific demographics of interest, say gender-based demographics, by generating bar charts for female voters, as shown in the second visualization at the second row of Figure~\ref{fig:elections_example}.
%Challenges associated with drill-down
\par There are three challenges associated with performing manual drill downs in this manner. First, it is often not clear which attributes to drill-down on. Analysts may use their intuition for choosing the drill-down attribute, but such arbitrary exploration may lead to large portions of the lattice being unexplored. Second, an uninformed path taken by analysts may lead to visualizations that are not very surprising or insightful. For example, an analyst may well end up wasting effort by drilling down from the \texttt{Black} visualization to the \texttt{Black Female} one in Figure~\ref{fig:elections_example}, since the two distributions are similar and therefore not very surprising. Last but most importantly, an analyst may encounter what we are calling the ``drill-down fallacy''. As shown in Figure~\ref{fig:elections_example}, an analyst can arrive at the \texttt{Black Female} visualization by either going through the purple or the orange drill-down path. An analyst who followed the purple path may be surprised at how drastically the \texttt{Black Female} voting behavior differs from that of the \texttt{Female}. This behavior is no longer surprising if the analyst had gone down the unsurprising orange path that we saw earlier, where the proper reference (i.e., the vote distribution for \texttt{Black}) explains the vote distribution for \texttt{Black Female}. In other words, even though the vote distribution for \texttt{Black Female} is very different from that of \texttt{Female}, the phenomenon can be explained by a more general ``root cause'' attributed to the voting behavior for the \texttt{Black} community. Attributing an overspecific cause to an effect, while ignoring the actual cause, not only leads to less interpretable explanations for the observed visualizations, but can also be detrimental to decision-making. For example, for the campaign manager, this could lead to a misallocation of campaign funds.
\par The aforementioned example demonstrates the \emph{drill-down fallacy}---incomplete insights that result from potentially confounding factors not explored along a drill-down path. In particular, while performing drill-downs on randomly selected paths, analysts may find a ``local difference'' in trends, without being aware of the more ``general phenomenon'' that could explain the trend of interest. Without the proper parent reference visualization that explains the behavior of the visualization of interest, analysts are at risk of falling prey to the drill-down fallacy. A naive solution to avoid this fallacy is to explore all potential drill-down paths. Unfortunately, this approach does not scale with the increasing number of factors in the drill-down path.
\par In this paper, we present a visual data exploration tool, titled \system, that addresses the three aforementioned challenges of exploration through three principles: (i) \textbf{Safety} (i.e., ensure that proper informative references are present to avoid drill-down fallacies), (ii) \textbf{Saliency} (i.e., identify interesting visualizations that convey new information or insights), and (iii) \textbf{Summarization} (i.e., succinctly convey the key insights present in a dataset). To facilitate safety, we develop a notion of \emph{informativeness}---the capability of a reference visualization to explain the visualization of interest. To facilitate saliency, we characterize the notion of \emph{interestingness}---the difference between a visualization and its informative reference in terms of underlying data distribution. Finally, to facilitate summarization, we embrace a \emph{collective} measure of visualization utility by recommending a connected network of visualizations that collectively offer informative insights. Based on these three principles, our tool, \system, automatically identifies a network of visualizations that succinctly conveys the key informative insights in a dataset. Our user study results demonstrate that our tool can guide an analyst towards meaningful insights for a variety of tasks. Our contributions include:
\begin{denselist}
\item Identifying and characterizing the notion of ``drill-down fallacy'', a common fallacy that have not yet been studied extensively in the past.
\item Introducing the novel concept of \emph{informativeness} that helps users identify meaningful insights that arise from something \textit{actually interesting} about the data (instead of confounding variables),
\item Designing a system, \system, that automatically identifies a network of visualizations that succinctly conveys the key informative insights in a dataset,
\item Demonstrating the efficacy of our system through a user study evaluation on how well users can retrieve interesting visualizations, judge the importance of attributes, and predict unseen visualizations, against two other summarization baselines.
\end{denselist}
%!TEX root = main.tex
\section{Data and User Models\label{sec:datamodel}}
\par In this section, we first describe how analysts explore the lattice through drill-downs and introduce a common fallacy that arises when analyst have limited time and attention to examine all possible factors that contribute to the observed visualization. Then, we discuss how to resolve the problem of finding informative visualizations for a given visualization.
% How users explore visualizations
% 	- Drill down
% 	- Expectation formation
% 		- focussing on bar chart 
% to explore the space of possible data subset
\par Research in visualization storytelling shows that people prefer visualization sequences structured hierarchically with increasing levels of aggregation~\cite{Kim2017,Hullman2017,Hullman2013}. In order to find the desired data subset, analysts often drill-down to explore data at different levels of granularity by adding one filter at a times. For each data subset that he encounters, he may want to visualize the distribution of measure values for each data subset through a bar chart. When analysts perform drill-downs, they naturally formulate their expectation based on the last visualization that they observe, known as the `parent', which is the visualization that can be obtained by removing one filter constraint from the current visualization in context (known as the `child' visualization). For example in Figure~\ref{fig:elections_example}, the visualizations Female and Black are the parents of the Black Female visualization. By extending this concept of parent-child relationships, we can organize the space of visualization from different data subsets to form a lattice as shown in Figure~\ref{fig:elections_example}.
\begin{figure}[h!]
\includegraphics[width=\linewidth]{figures/elections_example_lattice.pdf}
\caption{Example data subset lattice illustrating the misleading factor fallacy along the orange path as opposed to the informative purple path.}
\label{fig:elections_example}
\end{figure}
% Fallacies of Forming Expectations:
% 	- two extremes: 
% 		- random parent v.s. exhaustive parent browsing
% 		- limitation of analyst 
\par As the analyst is drilling down by adding one filter at a time, the analyst is prone to be misguided by parent visualizations that highly deviate from its child, overlooking other potential factors that may explain the seemingly-anomalous behavior. We refer to this phenomena as \emph{drill-down fallacy}, as this type of fallacy arises from the inductive nature of the drill-down operation. We demonstrate this fallacy with an example from the 2016 US Elections exit polls dataset. As shown in Figure~\ref{fig:elections_example}, an analyst can either arrive at the Black Females visualization by going through the purple path or the orange path. At random, if the analyst went down the purple path, he may be surprised at how much the Black Female voting behavior differs drastically from the vote distribution for females. This behavior can be explained if the analyst went down the orange path, where he sees the proper reference (vote distribution for Black) that explains the behavior of the Black Female distribution. While such fallacies can be prevented if the analyst browses through all possible parents of any visualization that he observes in the dataset, the prohibitively large number of visualizations and limited memory and attention of analysts make this task impractical.
% Problem definition 
% 	- picking right parent
\par Since it is impossible to examine all possible parents and potentially misleading if we simply picked a few parents to examine, our goal is to develop a mechanism that would  \emph{provide safe guarantee by picking the proper informative parent} as a reference when analysts navigate through the space of data subsets.  To model the informativeness of an observed parent in the context of an unseen visualization, we characterize the capability of the parent in predicting the unseen visualization. An observed parent is \emph{informative} if its data distribution closely follows the data distribution of the unseen child visualization, since the visualization helps the analyst form an accurate mental picture of what to expect from the unseen visualization. Specifically, we formulate the informativeness of an observed parent $V_i^j$ of an unseen visualization $V_i$ as the similarity between their data distributions measured using a distance function $D(V_i, V_i^j)$. The most informative parents $V_i^*$ of an unseen visualization $V_i$ are the ones whose data distributions are most similar to the unseen.
\begin{equation}
    V_i^*=\underset{V_i^j}{argmin}\ D(V_i, V_i^j)
\end{equation}
We regard a visualization as informative if its distance falls within a user-defined threshold $\theta\%$ close to its most informative parent:
\begin{equation}
    V_i^{*, \theta} = \{V_i^j : \frac{D(V_i, V_i^*)}{D(V_i, V_i^j)} \geq \theta\}
\end{equation}
For example in Figure~\ref{fig:elections_example}, while both visualization Black and Female visualizations are considered parents of the Black Female visualization, only the Black visualization are considered the informative parent of the black female population, for any values of $\theta \geq 11\%$ via the Euclidean distance metric. Note that, our proposed system can work with different distance metrics such as cosine similarity and earth mover's distance. Without loss of generality, we chose to use Euclidean distance metric for the remainder of our paper.

%!TEX root = main.tex
\section{System\label{sec:system}}
\subsection{System Objective}
%the visualization corresponding to black voters is the most informative parent of the visualization corresponding to black female voters. For $\theta <= 0.11$, the former remains the only informative parent of the latter.

\stitle{Interestingness:} While informative parents contribute to the prediction of an unseen visualization, the most interesting visualizations to recommend are those for which \emph{even the informative parents fail to accurately predict the visualization}. \dor{Can we justify this based on our findings?} To model the interestingness of an unseen visualization $V_i$ in the context of an observed parent $V_i^j$, we characterize the deviation between their data distributions using a distance function $D(V_i, V_i^j)$. The unseen visualizations whose data distributions deviate from the observed informative parents are \emph{interesting}. \cut{The most interesting unseen visualizations $V_\#$ are the ones that deviate most from their observed informative parents.
\begin{equation}
    V_\#=\underset{V_i}{argmax} \ D(V_i, V_i^{*, \theta})
\end{equation}
In Figure \ref{fig:elections_example}, the most interesting visualization to recommend is the one corresponding to white female voters. This visualization significantly differs from its informative parent---the visualization corresponding to female voters.} \dor{The argmax notation not necessary since we're just using this in our utility function. The election example is not convincing, the informative parent of white female is actually white and not female. Also the differences are not too significant.}

%\noindent Additional model extensions can be added to this objective function based user specification. For example, there may be $k$ visualizations that approximately yield equal contribution to the user's expectation. For simplicity of notation, we have assumed $k=1$ in the aforementioned model. In order, a user may want to prevent the recommendation of spuriously interesting subsets of the data. We can discard visualizations that falls below a certain subpopulation size threshold.
\stitle{Subpopulation size consideration:} The danger of spurious patterns and correlations in visualizations that contain small subpopulation size is a well-known problem in exploratory analysis~\cite{Binnig2017}. We take two preventive measures to avoid including such misleading visualization in our dashboards. First, in the lattice generation process discussed in Section~\ref{sec:algorithms}, we allow users to select an `iceberg condition' \footnote{The terminology is used in the discussion of iceberg cubes in OLAP literature~\cite{Xin2007}.} ($\delta$) to adjust the extent of pruning on visualizations whose sizes fall below a certain percentage of the overall population size. Second, we downweigh the interestingness edge utility $U(V_i, V_i^j)$ between a parent $V_i^j$ and a child visualization $V_i$ by the ratio of their sizes:
\begin{equation}
    U(V_i, V_i^j) = \frac{|V_i|}{|V_i^{j}|} \cdot D(V_i, V_i^j)
    \label{edge_utility}
\end{equation}

Given the lattice data model and the user model for visualization utility described above, the goal of our system is to generate a dashboard by selecting $k$ visualizations from the lattice. We enforce that the generated dashboard satisfies several requirements:
 \begin{enumerate}
  \item Dashboard must include the overall visualization (topmost visualization with no filter applied) to serve as reference to the rest of the visualizations in the dashboard.
  \item For each visualization except for the overall, at least one of its informative parents is included within the $k$ visualizations. This excludes the uninformative parents as exemplified in black female example in the dashboard, especially since our findings 3 and 4 show that showing multiple, improper parents can mislead the participants, resulting in a higher variance across their estimations. %This enforces that every visualization shown in the dashboard has an informative reference to compare against to create a connected story.
  \item The selected $k$ visualizations are collectively most ``interesting'' in presence of their informative parents as measured by the utility in Equation \ref{edge_utility}.
\end{enumerate}
 The problem of finding a connected subgraph in the lattice that has the maximum combined edge utility is  known as the maximum-weight connected subgraph problem~\cite{ErnstAlthaus2009} and is known to be NP-Complete, via a reduction from the \textsc{Clique Problem}~\cite{Parameswaran2010}. In Section~\ref{sec:algorithms}, we discuss heuristic algorithms used for deriving a locally optimal solution for ensuring interactive runtime.

\subsection{System Architecture}
We have implemented \system\ as a Flask web application on top of a PostgreSQL database. In Figure~\ref{system_architecture}, we present the system architecture of \system, which consists of three core modules: the traversal module, the query module, and the statistics module. The interaction manager deals with the supported user interaction described in Section~\ref{sec:interaction} and sends a request to the lattice module which  contains several algorithms for generating and traversing the visualization lattice described in Section~\ref{sec:algorithms}. For generating the visualization lattice, the lattice module passes a list of data subsets corresponding to visualizations to be generated to the query module. The query module translates these visualizations into queries, and then optimizes (by grouping) and executes the queries. The statistics module is an optional module that allows the lattice module to prune low-utility visualizations without actually generating them. Specifically, it generates coarse statistics for the unexplored visualizations based on the current list of explored visualizations. Finally, the dashboard renderer takes the resulting visualizations to be included in the dashboard and perform any rendering preprocessing procedures for display and navigation of the dashboard as described in Section \ref{sec:navigation}.
\begin{figure}[ht!]
\centering
\includegraphics[width=\linewidth]{figures/system_architecture.png}
\caption{System Architecture of \system. User starts with x and y axes of interest and requests for $k$ visualizations in the dashboard. The request is processed by generating the lattice with the help of the querying module, visualization selection through the lattice traversal algorithms, and finally the dashboard is displayed at the frontend through the dashboard renderer. }%  The interaction manager translates the request to the traversal module that ???? [should we look at the offline case??]}
\label{system_architecture}
\end{figure}

\subsection{Algorithms\label{sec:algorithms}}
We give an overview of our algorithms by first discussing the approaches to generate the visualization lattice, and then presenting a high-level overview of our traversal algorithms.

\stitle{Lattice Generation.} Our system supports two variants of traversal algorithms based on the lattice generation procedure---offline variants that first generate the complete lattice and then work towards identifying the maximum utility solution, and online variants that incrementally generate the lattice and simultaneously identify the solution. The offline variants are appropriate for datasets with a small number of low-cardinality attributes, where we can generate the entire lattice in a reasonable time; whereas the online variants are appropriate for datasets with large number of high-cardinality attributes, where we incrementally generate a partial lattice.

%In most cases, the lattice contains a large number of visualizations due to the presence of many attributes or high-cardinality attributes in the dataset. In such cases finding an optimal solution is computationally challenging.

\stitle{Lattice Traversal.} Given the materialized lattice, the objective of the traversal algorithm is to find the connected subgraph in the lattice that has the maximum combined edge utility. Here, we discuss the \textit{frontier greedy} algorithm which is used for generating the dashboards for our user study and defer our discussion on the details of other algorithms that we have developed to the technical report.
% \begin{figure}[ht!]
% \centering
% \includegraphics[width=0.4\linewidth]{figures/frontier.pdf}
% \caption{Toy example demonstrating the notion of ``frontier''. Nodes that have been picked to include in the dashboard are colored green. The neighbors of the set of picked nodes are the frontier nodes, shown in pink. Grey nodes are other unpicked nodes in the lattice.}
% \end{figure}
%We devised two classes of heuristics algorithms, namely, frontier-based algorithms, and path-merging algorithms. These algorithms are guaranteed to find a solution that satisfies the constraints of our problem, except for the optimality.
\techreport{The frontier-based algorithms traverse the lattice from root to downwards, incrementally adding new nodes (visualizations) to the current solution (dashboard) till it reaches the maximum capacity $k$. To achieve this, the algorithms maintain a list of candidate nodes---called \textit{frontier} nodes---any of which can be added to the current solution since their informative parent is already present in the solution. At each step, the algorithms add a node from frontier to the current solution, and update the frontier accordingly.  The frontier based algorithms can be further categorized into three types based on their node selection strategy (from frontier), namely greedy algorithm, random walk algorithm, and probabilistic algorithm. The greedy algorithm picks the current best node from frontier (thus concentrates on exploitation), random walk algorithm picks a random node (thus concentrates on exploration), and probabilistic algorithm picks a random high-utility node (thus trades off between exploration and exploitation).}
\par As described in Algorithm \ref{algo:frontier_greedy}, our algorithm obtains a list of candidate nodes known as the \textit{frontier} nodes (pink in Figure\ref{fig:lattice} left), which encompasses all neighbors of nodes in the existing subgraph solution. Any of the nodes in the frontier can be added to the current solution since their informative parent is guaranteed to be present in the solution. The \texttt{getFrontier} function scans and adds all children of leaf nodes of the current dashboard as part of the frontier. In the online version, it additionally checks for each child whether its informative parent is present in the current dashboard. At each step, our algorithm greedily picks the node with the maximum utility from the frontier to the current solution, and updates the frontier accordingly.

\techreport{The path merging algorithm first generate the informative paths from root to every candidate node. Then, it greedily merges the paths with high-utility to create a subgraph whose size is less than or equal to maximum capacity $k$.}

% \begin{algorithm}
%     \SetKwInOut{Input}{Input}
%     \SetKwInOut{Output}{Output}
%     \Input{Precomputed Lattice of Visualizations, $G = \{V_1, \ldots, V_n\}$}
%     \Output{A Dashboard of Size $k$, $S$}
%     $S = \{ V_{root}\}$\;
%     $F = get\_child(V_{root})$\;
%     \While{$size(S) < k$}
%     {
%     	$s_{next} = pick\_next(F)$\;
%     	$S = S \cup s_{next}$\;
%       \For{$i = 0;\ i < size(S);\ i = i + 1$}
%       {
%           $F = (F \cup get\_child(S[i])) - S$\;
%       }
%     }
%     return $S$\;
%     \caption{Frontier Based Algorithm}
% \end{algorithm}

\begin{algorithm}
  \begin{algorithmic}[1]
  \Procedure{pickVisualizations}{k,lattice}
  \State dashboard $\gets$ \{ $V_{overall}$ \}
  \While{|dashboard| < k}
      \State frontier $\gets$ getFrontier(dashboard,lattice)
      \State maxNode $\gets$ getMaxUtilityNode(frontier)
      \State dashboard $\gets$ dashboard $\cup$ \{maxNode\}
  \EndWhile
  \Return dashboard
  \EndProcedure
  \end{algorithmic}
  \caption{Frontier Greedy Algorithm}\label{algo:frontier_greedy}
\end{algorithm}

%\textbf{Greedy Algorithms:} Greedy algorithms select the locally optimal node to be added to the frontier.

%A specific implementation would need to specify a scoring function to nodes in frontier that is used to pop out the next node in each iteration. One can design a scoring function based on the trade-off between performance and complexity. In the most simple case, we can use the edge weights to score nodes in the frontier. That is, at each point we add a node with the highest interestingness value. We note that this is quite a greedy approach. Specifically, we might miss visualizations with high utility that are in deeper levels of the graph. Thus, another approach would be to extent the horizon for which we calculate a nodes utility. We denote such approach as a look-ahead approach. With a free parameter $n$, we would like to assign a score to each frontier node the corresponds to the expected utility of adding this node and $n-1$ more nodes who are its descendants. For example, we can run BFP for each node in frontier treating it as a root.

\techreport{The path merging algorithm first generate the informative paths from root to every candidate node. Then, it greedily merges the paths with high-utility to create a subgraph whose size is less than or equal to maximum capacity $k$.}

\subsection{User Interaction\label{sec:interaction}}
\begin{figure}[ht!]
\centering
\includegraphics[width=\linewidth]{figures/overview.jpeg}
\caption{Overview of the \system interface for the Police Dataset~\cite{ctrp3}. Users can select x and y axes of interest, as well as a choice of an aggregation function. Default values are set for system related parameters such as the number of visualizations to show in the dashboard (k), iceberg condition for pruning ($\delta$), and informative parent criterion ($\theta$), which can be adjusted by the users via the sliders if needed.}
\label{fig:overview}
\end{figure}
\par Figure \ref{fig:overview} shows an overview of the \system interface. After the user selects the x and y axes of interest, aggregation function, and optional system parameter settings, an initial dashboard of $k$ visualizations is displayed on the canvas, such as the one seen in main canvas of Figure \ref{fig:overview}.  The system provides toolbar buttons with keyboard binding for zooming in, out, and extent, as well as moving around the canvas. Alternatively, users can zoom and pan with mouse click and scroll.

%\hdev{(1) The second sentence is in passive voice. (2) What are the optional system parameters? Clearly state them. (1) Simplify the third sentence. A non-UI person may not know the meaning of some of the terms. (1) In fourth sentence, if you are using "alternatively", there's no need for "also".}

\par After browsing through the visualizations in the dashboard, users may be interested in getting more information about a particular node. \system supports a mechanism for users to request additional summarizations based on a chosen visualization of interest. As shown in Figure \ref{fig:altroot_expansion} (left), the analyst starts with a 5-visualization dashboard on a police stop dataset~\cite{ctrp3}. The dataset contains  records of vehicle and pedestrian stops from law enforcement departments in Connecticut, dated from 2013 to 2015. The analyst learns that for the drivers who had contraband found in the vehicle, the arrest rate for drivers who are 60 and over is surprisingly higher than usual, whereas for Asian drivers the arrest rate is lower. In addition, he is also interested in learning more about the other factor that contribute to high arrest rate: duration=30+min. He clicks on the corresponding visualization and requests for 2 additional visualizations. Upon seeing the updated dashboard in Figure~\ref{fig:altroot_expansion} (right), he learns that similar to the selected visualization, any visualization that involves the duration=30+min filter results in high ticketing and arrest rates. This implies that if a police stop lasts more than 30 minutes, the outcome would more or less be the same, independent of other factors, such as driver's race or age. \system uses the same models and algorithms as before, except the root node is now set as the selected visualization, rather than the overall visualization. This node expansion capability is similarly motivated by the idea of \textit{iterative view refinement} in other visual analytics system \cite{Wongsuphasawat2016,Hoque2017}, which is essential for the users to iterate on and explore different hypotheses.

\begin{figure}[ht!]
\centering
\includegraphics[width=\linewidth]{figures/expansion_example.pdf}
\caption{Left: Original k=5 dashboard with the duration=30+min visualization clicked. A pop-up is displayed to submit the request for additional summary visualizations to be generated. Right: Resulting dashboard after requesting for 2 more visualizations based on the visualization of interest.}
\label{fig:altroot_expansion}
\end{figure}

\subsection{Assistive tools for visualizing large lattices\label{sec:navigation}}
Due to the amount of space occupied by the hierarchical layout when the number of visualizations gets large, we have developed tools to help users navigate through different parts of the dashboard interactively.
\stitle{Navigation Minimap:}  When the user zooms in on the dashboard, an overview mini-map is shown on the upper left-hand side of the canvas to help users identify which region of the dashboard they are currently exploring, as shown in Figure \ref{fig:hover_minimap}.
\stitle{Collapsed visualizations:}
One observation that we found across several datasets was that many visualizations had identical distributions, which resulted in lots of wasted space. Apart from their attribute name, these visualizations are not very informative for the users, therefore, we offer an option to collapse these visualization, as demonstrated in Figure \ref{fig:collapse_demo}. A visualization can be collapsed if it has more than one redundant sibling and does not have any children, so that there are no hidden stories due to lower-level dependencies. As shown in Figure \ref{fig:hover_minimap}, collapsed nodes can be easily identified by an orange border and the details of which visualizations are in the collapsed node are displayed when the user hovers over the visualization.
%\afterpage
\begin{figure}[ht!]
\centering
\includegraphics[width=\linewidth]{figures/minimap_zoom.png}
\caption{Zoomed-in version of Figure \ref{fig:collapse_demo} showing the labels of a collapsed visualization when user hovers over the visualization. The navigation minimap is shown in the top-left to help users navigate through the large dashboard.}
\label{fig:hover_minimap}
\end{figure}

%!TEX root = main.tex
\change{\section{Evaluation Study Methods\label{sec:userstudy}}}
%between-subject
In this section, we describe the methodology
for \change{a user study}
we conducted for evaluating the
\change{usefulness of \system
for various exploratory analysis tasks}.
We aim to evaluate whether \change{\system's ``3S'' design principles enables analysts }to effortlessly identify insights in comparison
with \change{conventional approaches for multidimensional data exploration}.%other dashboard generation baselines
\subsection{\change{Participants and Conditions}}
We recruited 18 participants \change{(10 Male; 8 Female)}
with prior experience in working with data.
Participants included undergraduate
and graduate students, researchers,
and data scientists, with $1-14$ years of data
analysis experience (average: $5.61$).
\tr{This can include, but are not limited to, browsing and reading data, data cleaning and wrangling, data visualization and model building. The inclusion criteria is assessed based on a self-reporting basis in the pre-study survey.}
No participants reported prior experience
in working with the two datasets used in the study (described below).
Participants were randomly assigned two
of the three types of dashboards with \change{$k=10$}
visualizations generated \change{via the}
following conditions.
\papertext{\change{The specific dashboards for each dataset and condition can be found in our technical report~\cite{TR}}.}
%(described in Section \ref{sec:algorithms})
\stitle{\system:} The dashboards for this condition
are generated by the aforementioned
frontier greedy algorithm and displayed
in a hierarchical layout as in Figure~\ref{fig:overview}.
\cut{To ensure the informativeness of the generated dashboards,
we selected a more stringent $\theta$=90\%
criteria to generate the dashboards for our user study.}
To establish a fair comparison
with the two other conditions,
we deactivated \cut{iceberg pruning (by setting $\delta$=0) and}
the interactive node expansion capabilities.
\stitle{\BFS (short for breadth-first search):}
Starting from the visualization of the \change{overall} population,
$k$ visualizations are selected level-wise,
traversing down the subset lattice,
adding the visualizations at the first level
with $1$-filter combination one at a time,
\change{and then visualizations with $2$-filter combinations},
and so on,
until $k$ visualizations have been added.
This baseline is designed to simulate a dashboard
generated by a meticulous analyst who exhaustively
inspects all visualizations (i.e., filter combinations)
from the top down.
\change{These visualizations are then displayed
in a $5\times2$ table.}
\stitle{\cluster:} \change{In this condition,}
$k$-means clustering is first performed on the data distributions
\change{of all of the visualizations in the lattice}.
This results in $k$ clusters \change{that cover the rest of the}
visualizations.
\change{For each cluster, we select
the visualization with the least number of filter conditions
as the cluster representative for interpretability}\tr{\footnote{Since the clusters cover all visualizations in the dataset and the overall visualization has the minimum number of filter across all visualization, the overall visualization is guaranteed to be picked as one of the displayed visualizations.}}
and display them in a $5\times2$ table layout.
This baseline is designed to showcase a diverse set of
distributions within the dataset.
\tr{
  \begin{figure*}[ht!]
  \centering
  \includegraphics[width=0.95\linewidth]{figures/dashboard_examples.pdf}
  \caption{\tr{Specific dashboards for each dataset and condition used in the study.}}
  \label{fig:dashboards}
  \vspace{-10pt}
  \end{figure*}
}
\smallskip
\stitle{Dataset Descriptions.}
Each participant was assigned two different
conditions on two different datasets
\change{(Police Stop and Autism, described below)}.
The ordering of each condition was
randomized to prevent confounding learning effects.
The study began with a 5-minute tutorial
using dashboards generated from the Titanic dataset~\cite{titanic}
for each condition.
To prevent bias across conditions,
participants were not provided an explanation of
how the dashboards were generated and
why the visualizations were arranged in a particular way.

\change{The first dataset in the study} was the aforementioned
Police Stop dataset.
The attributes in the dataset
include driver gender, age, race, stop time of day,
stop outcome, whether a search was conducted,
and whether contraband was found.
We generated dashboards of bar chart visualizations
with x-axis as the stop outcome
(i.e., whether the police stop resulted in a
ticket, warning, or arrest) and y-axis as the percentage of police stops that led to each outcome. %We randomize the ordering for each task combination to prevent confounding learning effects. %%, which contains visualizations of the \% of police stop that resulted in a warning, ticket, or an arrest. %, which contains a total of 312948 records of vehicle and pedestrian stops from law enforcement departments in Connecticut, dated from 2013 to 2015.

The second dataset in the study
was the Autism dataset~\cite{autism},
\change{describing} the results of autism spectrum
disorder screening for 704 \change{adults}.
The attributes in the dataset are binary responses
to 10 diagnostic questions
as part of the screening process.
This dataset serves as a data-agnostic condition,
since there was no descriptions
of the questions or answer labels provided to
\change{our study participants}.
We generated dashboard visualizations
based on \change{the percentage of adults that were diagnosed with autism}.

\change{\subsection{Study Procedure}
\par After the tutorial, for each dataset, participants}
were given some time to read through a worksheet
containing the descriptions of the data attributes.
Then, they were given an attention check question
where they were
\change{provided}
a verbal description of the visualization filter
\change{(i.e., data subset)}
and asked about the corresponding visualization in the dashboard.
After understanding the dataset and chart schema,
participants were asked to accomplish
\change{various
tasks.
Since \system was developed
based on a joint utility objective,
it is impossible to design tasks that evaluate
each of the ``3S'' principles individually.
Instead, our tasks were selected to measure
the overall efficacy and usefulness
of the dashboards
in helping a participant
understand and become aware of different aspects of
and insights within a dataset
during drill-down analysis.
These different aspects of
dataset understanding
can be roughly illustrated
via Figure~\ref{fig:frontier_greedy},
from insights
gained from \emph{individual}
displayed visualizations (blue selected nodes),
to predicting behavior of \emph{related}
visualizations (green related nodes),
to understanding \emph{overall} attribute importance
(entire lattice, a mix of green, blue, and unselected white nodes).
}

\stitle{\change{Labeling (Individual Assessment):}}
Participants were asked to talk aloud
as they interpreted the visualizations
in the dashboard and \change{label}
each one as interesting or
not interesting, \change{or leave it unselected}.
This \change{subjective task measures}
how \change{interesting \emph{individual}
selected visualizations were}
to participants. % (RQ\change{1}).%well are participants at retrieving interesting visualizations (RQ1).
\stitle{\change{Prediction (Related Assessment):}}
Participants were given a separate worksheet
and asked to sketch an estimate for a visualization
that is not present in the dashboard.
For every condition, the visualization to be estimated
contained $2$ filter combinations,
with exactly one parent present in the given dashboard.
After making the prediction,
participants were shown the actual data
distribution and asked to rate on a Likert scale
of 10 how surprising the result was
\change{(1: not surprising and 10: very surprising)}.
\change{This task measured how well participants
inferred the behavior of \emph{related},
unobserved visualizations based on a
limited set of selected dashboard visualizations.}
\stitle{\change{Ranking (Overall Assessment):}}
Participants were given a sheet of paper
with all the attributes listed
and asked to rank the attributes
in order of importance in contributing
to a particular outcome
(e.g., factors leading to an arrest or autism diagnosis).
Participants were allowed to assign equal ranks to more than one attribute or skip attributes that they were unable to infer importance for.
Attribute ranking tasks are \change{common in many data science
use-cases, such as feature selection and key driver analysis. Since all dashboards were equal in size, our goal was to check whether this size limitation came at the cost of \emph{overall} dataset understanding. Thus, the goal of this task was to study participant's overall dataset understanding by measuring how well participants judged the relative importance of each attribute.}% in contributing towards an outcome. %(RQ3).
 %(RQ2).}
%\change{This task measured how \emph{informative} the displayed visualizations are in allowing participants to predict unseen visualizations.}
\smallskip
\par At the end of the study, we asked two open-ended questions regarding the insights \change{gained by} participants and what they \change{liked or disliked} about each dashboard. On average, the study lasted around 48 minutes.
%\agp{Order of RQs here and in study results are different. The way the RQs are described are different as well. Please fix before I look at S5}\dor{Earlier, we arranged the results in S5 in the chronological order in which it was conducted through the study, but the interestingness task is the most subjective and underwhelming result, so we instead focus on the informative task first as RQ1 to reinforce the connection with drill-down fallacy. I've changed the RQ labels in S4 to fit this, albeit still a bit awkward in S4.}

%!TEX root = main.tex

	\section{Study Results}
	\bchange{
	 We introduce the study findings for each task starting from the narrowest scope of \emph{individual} visualizations to the widest scope of \emph{overall} dataset understanding.
}
\change{\stitle{RQ1: How are \emph{individual} selected visualizations in the dashboard perceived subjectively by the users?}
}
\npar Using click-stream data logged from the user study, 
we recorded whether a
participant \bchange{labeled each} visualization 
in the dashboard as interesting, not interesting, 
or left the visualization unselected. 
Table~\ref{table:interestingScore} summarizes 
\bchange{the} counts of visualizations 
marked as interesting or not interesting 
aggregated across conditions. 
We also normalize the interestingness count 
by the total number of selected visualizations 
to account for variations in how some participants 
select more visualizations than others. 
The results indicate that participants who 
used \system\ \change{saw} more 
visualizations that they found interesting compared 
to the \BFS and \cluster \change{\xspace conditions}. 
\change{While  \bchange{this task is 
inherently subjective, with many possible
reasons why a participant may have marked a visualization 
as interesting, this result is indicative of the fact that the
selected visualizations were deemed to be relevant
by users. We will drill into possible reasons why in the next section.}}
\begin{table}[h!]
\vspace{-5pt}
	\centering
	\begin{tabular}{lrrr}
	 \small{Condition}             &   \small{\system} &   \small{\BFS} &   \small{\cluster} \\
	\hline
	 \small{Interesting}            &  \cellcolor{blue!25}       66    & 61    &      51   \\
	 \small{Not Interesting}        &  \cellcolor{blue!25}       10    & 20    &      22   \\
	 \small{Interesting (Normalized)} &   \cellcolor{blue!25}       0.87 &  0.75 &       0.7 \\
	\end{tabular}
	\caption{Total counts of visualizations marked as interesting or not interesting across the different conditions. \system leads to more visualizations marked as interesting and fewer visualizations marked as uninteresting.}
	\label{table:interestingScore}
	\vspace{-20pt}
\end{table}
\stitle{\change{RQ2: How well do dashboard visualizations provide users with an accurate \bchange{understanding} of \emph{related} visualizations?}}
\npar 
\bchange{As discussed in Section~\ref{sec:problem}, 
contextualizing visualizations correctly with informative
references can help prevent users from falling prey to 
drill-down fallacies. To this end,
the prediction task aims to assess whether
users can employ visualizations in the dashboard
to correctly predict unseen ones. 
Indeed, if the dashboard is constructed well,
one would expect that only the visualizations
that are not very surprising relative to their informative parents
would be excluded from the dashboard 
(i.e., their deviation from their informative parents is not large).} 
\begin{figure}[h!]
\centering
\vspace{-10pt}
\includegraphics[width=0.95\linewidth]{figures/prediction_surprisingness_distance.pdf}
\vspace{-5pt}
\caption{Left: Euclidean distance between predicted and ground truth. In general, predictions made using \system are closer to ground truth. Right: Surprisingness rating reported by users after seeing the actual visualizations on a Likert scale of 10. \system participants had a more accurate mental model of the unseen visualization and therefore reported less surprise than compared to the baselines.}
\vspace{-10pt}
\label{fig:distance}
\end{figure}
\par The accuracy of participants' predictions 
\bchange{is measured using}
the Euclidean distance between \change{their} predicted distributions and ground truth data distributions. 
As shown in Figure~\ref{fig:distance} (left), 
predictions made using \system\ (highlighted in red) 
were closer to the actual distribution than compared to the baselines, 
as indicated by the smaller Euclidean distances. 
Figure~\ref{fig:distance} (right) also shows 
that \system participants \change{were able to more 
accurately reason about the expected properties of 
unseen data subsets \bchange{(or visualizations)}, 
since they rated the resulting visualizations to be less surprising}. 
\cluster may have performed better \bchange{for} 
the Police dataset than it did \bchange{for} 
the Autism \bchange{one},
\bchange{for the same reason as in the attribute ranking task,
where more univariate visualizations happened to be selected,
described subsequently.} 


\par We also compute the variance of participants' predictions across the same condition. In this case, low variance implies 
that 
\bchange{there is consistency or agreement
between the predictions of participants 
who consumed the same
dashboard}, 
whereas high variance implies that the 
dashboard did not convey a clear data-driven story 
that could guide participants' predictions. 
So instead, participants \bchange{had to rely on prior knowledge or} guessing to 
\bchange{inform} their prediction(s). These trends can be observed in both Figure~\ref{fig:distance} and in more detail in Figure \ref{fig:actual_predictions}, where the prediction variance amongst participants who used \system\ is generally lower than the variance for the baselines. 
\bchange{Overall, \system provides participants 
with a more accurate and consistent model of 
\emph{related} visualizations.}
\begin{figure}[h!]
\vspace{-10pt}
\centering
\includegraphics[width=0.85\linewidth]{figures/prediction.pdf}
\vspace{-10pt}
\caption{Mean and variance of predicted values. Predictions based on \system exhibit lower variance \change{(error bars)} and closer proximity to the ground truth values (dotted).\agp{Fix}}
\label{fig:actual_predictions}
\vspace{-10pt}
\end{figure}
\change{
	\stitle{RQ3: How well does the dashboard convey information regarding \bchange{the} \emph{overall} dataset schema?}% information regarding the relative importance of different attributes
\npar 
\bchange{We use the common  task 
of judging the relative importance of attributes 
as an indicator of the participants' overall understanding.}}
To determine \bchange{ground truth} attribute importance, 
we computed the Cramer's V statistics 
between attributes to be ranked 
and the attributes of interest. 
Cramer's V is \bchange{commonly used for} 
determining 
the strength of association between categorical attributes~\cite{McHugh2013}. We deem an attribute as important if it has one 
of the top-three\footnote{\change{This relevancy cutoff is visually-determined via the elbow method to indicate which rank the Cramer's V score drops off significantly.}} Cramer's V scores amongst all attributes of the dataset. 
For the list of rankings provided by each participant, 
we first remove attributes \bchange{that} participants chose not to rank. 
We compute the F-scores and average precision (AP) at k 
\bchange{relative to the ground truth
for various values of $k$} 
\tr{(from 1 up to the number of ranked attributes, with k values corresponding to attributes ranked as ties deduplicated)\agp{I don't understand this parenthetical note so I have moved it to the TR)}}. 
Table \ref{table:ranking_results} \change{reports the average across participants} in each condition, after picking the best performing $k$ value for each user based on F-score and AP respectively. Both measures capture how accurately participants were able to \bchange{identify} 
the three most important attributes for each dataset.
\begin{table}[ht!]
	\centering
	\vspace{-10pt}
	\begin{tabular}{lllll}
	         & \multicolumn{2}{c}{Police}                                   & \multicolumn{2}{c}{Autism}                                   \\ \hline
	Metric   & F                             & AP                            & F                             & AP                            \\ \hline
	\system  & \cellcolor{blue!25}0.750 & \cellcolor{blue!25}0.867 & 0.723                         & 0.600                         \\ 
	\cluster & 0.739                         & 0.691                         & \cellcolor{blue!25}0.725 & \cellcolor{blue!25}0.665 \\
	\BFS     & 0.739                         & 0.592                         & 0.222                         & 0.200                         \\ 
	\end{tabular}
	\caption{Best AP and F-scores for the attribute ranking task.}
	\label{table:ranking_results}
	\vspace{-20pt}
\end{table}
\par 
\bchange{For this task,
we expected \BFS to have an inherent
advantage, 
since \BFS dashboards consist of all univariate distributions, 
providing more high-level information regarding each attribute.
However,}
both \system and \cluster (which contained more `local' information) 
performed better than \BFS. 
The problem with \BFS is that given 
\bchange{a limited dashboard budget or real-estate of $k = 10$ 
visualizations that could be displayed, 
not all univariate distributions were shown}. 
For the Police dataset, it happened to select several  
important attributes (related to contraband and search) 
to display in the first 10 visualizations. 
However, \bchange{for Autism,} 
only visualizations \bchange{corresponding to} 
binary diagnostic 
questions 1-4 fit in the dashboard. 
So the poor ranking behavior comes from the fact that 
the \BFS generated dashboard failed to display 
the three most important attributes (questions 5, 6 and 9) 
given the limited budget. 
This \bchange{demonstrates \BFS's lack of 
consistency across different datasets,
due to the fact that exhaustive exploration
can only lead to limited understanding of the data.}

\par We see that \system\ performs better 
than \cluster for the Police dataset 
and closely follows \cluster for the Autism dataset. 
It is not entirely surprising that \cluster did well, 
since it is a well-established method for 
summarizing high-dimensional data~\cite{Han2005}. 
For Autism, \cluster happened to pick the majority of visualizations (8/10) as univariate distributions that exhibited high-skew and diversity, 
leading to more informed inference \bchange{of} attribute importance. 
Since clustering seeks visualizations 
that exhibit diversity in the shape of the data distributions, 
it could potentially result in visualizations with many filter combinations. 
For the police dataset, 6 out of 10 visualizations had 
\change{more than 2} filters, 
making it difficult to interpret \change{the visualization} 
without \bchange{an} appropriate context to compare against.

\par \bchange{Overall,} both \BFS and \cluster 
do not provide consistent guarantees for highlighting important visualizations across different datasets. In general, our results indicate that \change{participants} gain a better \bchange{\emph{overall} dataset} understanding regarding attribute importance using \system, with only a few \bchange{targeted visualizations that tell the ``entire story''. This is without \system being explicitly optimized 
for the ranking task}.

\section{Discussion}
\subsection{Statistical Paradoxes}\dor{make title full sentences}
Visualizations are powerful representations for studying different distributions or patterns in a dataset. But our intuition often misleads us when it comes to interpreting those patterns.\dor{CITE vis papers that talks about misleading insights (jeremy boy et al CHI 2016, etc)} There are several statistical paradoxes that persuade people to draw incorrect conclusions from observed data or visualizations. The key reason why many of these paradoxes emerge is the \emph{incompleteness} of the observed data (or visualizations). For example, the presence of latent confounding variable causes Simpson's paradox. Similarly, the absence of (or disregard to) base rate information causes base rate fallacy. We assert \dor{too strong of a sentence} that distributional awareness can be useful in avoiding such statistical paradoxes. If an analyst is aware of all distributions in a given dataset, he/she is less prone to many statistical paradoxes. However, given the large number of dimensions and high cardinality of these dimension in modern datasets, it is not possible for an analyst to explore and memorize all distributions. Therefore, a more evolved approach is to be aware of the exceptional distributions. In this work, we propose a first step towards this goal, where we identify the exceptional distributions in terms of their informative references. The remaining (unseen) distributions in the dataset are rather unsurprising and can be inferred from the visualizations in the dashboard. \dor{I would recommend first talk about issue with large dimension + danger of multiple hypothesis testing + incomplete testing, point out problem, then talk about how our system resolves this.}

\subsection{Structural Insight}
Our proposed dashboard consists of a hierarchy of visualizations, where each visualization is linked to its most informative parent. The shape or structure of the hierarchy contains useful information that augments the information learnt from the visualizations. \dor{what's interesting here is that while many work have looked at visualization presentation, layout of presentation never considered, we find in Sec 5 that this is actually important and can encode info.} For example, the depth and branching factor of the hierarchy could inform a user regarding the configuration of insights. Deep hierarchies contain long paths, i.e., insights are present at lower level visualizations with multiple constraints. In contrast, bushy hierarchies (with high branching factor) contain cases where multiple visualizations have the same informative parent and they differ from that parent. We assert that the depth and branching factor could be a meaningful constraint in our problem formulation. Some applications for example, funnel exploration require studying deep hierarchies, whereas others for example, building decision trees require studying bushy hierarchies. A natural extension of our current problem formulation is to allow users to select the depth and branching factor for the hierarchy.

\subsection{Other Visualization Lattices}
In this work, we explore the space of data subsets to generate our visualization lattice. Note that it is possible to explore the space of dimension attributes in x-axis to generate a different visualization lattice. In particular, given a combination of dimension attributes $X = \{X_1, \ldots, X_n\}$, adding one or more new dimensions in $X$ will generate a new combination. An ancestor-descendant relationship exists between these dimension combinations, following the same principles of Section 3.1. These relationships lead to a new lattice, which we call the dimension combination lattice. Our informative deviation based approach could be used for traversing the dimension combination lattice. However, we observe that most users do not visualize more than two attributes in x-axis. Therefore, traversing the dimension combination lattice is not very useful for most applications.

%\subsection{Utility Metrics} 




%!TEX root = main.tex
\section{Related Work\label{sec:related}}
Our work draws from, and improves upon, past research in multidimensional data exploration and fallacies in visual analytics. \change{Other less-relevant past work on decision tree visualization and visualization storytelling is included as reference in the technical report.}
\subsection{Guided Exploration of Multidimensional Data}
Given a dataset, tools such as Tableau support automatic generation of visualizations based on perceptual graphical presentation rules~\cite{Mackinlay2007,Wongsuphasawat2016}. A more recent body of work automatically selects visualizations based on statistical measures, such as scagnostics and deviation. Given a scatterplot, Anand et al. \cite{Anand2015} applies randomized permutation tests to select partitioning variables that reveals interesting small multiples using scagnostics. Given a bar chart, Vartak et al. \cite{Vartak2015} finds other interesting bar charts that deviate from the input chart using a deviation-based measure. Our work extends the deviation-based measure to formulate user expectation. However, unlike existing works, we concentrate on informativeness, which enables our system to avoid drill-down fallacies.
%\cite{Elmqvist2008Rolling} presents an interactive tool to explore multidimensional data using a matrix of scatterplots that shows the relationship between all pairs of attributes.
% \dor{add smart drill-down~\cite{Joglekar2015}}
\subsection{Preventing Biases and Statistical Fallacies}
Visualizations are powerful representations for discovering trends and patterns in a dataset; however, cognitive biases and statistical fallacies could mislead analysts' interpretation of those patterns~\cite{Alipourfard2018WSDM,Wall2017,Zgraggen2018CHI,Armstrong2014}. Wall et al.~\cite{Wall2017} presents six metrics to systematically detect and quantify bias from user interactions in visual analytics. These metrics are based on coverage and distribution, which focus on the assessment of the process by which users sample the data space. Alipourfard et al.~\cite{Alipourfard2018WSDM} presents a statistical method to automatically identify Simpson's paradox by comparing statistical trends in the aggregate data to those in the disaggregated subgroups. Zgraggen et al.~\cite{Zgraggen2018CHI} presents a method to detect the presence of the multiple comparisons problem in visual analysis. In this paper, we concentrate on a novel type of fallacy during drill-down exploration that has not been addressed by past work. %drill-down fallacy, a fallacy that has not been addressed before in visual analytics literature.
\tr{
  \subsection{Decision Tree Visualization}
  The popularity of decision trees in a variety of classification tasks have led to the development of visualizations that make these models more interpretable~\cite{Ankerst1999,Hermann2017,Terence2018}. These visualizations often contain a visual representation of the rules as paths connecting the decision nodes, illustrating the proportion of sample along different paths, as well as statistics regarding the prediction accuracy at every node. Though our dashboards visually look similar to decision trees, the underlying objectives are different for the two methods. During tree construction, a decision tree algorithm aims to improve the classification accuracy of a target variable, typically by minimizing the entropy of distribution from parent node to child node~\cite{Quinlan1986}. In contrast, our method aims to deliver informative insights, by maximizing the informative deviation between parent and child nodes. Consequently, the generated outcomes are different for the two methods---a decision tree well explains the general rules (e.g., if stop duration is more than 30 minutes, the driver has 60\% probability of being arrested), whereas our method well explains the exceptions (e.g., if a stop duration is more than 30 minutes and the driver's race is Asian, the probability of arrest goes down to 35\%). Note that the general rule is useful for predicting the stop outcome for an unlabeled test datapoint (classification), whereas the exception is useful for realizing when the general rule no longer holds (insight). The latter insight may not be discovered by a decision tree as it does not directly improve classification accuracy. Another key difference between the two methods is \emph{coverage}---a decision tree covers the entire dataset (consistent with its classification goal), whereas our method highlights only the interesting regions of a dataset (consistent with its insight goal).

  \subsection{Storytelling with Visualization Sequences}
  Visualizations are often arranged in a sequence to narrate a data-driven story. Existing work on visualization sequences and storytelling has studied the structures of narrative visualizations~\cite{Hullman2017,Segel2010}, effects of augmenting exploratory information visualizations with narration~\cite{Boy2015} and, more recently, ways to automate the creation of visualization sequences~\cite{Hullman2013,Kim2017}. Most of these work have adopted a linear layout (motivated by slidedecks) to present the visualization sequences. Hullman et al.~\cite{Hullman2017} found that most people prefer visualization sequences structured hierarchically based on shared data properties such as levels of aggregation. Kim et al.~\cite{Kim2017} modeled relationships between charts by empirically estimating transition (edge) cost between moving from one visualization (node) to another. They found that participants preferred ``\textit{starting from the entire data and introducing increasing levels of summarization}''. Our work is the first to automatically organize visualizations in a hierarchical layout for summarizing data distributions across the space of data subsets.
}

%!TEX root = main.tex
\section{Conclusion}
\par Common analytics tasks, such as causal inference, feature selection, and outlier detection require studying data distributions at different levels of data granularity~\cite{Anand2015,Heer2012,Wu2013,Hullman2017}. However, without knowing \textit{what} subset of data contains an insightful distribution, manually exploring distributions from all possible data subsets can be tedious and inefficient. Moreover, when examining data subsets by adding one filter at a time, analysts can fall prey to the drill-down fallacy, where they mistakenly attribute the interestingness of a visualization to a ``local difference'', while overlooking a more general explanation for the root cause of the behavior. To address these issues, we presented \system, an interactive visualization recommendation system that automatically selects a small set of informative and interesting visualizations to summarize key distributions within a dataset. Our user study demonstrates that \system can guide analysts towards more informed decisions for retrieving interesting visualizations, judging the relative importance of attributes, and predicting unseen visualizations, than compared to two \change{other baselines}. Study participants also find dashboard generated by \system to be more interpretable and ``human-like'', leading to more discovered insights. Our work is one of the first automated systems that guides analysts across the space of data subsets by summarizing key insights with safety guarantees---a step towards our grander vision of developing intelligent tools for accelerating and assisting with visual data discovery.  
% discovery
% - Drill down is hard and dangerous
% - Drill down fallacy
% - In this paper, we develop ----
% - \system does X, Y , Z
% Our user study shows that ----\system compared to baselines
% 	- perform better in a wide range of analytic task such as attribute ranking, prediction, and interestingness.
% 	- interpretable, more insights
% - Wider implications

\medskip
 \stitle{Acknowledgments.} \bchange{We thank the anonymous reviewers for their valuable feedback. We acknowledge support from grants IIS-1513407, IIS-1633755, IIS-1652750, and IIS-1733878 awarded by the National Science Foundation, and funds from Microsoft, 3M, Adobe, Toyota Research Institute, Google, and the Siebel Energy Institute. The content is solely the responsibility of the authors and does not necessarily represent the official views of the funding agencies and organizations.}

% \newpage
\bibliographystyle{SIGCHI-Reference-Format}
% \bibliographystyle{ACM-Reference-Format}
\bibliography{reference}

\end{document}
