%!TEX root = main.tex
\section{Conclusion}
\agp{If after everything is done, there is need for further space reduction, we can shrink this.}
\par Common analytics tasks, such as causal inference, feature selection, and outlier detection require studying data distributions at different levels of data granularity~\cite{Anand2015,Heer2012,Wu2013,Hullman2017}. However, without knowing \textit{what} subset of data contains an insightful distribution, manually exploring distributions from all possible data subsets can be tedious and inefficient. Moreover, when examining data subsets by adding one filter at a time, analysts can fall prey to the drill-down fallacy, where they mistakenly attribute the interestingness of a visualization to a ``local difference'', while overlooking a more general explanation for the root cause of the behavior. To address these issues, we presented \system, an interactive visualization recommendation system that automatically selects a small set of informative and interesting visualizations to \change{convey} key distributions within a dataset. Our user study demonstrates that \system can guide \change{participants toward} more informed decisions for retrieving interesting visualizations, judging the relative importance of attributes, and predicting unseen visualizations than compared to two \change{other baselines}. Study participants also find dashboard generated by \system to be more interpretable and ``human-like'', leading to more discovered insights. Our work is one of the first automated systems that guides analysts across the space of data subsets by summarizing key insights with safety guarantees---a step towards our grander vision of developing intelligent tools for accelerating and assisting with visual data discovery.  
% discovery
% - Drill down is hard and dangerous
% - Drill down fallacy
% - In this paper, we develop ----
% - \system does X, Y , Z
% Our user study shows that ----\system compared to baselines
% 	- perform better in a wide range of analytic task such as attribute ranking, prediction, and interestingness.
% 	- interpretable, more insights
% - Wider implications
