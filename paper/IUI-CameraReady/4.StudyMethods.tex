%!TEX root = main.tex
\change{\section{Evaluation Study Methods\label{sec:userstudy}}}
%between-subject
In this section, we describe the methodology and results for \change{a user study} we have conducted for evaluating the utility of \system. To assess the efficacy of \system across various exploratory analysis goals, we \change{evaluate how \system dashboards enable participants to gain \textit{interesting}, \textit{informative},  \textit{summarized} insights (corresponding to the three aforementioned design objectives---saliency, summarization, and safety) compared to baseline approaches.}
% addressing the following research questions:
% \begin{denselist}
% 	\item RQ1: How \textit{interesting} are the visualizations in the dashboard perceived subjectively by the users?
% 	\item RQ2: How well \change{does} the dashboard \textit{summarize} the relative importance of different attributes within a given dataset?
% 	\item RQ3: How \textit{informative} are the visualizations in the dashboard at providing users with an accurate understanding of unseen child visualizations?
% \end{denselist}
% These research questions roughly correspond to the three S's in our objective---saliency, summarization, and safety.
\change{\subsection{Participants and Conditions}}
We recruited 18 participants \change{(10 Male; 8 Female)} with prior experience in working with data. Participants included undergraduate and graduate students, researchers, and data scientists, with 1 to 14 years of data analysis experience (average = 5.61). \tr{This can include, but are not limited to, browsing and reading data, data cleaning and wrangling, data visualization and model building. The inclusion criteria is assessed based on a self-reporting basis in the pre-study survey.} No participants reported prior experience in working with the two datasets used in the study (described below). Participants were randomly assigned two of the three types of dashboards with k=10 visualizations generated by following conditions. \change{The dashboards for each condition is shown in the technical report for reference.}
%(described in Section \ref{sec:algorithms})
\stitle{\system:} The dashboards for this condition are generated by the aforementioned frontier greedy algorithm and displayed in a hierarchical layout (as seen in Figure~\ref{fig:overview}). \tr{To ensure the informativeness of the generated dashboards, we selected a more stringent $\theta$=90\% criteria to generate the dashboards for our user study.} In order to establish a fair comparison with the two other conditions, we deactivated \tr{iceberg pruning (by setting $\delta$=0) and} the interactive node expansion capabilities.
\stitle{\BFS (short for breadth-first search):} Starting from the visualization of the entire population, $k$ visualizations are selected level-wise, traversing down the subset lattice, adding the visualizations at the first level with 1-filter combination one at a time, proceeding with the 2-, 3-, and so on, until $k$ visualizations have been added to the dashboard. This baseline is designed to simulate a dashboard generated by a meticulous analyst who exhaustively inspects all possible visualizations (i.e., filter combinations) from the top-down. The chosen visualizations are displayed in a 5x2 table in the traversed order.
\stitle{\cluster:} K-Means clustering is performed on the data distributions of all possible visualizations for the dataset. This results in $k$ clusters covering all visualizations of the dataset, corresponding to $k$, the number of visualizations to be shown in the dashboard. For each representative cluster, we select the visualization with the least number of filter conditions for interpretability and display them in a 5x2 table layout. \opt{Since the clusters cover all visualizations in the dataset and the overall visualization has the minimum number of filter across all visualization, the overall visualization is guaranteed to be picked as one of the displayed visualizations.} This baseline is designed to showcase a diverse set of pattern distributions within the dataset.
\par Each participant was assigned two different conditions on two different datasets. The ordering of each condition was randomized to prevent confounding learning effects. The study began with a 5-minute tutorial using dashboards generated from the Titanic dataset~\cite{titanic} for each condition. To prevent bias across conditions, participants were not provided an explanation of how the dashboards were generated and why the visualizations were arranged in a particular way. Then, participants proceeded onto the aforementioned Police Stop dataset. The attributes in the dataset include driver gender, age, race, stop time of day, stop outcome, whether a search was conducted, and whether contraband was found. We generated dashboards of bar chart visualizations with x-axis as the stop outcome (i.e., whether the police stop resulted in a ticket, warning, or arrest) and y-axis as the percentage of police stops that led to each outcome. %We randomize the ordering for each task combination to prevent confounding learning effects. %%, which contains visualizations of the \% of police stop that resulted in a warning, ticket, or an arrest. %, which contains a total of 312948 records of vehicle and pedestrian stops from law enforcement departments in Connecticut, dated from 2013 to 2015.
\par The second dataset in the study is the Autism dataset~\cite{autism}, which includes the result of autism spectrum disorder screening for 704 adults. The attributes in the dataset are binary responses to 10 diagnostic questions that are part of the screening process. This dataset serves as a data-agnostic condition, since there was no descriptions of the questions or answer labels provided to the user. We generate dashboard visualizations based on whether the participant is diagnosed with autism or not.
\change{\subsection{Study Procedure}}
\par Participants were given some time to read through a worksheet containing descriptions of the data attributes. Then, they were given an attention check question where they were given a verbal description of the visualization filter and asked about the distributions for the corresponding visualization in the dashboard. After understanding the dataset and chart schema, participants were asked to accomplish the following tasks in the prescribed order below:
\stitle{Interestingness Labelling:} Participants were asked to talk aloud as they interpreted the visualizations in the dashboard and mark each visualization as either interesting, not interesting, or leave it as unselected. This task was intended to measure how interesting \change{the selected visualizations were} to participants (RQ1).%well are participants at retrieving interesting visualizations (RQ1).

\stitle{Attribute Ranking:} Participants were given a sheet of paper with all the attributes listed and asked to rank the attributes in order of importance in contributing to a particular outcome (e.g., factors leading to an arrest or autism diagnosis). Participants were allowed to assign equal ranks to more than one attribute or skip attributes that they were unable to infer importance for. Attribute ranking tasks are common in feature selection and other data science tasks. The goal of this task was to measure how well participants understood the relative importance of each attribute in contributing towards an outcome (RQ2).

\stitle{Informative Prediction:} Participants were given a separate worksheet and asked to sketch an estimate for a visualization that is not present in the dashboard. For every condition, the visualization to be estimated contained 2 filter combinations, with exactly one parent present in the given dashboard. After making the prediction, participants were shown the actual data distribution and asked to rate on a Likert scale of 10 how surprising the result was (where 1 is not surprising and 10 is very surprising). The prediction task measured how accurate participants are at predicting an unseen visualization, estimating how well they understood key informative insights that influences other distributions from the dataset (RQ3).
\par We repeated the same study procedure described above for the Autism dataset. At the end of the study, we asked two open-ended questions regarding the insights that participants have learned and what they like or dislike about each dashboard. On average, the study lasted around 48 minutes.
