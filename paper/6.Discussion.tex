\section{Discussion}
\subsection{Statistical Paradoxes}\dor{make title full sentences}
Visualizations are powerful representations for studying different distributions or patterns in a dataset. But our intuition often misleads us when it comes to interpreting those patterns.\dor{CITE vis papers that talks about misleading insights (jeremy boy et al CHI 2016, etc)} There are several statistical paradoxes that persuade people to draw incorrect conclusions from observed data or visualizations. The key reason why many of these paradoxes emerge is the \emph{incompleteness} of the observed data (or visualizations). For example, the presence of latent confounding variable causes Simpson's paradox. Similarly, the absence of (or disregard to) base rate information causes base rate fallacy. We assert \dor{too strong of a sentence} that distributional awareness can be useful in avoiding such statistical paradoxes. If an analyst is aware of all distributions in a given dataset, he/she is less prone to many statistical paradoxes. However, given the large number of dimensions and high cardinality of these dimension in modern datasets, it is not possible for an analyst to explore and memorize all distributions. Therefore, a more evolved approach is to be aware of the exceptional distributions. In this work, we propose a first step towards this goal, where we identify the exceptional distributions in terms of their informative references. The remaining (unseen) distributions in the dataset are rather unsurprising and can be inferred from the visualizations in the dashboard. \dor{I would recommend first talk about issue with large dimension + danger of multiple hypothesis testing + incomplete testing, point out problem, then talk about how our system resolves this.}

\subsection{Structural Insight}
Our proposed dashboard consists of a hierarchy of visualizations, where each visualization is linked to its most informative parent. The shape or structure of the hierarchy contains useful information that augments the information learnt from the visualizations. \dor{what's interesting here is that while many work have looked at visualization presentation, layout of presentation never considered, we find in Sec 5 that this is actually important and can encode info.} For example, the depth and branching factor of the hierarchy could inform a user regarding the configuration of insights. Deep hierarchies contain long paths, i.e., insights are present at lower level visualizations with multiple constraints. In contrast, bushy hierarchies (with high branching factor) contain cases where multiple visualizations have the same informative parent and they differ from that parent. We assert that the depth and branching factor could be a meaningful constraint in our problem formulation. Some applications for example, funnel exploration require studying deep hierarchies, whereas others for example, building decision trees require studying bushy hierarchies. A natural extension of our current problem formulation is to allow users to select the depth and branching factor for the hierarchy.

\subsection{Other Visualization Lattices}
In this work, we explore the space of data subsets to generate our visualization lattice. Note that it is possible to explore the space of dimension attributes in x-axis to generate a different visualization lattice. In particular, given a combination of dimension attributes $X = \{X_1, \ldots, X_n\}$, adding one or more new dimensions in $X$ will generate a new combination. An ancestor-descendant relationship exists between these dimension combinations, following the same principles of Section 3.1. These relationships lead to a new lattice, which we call the dimension combination lattice. Our informative deviation based approach could be used for traversing the dimension combination lattice. However, we observe that most users do not visualize more than two attributes in x-axis. Therefore, traversing the dimension combination lattice is not very useful for most applications.

%\subsection{Utility Metrics} 



