%!TEX root = main.tex
\section{System\label{sec:system}}
\subsection{System Objective}
%the visualization corresponding to black voters is the most informative parent of the visualization corresponding to black female voters. For $\theta <= 0.11$, the former remains the only informative parent of the latter.

\stitle{Interestingness:} While informative parents contribute to the prediction of an unseen visualization, the most interesting visualizations to recommend are those for which \emph{even the informative parents fail to accurately predict the visualization}. \dor{Can we justify this based on our findings?} To model the interestingness of an unseen visualization $V_i$ in the context of an observed parent $V_i^j$, we characterize the deviation between their data distributions using a distance function $D(V_i, V_i^j)$. The unseen visualizations whose data distributions deviate from the observed informative parents are \emph{interesting}. \cut{The most interesting unseen visualizations $V_\#$ are the ones that deviate most from their observed informative parents.
\begin{equation}
    V_\#=\underset{V_i}{argmax} \ D(V_i, V_i^{*, \theta})
\end{equation}
In Figure \ref{fig:elections_example}, the most interesting visualization to recommend is the one corresponding to white female voters. This visualization significantly differs from its informative parent---the visualization corresponding to female voters.} \dor{The argmax notation not necessary since we're just using this in our utility function. The election example is not convincing, the informative parent of white female is actually white and not female. Also the differences are not too significant.}

%\noindent Additional model extensions can be added to this objective function based user specification. For example, there may be $k$ visualizations that approximately yield equal contribution to the user's expectation. For simplicity of notation, we have assumed $k=1$ in the aforementioned model. In order, a user may want to prevent the recommendation of spuriously interesting subsets of the data. We can discard visualizations that falls below a certain subpopulation size threshold.
\stitle{Subpopulation size consideration:} The danger of spurious patterns and correlations in visualizations that contain small subpopulation size is a well-known problem in exploratory analysis~\cite{Binnig2017}. We take two preventive measures to avoid including such misleading visualization in our dashboards. First, in the lattice generation process discussed in Section~\ref{sec:algorithms}, we allow users to select an `iceberg condition' \footnote{The terminology is used in the discussion of iceberg cubes in OLAP literature~\cite{Xin2007}.} ($\delta$) to adjust the extent of pruning on visualizations whose sizes fall below a certain percentage of the overall population size. Second, we downweigh the interestingness edge utility $U(V_i, V_i^j)$ between a parent $V_i^j$ and a child visualization $V_i$ by the ratio of their sizes:
\begin{equation}
    U(V_i, V_i^j) = \frac{|V_i|}{|V_i^{j}|} \cdot D(V_i, V_i^j)
    \label{edge_utility}
\end{equation}

Given the lattice data model and the user model for visualization utility described above, the goal of our system is to generate a dashboard by selecting $k$ visualizations from the lattice. We enforce that the generated dashboard satisfies several requirements:
 \begin{enumerate}
  \item Dashboard must include the overall visualization (topmost visualization with no filter applied) to serve as reference to the rest of the visualizations in the dashboard.
  \item For each visualization except for the overall, at least one of its informative parents is included within the $k$ visualizations. This excludes the uninformative parents as exemplified in black female example in the dashboard, especially since our findings 3 and 4 show that showing multiple, improper parents can mislead the participants, resulting in a higher variance across their estimations. %This enforces that every visualization shown in the dashboard has an informative reference to compare against to create a connected story.
  \item The selected $k$ visualizations are collectively most ``interesting'' in presence of their informative parents as measured by the utility in Equation \ref{edge_utility}.
\end{enumerate}
 The problem of finding a connected subgraph in the lattice that has the maximum combined edge utility is  known as the maximum-weight connected subgraph problem~\cite{ErnstAlthaus2009} and is known to be NP-Complete, via a reduction from the \textsc{Clique Problem}~\cite{Parameswaran2010}. In Section~\ref{sec:algorithms}, we discuss heuristic algorithms used for deriving a locally optimal solution for ensuring interactive runtime.

\subsection{System Architecture}
We have implemented \system\ as a Flask web application on top of a PostgreSQL database. In Figure~\ref{system_architecture}, we present the system architecture of \system, which consists of three core modules: the traversal module, the query module, and the statistics module. The interaction manager deals with the supported user interaction described in Section~\ref{sec:interaction} and sends a request to the lattice module which  contains several algorithms for generating and traversing the visualization lattice described in Section~\ref{sec:algorithms}. For generating the visualization lattice, the lattice module passes a list of data subsets corresponding to visualizations to be generated to the query module. The query module translates these visualizations into queries, and then optimizes (by grouping) and executes the queries. The statistics module is an optional module that allows the lattice module to prune low-utility visualizations without actually generating them. Specifically, it generates coarse statistics for the unexplored visualizations based on the current list of explored visualizations. Finally, the dashboard renderer takes the resulting visualizations to be included in the dashboard and perform any rendering preprocessing procedures for display and navigation of the dashboard as described in Section \ref{sec:navigation}.
\begin{figure}[ht!]
\centering
\includegraphics[width=\linewidth]{figures/system_architecture.png}
\caption{System Architecture of \system. User starts with x and y axes of interest and requests for $k$ visualizations in the dashboard. The request is processed by generating the lattice with the help of the querying module, visualization selection through the lattice traversal algorithms, and finally the dashboard is displayed at the frontend through the dashboard renderer. }%  The interaction manager translates the request to the traversal module that ???? [should we look at the offline case??]}
\label{system_architecture}
\end{figure}

\subsection{Algorithms\label{sec:algorithms}}
We give an overview of our algorithms by first discussing the approaches to generate the visualization lattice, and then presenting a high-level overview of our traversal algorithms.

\stitle{Lattice Generation.} Our system supports two variants of traversal algorithms based on the lattice generation procedure---offline variants that first generate the complete lattice and then work towards identifying the maximum utility solution, and online variants that incrementally generate the lattice and simultaneously identify the solution. The offline variants are appropriate for datasets with a small number of low-cardinality attributes, where we can generate the entire lattice in a reasonable time; whereas the online variants are appropriate for datasets with large number of high-cardinality attributes, where we incrementally generate a partial lattice.

%In most cases, the lattice contains a large number of visualizations due to the presence of many attributes or high-cardinality attributes in the dataset. In such cases finding an optimal solution is computationally challenging.

\stitle{Lattice Traversal.} Given the materialized lattice, the objective of the traversal algorithm is to find the connected subgraph in the lattice that has the maximum combined edge utility. Here, we discuss the \textit{frontier greedy} algorithm which is used for generating the dashboards for our user study and defer our discussion on the details of other algorithms that we have developed to the technical report.
% \begin{figure}[ht!]
% \centering
% \includegraphics[width=0.4\linewidth]{figures/frontier.pdf}
% \caption{Toy example demonstrating the notion of ``frontier''. Nodes that have been picked to include in the dashboard are colored green. The neighbors of the set of picked nodes are the frontier nodes, shown in pink. Grey nodes are other unpicked nodes in the lattice.}
% \end{figure}
%We devised two classes of heuristics algorithms, namely, frontier-based algorithms, and path-merging algorithms. These algorithms are guaranteed to find a solution that satisfies the constraints of our problem, except for the optimality.
\techreport{The frontier-based algorithms traverse the lattice from root to downwards, incrementally adding new nodes (visualizations) to the current solution (dashboard) till it reaches the maximum capacity $k$. To achieve this, the algorithms maintain a list of candidate nodes---called \textit{frontier} nodes---any of which can be added to the current solution since their informative parent is already present in the solution. At each step, the algorithms add a node from frontier to the current solution, and update the frontier accordingly.  The frontier based algorithms can be further categorized into three types based on their node selection strategy (from frontier), namely greedy algorithm, random walk algorithm, and probabilistic algorithm. The greedy algorithm picks the current best node from frontier (thus concentrates on exploitation), random walk algorithm picks a random node (thus concentrates on exploration), and probabilistic algorithm picks a random high-utility node (thus trades off between exploration and exploitation).}
\par As described in Algorithm \ref{algo:frontier_greedy}, our algorithm obtains a list of candidate nodes known as the \textit{frontier} nodes (pink in Figure\ref{fig:lattice} left), which encompasses all neighbors of nodes in the existing subgraph solution. Any of the nodes in the frontier can be added to the current solution since their informative parent is guaranteed to be present in the solution. The \texttt{getFrontier} function scans and adds all children of leaf nodes of the current dashboard as part of the frontier. In the online version, it additionally checks for each child whether its informative parent is present in the current dashboard. At each step, our algorithm greedily picks the node with the maximum utility from the frontier to the current solution, and updates the frontier accordingly.

\techreport{The path merging algorithm first generate the informative paths from root to every candidate node. Then, it greedily merges the paths with high-utility to create a subgraph whose size is less than or equal to maximum capacity $k$.}

% \begin{algorithm}
%     \SetKwInOut{Input}{Input}
%     \SetKwInOut{Output}{Output}
%     \Input{Precomputed Lattice of Visualizations, $G = \{V_1, \ldots, V_n\}$}
%     \Output{A Dashboard of Size $k$, $S$}
%     $S = \{ V_{root}\}$\;
%     $F = get\_child(V_{root})$\;
%     \While{$size(S) < k$}
%     {
%     	$s_{next} = pick\_next(F)$\;
%     	$S = S \cup s_{next}$\;
%       \For{$i = 0;\ i < size(S);\ i = i + 1$}
%       {
%           $F = (F \cup get\_child(S[i])) - S$\;
%       }
%     }
%     return $S$\;
%     \caption{Frontier Based Algorithm}
% \end{algorithm}

\begin{algorithm}
  \begin{algorithmic}[1]
  \Procedure{pickVisualizations}{k,lattice}
  \State dashboard $\gets$ \{ $V_{overall}$ \}
  \While{|dashboard| < k}
      \State frontier $\gets$ getFrontier(dashboard,lattice)
      \State maxNode $\gets$ getMaxUtilityNode(frontier)
      \State dashboard $\gets$ dashboard $\cup$ \{maxNode\}
  \EndWhile
  \Return dashboard
  \EndProcedure
  \end{algorithmic}
  \caption{Frontier Greedy Algorithm}\label{algo:frontier_greedy}
\end{algorithm}

%\textbf{Greedy Algorithms:} Greedy algorithms select the locally optimal node to be added to the frontier.

%A specific implementation would need to specify a scoring function to nodes in frontier that is used to pop out the next node in each iteration. One can design a scoring function based on the trade-off between performance and complexity. In the most simple case, we can use the edge weights to score nodes in the frontier. That is, at each point we add a node with the highest interestingness value. We note that this is quite a greedy approach. Specifically, we might miss visualizations with high utility that are in deeper levels of the graph. Thus, another approach would be to extent the horizon for which we calculate a nodes utility. We denote such approach as a look-ahead approach. With a free parameter $n$, we would like to assign a score to each frontier node the corresponds to the expected utility of adding this node and $n-1$ more nodes who are its descendants. For example, we can run BFP for each node in frontier treating it as a root.

\techreport{The path merging algorithm first generate the informative paths from root to every candidate node. Then, it greedily merges the paths with high-utility to create a subgraph whose size is less than or equal to maximum capacity $k$.}

\subsection{User Interaction\label{sec:interaction}}
\begin{figure}[ht!]
\centering
\includegraphics[width=\linewidth]{figures/overview.jpeg}
\caption{Overview of the \system interface for the Police Dataset~\cite{ctrp3}. Users can select x and y axes of interest, as well as a choice of an aggregation function. Default values are set for system related parameters such as the number of visualizations to show in the dashboard (k), iceberg condition for pruning ($\delta$), and informative parent criterion ($\theta$), which can be adjusted by the users via the sliders if needed.}
\label{fig:overview}
\end{figure}
\par Figure \ref{fig:overview} shows an overview of the \system interface. After the user selects the x and y axes of interest, aggregation function, and optional system parameter settings, an initial dashboard of $k$ visualizations is displayed on the canvas, such as the one seen in main canvas of Figure \ref{fig:overview}.  The system provides toolbar buttons with keyboard binding for zooming in, out, and extent, as well as moving around the canvas. Alternatively, users can zoom and pan with mouse click and scroll.

%\hdev{(1) The second sentence is in passive voice. (2) What are the optional system parameters? Clearly state them. (1) Simplify the third sentence. A non-UI person may not know the meaning of some of the terms. (1) In fourth sentence, if you are using "alternatively", there's no need for "also".}

\par After browsing through the visualizations in the dashboard, users may be interested in getting more information about a particular node. \system supports a mechanism for users to request additional summarizations based on a chosen visualization of interest. As shown in Figure \ref{fig:altroot_expansion} (left), the analyst starts with a 5-visualization dashboard on a police stop dataset~\cite{ctrp3}. The dataset contains  records of vehicle and pedestrian stops from law enforcement departments in Connecticut, dated from 2013 to 2015. The analyst learns that for the drivers who had contraband found in the vehicle, the arrest rate for drivers who are 60 and over is surprisingly higher than usual, whereas for Asian drivers the arrest rate is lower. In addition, he is also interested in learning more about the other factor that contribute to high arrest rate: duration=30+min. He clicks on the corresponding visualization and requests for 2 additional visualizations. Upon seeing the updated dashboard in Figure~\ref{fig:altroot_expansion} (right), he learns that similar to the selected visualization, any visualization that involves the duration=30+min filter results in high ticketing and arrest rates. This implies that if a police stop lasts more than 30 minutes, the outcome would more or less be the same, independent of other factors, such as driver's race or age. \system uses the same models and algorithms as before, except the root node is now set as the selected visualization, rather than the overall visualization. This node expansion capability is similarly motivated by the idea of \textit{iterative view refinement} in other visual analytics system \cite{Wongsuphasawat2016,Hoque2017}, which is essential for the users to iterate on and explore different hypotheses.

\begin{figure}[ht!]
\centering
\includegraphics[width=\linewidth]{figures/expansion_example.pdf}
\caption{Left: Original k=5 dashboard with the duration=30+min visualization clicked. A pop-up is displayed to submit the request for additional summary visualizations to be generated. Right: Resulting dashboard after requesting for 2 more visualizations based on the visualization of interest.}
\label{fig:altroot_expansion}
\end{figure}

\subsection{Assistive tools for visualizing large lattices\label{sec:navigation}}
Due to the amount of space occupied by the hierarchical layout when the number of visualizations gets large, we have developed tools to help users navigate through different parts of the dashboard interactively.
\stitle{Navigation Minimap:}  When the user zooms in on the dashboard, an overview mini-map is shown on the upper left-hand side of the canvas to help users identify which region of the dashboard they are currently exploring, as shown in Figure \ref{fig:hover_minimap}.
\stitle{Collapsed visualizations:}
One observation that we found across several datasets was that many visualizations had identical distributions, which resulted in lots of wasted space. Apart from their attribute name, these visualizations are not very informative for the users, therefore, we offer an option to collapse these visualization, as demonstrated in Figure \ref{fig:collapse_demo}. A visualization can be collapsed if it has more than one redundant sibling and does not have any children, so that there are no hidden stories due to lower-level dependencies. As shown in Figure \ref{fig:hover_minimap}, collapsed nodes can be easily identified by an orange border and the details of which visualizations are in the collapsed node are displayed when the user hovers over the visualization.
%\afterpage
\begin{figure}[ht!]
\centering
\includegraphics[width=\linewidth]{figures/minimap_zoom.png}
\caption{Zoomed-in version of Figure \ref{fig:collapse_demo} showing the labels of a collapsed visualization when user hovers over the visualization. The navigation minimap is shown in the top-left to help users navigate through the large dashboard.}
\label{fig:hover_minimap}
\end{figure}
