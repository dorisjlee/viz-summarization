\documentclass[sigchi]{acmart}
\settopmatter{printacmref=false} % Removes citation information below abstract
\renewcommand\footnotetextcopyrightpermission[1]{} % removes footnote with conference information in first column
\pagestyle{plain} % removes running headers
% Load basic packages
\usepackage{booktabs} % For formal tables
\usepackage{balance}  % to better equalize the last page
\usepackage{graphics} % for EPS, load graphicx instead
\usepackage[T1]{fontenc}
\usepackage{xcolor}
\usepackage{booktabs}
\usepackage{textcomp}
\usepackage{xspace}
\usepackage{setspace}
\usepackage[textsize=tiny]{todonotes}
% Some optional stuff you might like/need.
\usepackage{microtype} % Improved Tracking and Kerning
% \usepackage[all]{hypcap}  % Fixes bug in hyperref caption linking
\usepackage{ccicons}  % Cite your images correctly!
% \usepackage[utf8]{inputenc} % for a UTF8 editor only
\usepackage{verbatim}
\usepackage{relsize}
\usepackage{etoolbox}
\usepackage{lipsum}   % for filler text
\usepackage{setspace} % for \onehalfspacing and \singlespacing macros
\usepackage[normalem]{ulem}
\usepackage{xcolor}
\usepackage{fixltx2e}
\usepackage{amsmath}
\usepackage{amssymb}
\usepackage{afterpage}
\usepackage{microtype}                 % use micro-typography (slightly more compact, better to read)
\PassOptionsToPackage{warn}{textcomp}  % to address font issues with \textrightarrow
\usepackage{times}                     % we use Times as the main font
\renewcommand*\ttdefault{txtt}         % a nicer typewriter font
\usepackage{tabu}                      % only used for the table example
\usepackage{booktabs}                  % only used for the table example
% \usepackage[linesnumbered,ruled]{algorithm2e}
\usepackage{algorithm}
\usepackage[noend]{algpseudocode}

\usepackage{tikz}
\usetikzlibrary{shapes,arrows}
\usepackage{subfig}
\captionsetup{belowskip=-5pt,aboveskip=5pt}
% \BeforeBeginEnvironment{table}{\vskip-1ex}
% \AfterEndEnvironment{table}{\vskip-1ex}

\DeclareMathOperator*{\argmax}{arg\,max}
% llt: Define a global style for URLs, rather that the default one
\makeatletter
\def\url@leostyle{%
  \@ifundefined{selectfont}{
    \def\UrlFont{\sf}
  }{
    \def\UrlFont{\small\bf\ttfamily}
  }}
\makeatother

\newenvironment{denselist}{
    \begin{list}{\small{$\bullet$}}%
    {\setlength{\itemsep}{0ex} \setlength{\topsep}{0ex}
    \setlength{\parsep}{0pt} \setlength{\itemindent}{0pt}
    \setlength{\leftmargin}{1.5em}
    \setlength{\partopsep}{0pt}}}%
    {\end{list}}

\newcommand{\squishlist}{
   \begin{list}{$\bullet$}
    { \setlength{\itemsep}{0pt}
      \setlength{\parsep}{2pt}
      \setlength{\topsep}{0pt}
      \setlength{\partopsep}{0pt}
      \leftmargin=25pt
\rightmargin=0pt
\labelsep=5pt
\labelwidth=10pt
\itemindent=0pt
\listparindent=0pt
\itemsep=\parsep
    }
}
\newcommand{\squishend}{\end{list}}
\newcommand{\npar}{\par \noindent}
% use extensively to toggle between paper and TR
\newcommand{\eat}[1]{}
% \newcommand{\papertext}[1]{{\leavevmode\color{blue}{#1}}}
% \newcommand{\techreport}[1]{{\leavevmode\color{red}{#1}}}
\newcommand{\papertext}[1]{#1}
\newcommand{\techreport}[1]{}
\newcommand{\boldpara}[1]{\textbf{\paragraph{#1}}}
% de-facto paragraph format
\newcommand{\stitle}[1]{\par\noindent\textbf{#1}}
\newcommand{\tvcg}[1]{{\leavevmode\color{blue}{#1}}}
\newcommand{\cut}[1]{{\leavevmode\color{lightgray}{#1}}}
\newcommand{\ccut}[1]{} %confirmed cut
\newcommand{\system}{\textsc{Storyboard}\xspace}
\newcommand{\cluster}{\textsc{Cluster}\xspace}
\newcommand{\BFS}{\textsc{BFS}\xspace}
\def\plaintitle{\system : Hierarchical Summary of Equivalent Visualizations across Data Subsets}
\def\plainauthor{Doris Jung-Lin Lee*, Himel Dev*, Huizi Hu, Hazem Elmeleegy, Aditya Parameswaran}
\def\emptyauthor{}
\def\plainkeywords{Data visualization, exploratory data analysis, visual query, scientific data.}
\def\plaingeneralterms{Documentation, Standardization}

\newcommand{\agp}[1]{\textcolor{blue}{Aditya: #1}}
\newcommand{\dor}[1]{\textcolor{green}{Doris: #1}}
\newcommand{\hdev}[1]{\textcolor{magenta}{Himel: #1}}
\newcommand{\haz}[1]{\textcolor{orange}{Hazem: #1}}
\newcommand\notes[1]{\textcolor{red}{#1}}
\urlstyle{leo}
\AtBeginEnvironment{quote}{\small}
% To make various LaTeX processors do the right thing with page size.
\def\pprw{8.5in}
\def\pprh{11in}
\special{papersize=\pprw,\pprh}
\setlength{\paperwidth}{\pprw}
\setlength{\paperheight}{\pprh}
\setlength{\pdfpagewidth}{\pprw}
\setlength{\pdfpageheight}{\pprh}


% \setcopyright{licensedcagov}
%\setcopyright{cagovmixed}
%\setcopyright{licensedothergov}

% DOI
% \acmDOI{10.475/123_4}

% ISBN
% \acmISBN{123-4567-24-567/08/06}

%Conference
% \acmConference[WOODSTOCK'97]{ACM Woodstock conference}{July 1997}{El
%   Paso, Texas USA}
% \acmYear{1997}
% \copyrightyear{2016}

% \acmPrice{15.00}


%%%%%%%%%%%%%%%%%%%%%%%%%%%%%%%%%%%%%%%%%%%%%%%%%%%%%%%%%%%%%%%%
%%%%%%%%%%%%%%%%%%%%%% START OF THE PAPER %%%%%%%%%%%%%%%%%%%%%%
%%%%%%%%%%%%%%%%%%%%%%%%%%%%%%%%%%%%%%%%%%%%%%%%%%%%%%%%%%%%%%%%%

\begin{document}
\title{\system : Navigating Through Data Slices with Hierarchical Summary of Visualizations}
% \author{\plainauthor}
\begin{abstract}
The task of navigating through a large, multidimensional dataset is a common challenge in exploratory analysis. Due to limitations on the number of visualizations that an analyst can examine at one time, the narrow scope of drill-downs can often lead to inductive fallacies. %Not only is manual drill-down and roll-up on data subsets tedious and inefficient for the analyst, the massive space of data subsets, lack of interesting patterns in most data subsets, fallacies of spurious correlations, and pitfalls of statistical paradoxes calls for a systematic and effective way for analysts to make sense of and navigate through the large space of possible visualizations. 
In this paper, we present \system, an interactive visualization recommendation system provide safe guarantee during drill-down exploration by picking the proper visualization reference that leads to interesting and informative trends. Given a dataset and the x and y axes of interest, \system\ intelligently explores the lattice of equivalent visualizations across data subsets, and recommends interesting and informative visualizations. The recommended visualizations are then displayed in an interactive dashboard, where the visualizations are organized into a hierarchical layout. Our evaluation study shows that visualization dashboards generated by \system\ are interpretable and leads to higher performance in data analytic tasks compared to the competing baselines.
\end{abstract}
\keywords{exploratory data analysis, visualization recommendation.}
% \begin{teaserfigure}
%   \centering
%   \includegraphics[width=\linewidth]{figures/US_Election_Example.pdf}
%   \caption{A set of visualizations from the 2016 Election polls. These visualizations show the percentage of votes for three candidates (Donald Trump, Hilary Clinton, and Others) in different demographic groups (based on race and gender).}
%   \label{fig:elections_example}
% \end{teaserfigure}

\maketitle

% \begin{figure*}[bht]
% \centering
% \includegraphics[width=\linewidth]{figures/US_Election_Example.pdf}
% \caption{A set of visualizations from the 2016 Election polls. These visualizations show the percentage of votes for three candidates (Donald Trump, Hilary Clinton, and Others) in different demographic groups (based on race and gender).}
% \label{fig:elections_example}
% \end{figure*}

%!TEX root = main.tex
\section{Introduction}
%Exploring multidiemnsional dataset is hard
\par Visual data exploration is the \emph{de facto} first step in understanding multi-dimensional datasets. This exploration enables analysts to identify trends and patterns, generate and verify hypotheses, and detect outliers and anomalies. However, as datasets grow in size and complexity, visual data exploration ends up becoming challenging. In particular, to identify patterns that merit further investigation, an analyst may need to explore different subsets of the data to determine when and where certain patterns occur. Manually generating and examining each visualization in this space of data subsets (which grows exponentially in number of attributes) presents a major bottleneck in exploration.
%Drill-Down for exploration
\begin{figure}[ht!]
% \includegraphics[width=\linewidth]{figures/elections_example_lattice.pdf}
\includegraphics[width=\linewidth]{figures/elections_example_lattice_teaser.pdf}
\caption{Example data subset lattice from the 2016 US election dataset illustrating the drill-down fallacy along the purple path as opposed to the informative orange path.}
\label{fig:elections_example}
\end{figure}
\par One way of navigating this combinatorial space is to perform drill-downs on the space of data subsets (hereafter referred to as \emph{lattice}). For example, a campaign manager who is interested in understanding the voting patterns across different demographics (say, race, gender, or social class) using the 2016 US election exit polls~\cite{exitpolls} may first generate a bar chart for the entire population, where the x-axis shows the election candidates and the y-axis the percentage of votes for each of these candidates. In Figure~\ref{fig:elections_example}, the visualization at the top of the lattice corresponds to this overall population. They may then drill down to specific demographics of interest, say gender-based demographics, by generating bar charts for female voters, as shown in the second visualization at the second row of Figure~\ref{fig:elections_example}.
%Challenges associated with drill-down
\par There are three challenges associated with performing manual drill downs in this manner. First, it is often not clear which attributes to drill-down on. Analysts may use their intuition for choosing the drill-down attribute, but such arbitrary exploration may lead to large portions of the lattice being unexplored. Second, an uninformed path taken by analysts may lead to visualizations that are not very surprising or insightful. For example, an analyst may well end up wasting effort by drilling down from the \texttt{Black} visualization to the \texttt{Black Female} one in Figure~\ref{fig:elections_example}, since the two distributions are similar and therefore not very surprising. Last but most importantly, an analyst may encounter what we are calling the ``drill-down fallacy''. As shown in Figure~\ref{fig:elections_example}, an analyst can arrive at the \texttt{Black Female} visualization by either going through the purple or the orange drill-down path. An analyst who followed the purple path may be surprised at how drastically the \texttt{Black Female} voting behavior differs from that of the \texttt{Female}. This behavior is no longer surprising if the analyst had gone down the unsurprising orange path that we saw earlier, where the proper reference (i.e., the vote distribution for \texttt{Black}) explains the vote distribution for \texttt{Black Female}. In other words, even though the vote distribution for \texttt{Black Female} is very different from that of \texttt{Female}, the phenomenon can be explained by a more general ``root cause'' attributed to the voting behavior for the \texttt{Black} community. Attributing an overspecific cause to an effect, while ignoring the actual cause, not only leads to less interpretable explanations for the observed visualizations, but can also be detrimental to decision-making. For example, for the campaign manager, this could lead to a misallocation of campaign funds.
\par The aforementioned example demonstrates the \emph{drill-down fallacy}---incomplete insights that result from potentially confounding factors not explored along a drill-down path. In particular, while performing drill-downs on randomly selected paths, analysts may find a ``local difference'' in trends, without being aware of the more ``general phenomenon'' that could explain the trend of interest. Without the proper parent reference visualization that explains the behavior of the visualization of interest, analysts are at risk of falling prey to the drill-down fallacy. A naive solution to avoid this fallacy is to explore all potential drill-down paths. Unfortunately, this approach does not scale with the increasing number of factors in the drill-down path.
\par In this paper, we present a visual data exploration tool, titled \system, that addresses the three aforementioned challenges of exploration through three principles: (i) \textbf{Safety} (i.e., ensure that proper informative references are present to avoid drill-down fallacies), (ii) \textbf{Saliency} (i.e., identify interesting visualizations that convey new information or insights), and (iii) \textbf{Summarization} (i.e., succinctly convey the key insights present in a dataset). To facilitate safety, we develop a notion of \emph{informativeness}---the capability of a reference visualization to explain the visualization of interest. To facilitate saliency, we characterize the notion of \emph{interestingness}---the difference between a visualization and its informative reference in terms of underlying data distribution. Finally, to facilitate summarization, we embrace a \emph{collective} measure of visualization utility by recommending a connected network of visualizations that collectively offer informative insights. Based on these three principles, our tool, \system, automatically identifies a network of visualizations that succinctly conveys the key informative insights in a dataset. Our user study results demonstrate that our tool can guide an analyst towards meaningful insights for a variety of tasks. Our contributions include:
\begin{denselist}
\item Identifying and characterizing the notion of ``drill-down fallacy'', a common fallacy that have not yet been studied extensively in the past.
\item Introducing the novel concept of \emph{informativeness} that helps users identify meaningful insights that arise from something \textit{actually interesting} about the data (instead of confounding variables),
\item Designing a system, \system, that automatically identifies a network of visualizations that succinctly conveys the key informative insights in a dataset,
\item Demonstrating the efficacy of our system through a user study evaluation on how well users can retrieve interesting visualizations, judge the importance of attributes, and predict unseen visualizations, against two other summarization baselines.
\end{denselist}
%!TEX root = main.tex
\section{Data and User Models\label{sec:datamodel}}
\par In this section, we first describe how analysts explore the lattice through drill-downs and introduce a common fallacy that arises when analyst have limited time and attention to examine all possible factors that contribute to the observed visualization. Then, we discuss how to resolve the problem of finding informative visualizations for a given visualization.
% How users explore visualizations
% 	- Drill down
% 	- Expectation formation
% 		- focussing on bar chart 
% to explore the space of possible data subset
\par Research in visualization storytelling shows that people prefer visualization sequences structured hierarchically with increasing levels of aggregation~\cite{Kim2017,Hullman2017,Hullman2013}. In order to find the desired data subset, analysts often drill-down to explore data at different levels of granularity by adding one filter at a times. For each data subset that he encounters, he may want to visualize the distribution of measure values for each data subset through a bar chart. When analysts perform drill-downs, they naturally formulate their expectation based on the last visualization that they observe, known as the `parent', which is the visualization that can be obtained by removing one filter constraint from the current visualization in context (known as the `child' visualization). For example in Figure~\ref{fig:elections_example}, the visualizations Female and Black are the parents of the Black Female visualization. By extending this concept of parent-child relationships, we can organize the space of visualization from different data subsets to form a lattice as shown in Figure~\ref{fig:elections_example}.
\begin{figure}[h!]
\includegraphics[width=\linewidth]{figures/elections_example_lattice.pdf}
\caption{Example data subset lattice illustrating the misleading factor fallacy along the orange path as opposed to the informative purple path.}
\label{fig:elections_example}
\end{figure}
% Fallacies of Forming Expectations:
% 	- two extremes: 
% 		- random parent v.s. exhaustive parent browsing
% 		- limitation of analyst 
\par As the analyst is drilling down by adding one filter at a time, the analyst is prone to be misguided by parent visualizations that highly deviate from its child, overlooking other potential factors that may explain the seemingly-anomalous behavior. We refer to this phenomena as \emph{drill-down fallacy}, as this type of fallacy arises from the inductive nature of the drill-down operation. We demonstrate this fallacy with an example from the 2016 US Elections exit polls dataset. As shown in Figure~\ref{fig:elections_example}, an analyst can either arrive at the Black Females visualization by going through the purple path or the orange path. At random, if the analyst went down the purple path, he may be surprised at how much the Black Female voting behavior differs drastically from the vote distribution for females. This behavior can be explained if the analyst went down the orange path, where he sees the proper reference (vote distribution for Black) that explains the behavior of the Black Female distribution. While such fallacies can be prevented if the analyst browses through all possible parents of any visualization that he observes in the dataset, the prohibitively large number of visualizations and limited memory and attention of analysts make this task impractical.
% Problem definition 
% 	- picking right parent
\par Since it is impossible to examine all possible parents and potentially misleading if we simply picked a few parents to examine, our goal is to develop a mechanism that would  \emph{provide safe guarantee by picking the proper informative parent} as a reference when analysts navigate through the space of data subsets.  To model the informativeness of an observed parent in the context of an unseen visualization, we characterize the capability of the parent in predicting the unseen visualization. An observed parent is \emph{informative} if its data distribution closely follows the data distribution of the unseen child visualization, since the visualization helps the analyst form an accurate mental picture of what to expect from the unseen visualization. Specifically, we formulate the informativeness of an observed parent $V_i^j$ of an unseen visualization $V_i$ as the similarity between their data distributions measured using a distance function $D(V_i, V_i^j)$. The most informative parents $V_i^*$ of an unseen visualization $V_i$ are the ones whose data distributions are most similar to the unseen.
\begin{equation}
    V_i^*=\underset{V_i^j}{argmin}\ D(V_i, V_i^j)
\end{equation}
We regard a visualization as informative if its distance falls within a user-defined threshold $\theta\%$ close to its most informative parent:
\begin{equation}
    V_i^{*, \theta} = \{V_i^j : \frac{D(V_i, V_i^*)}{D(V_i, V_i^j)} \geq \theta\}
\end{equation}
For example in Figure~\ref{fig:elections_example}, while both visualization Black and Female visualizations are considered parents of the Black Female visualization, only the Black visualization are considered the informative parent of the black female population, for any values of $\theta \geq 11\%$ via the Euclidean distance metric. Note that, our proposed system can work with different distance metrics such as cosine similarity and earth mover's distance. Without loss of generality, we chose to use Euclidean distance metric for the remainder of our paper.

% %!TEX root = main.tex
\section{System\label{sec:system}}
\subsection{System Objective}
%the visualization corresponding to black voters is the most informative parent of the visualization corresponding to black female voters. For $\theta <= 0.11$, the former remains the only informative parent of the latter.

\stitle{Interestingness:} While informative parents contribute to the prediction of an unseen visualization, the most interesting visualizations to recommend are those for which \emph{even the informative parents fail to accurately predict the visualization}. \dor{Can we justify this based on our findings?} To model the interestingness of an unseen visualization $V_i$ in the context of an observed parent $V_i^j$, we characterize the deviation between their data distributions using a distance function $D(V_i, V_i^j)$. The unseen visualizations whose data distributions deviate from the observed informative parents are \emph{interesting}. \cut{The most interesting unseen visualizations $V_\#$ are the ones that deviate most from their observed informative parents.
\begin{equation}
    V_\#=\underset{V_i}{argmax} \ D(V_i, V_i^{*, \theta})
\end{equation}
In Figure \ref{fig:elections_example}, the most interesting visualization to recommend is the one corresponding to white female voters. This visualization significantly differs from its informative parent---the visualization corresponding to female voters.} \dor{The argmax notation not necessary since we're just using this in our utility function. The election example is not convincing, the informative parent of white female is actually white and not female. Also the differences are not too significant.}

%\noindent Additional model extensions can be added to this objective function based user specification. For example, there may be $k$ visualizations that approximately yield equal contribution to the user's expectation. For simplicity of notation, we have assumed $k=1$ in the aforementioned model. In order, a user may want to prevent the recommendation of spuriously interesting subsets of the data. We can discard visualizations that falls below a certain subpopulation size threshold.
\stitle{Subpopulation size consideration:} The danger of spurious patterns and correlations in visualizations that contain small subpopulation size is a well-known problem in exploratory analysis~\cite{Binnig2017}. We take two preventive measures to avoid including such misleading visualization in our dashboards. First, in the lattice generation process discussed in Section~\ref{sec:algorithms}, we allow users to select an `iceberg condition' \footnote{The terminology is used in the discussion of iceberg cubes in OLAP literature~\cite{Xin2007}.} ($\delta$) to adjust the extent of pruning on visualizations whose sizes fall below a certain percentage of the overall population size. Second, we downweigh the interestingness edge utility $U(V_i, V_i^j)$ between a parent $V_i^j$ and a child visualization $V_i$ by the ratio of their sizes:
\begin{equation}
    U(V_i, V_i^j) = \frac{|V_i|}{|V_i^{j}|} \cdot D(V_i, V_i^j)
    \label{edge_utility}
\end{equation}

Given the lattice data model and the user model for visualization utility described above, the goal of our system is to generate a dashboard by selecting $k$ visualizations from the lattice. We enforce that the generated dashboard satisfies several requirements:
 \begin{enumerate}
  \item Dashboard must include the overall visualization (topmost visualization with no filter applied) to serve as reference to the rest of the visualizations in the dashboard.
  \item For each visualization except for the overall, at least one of its informative parents is included within the $k$ visualizations. This excludes the uninformative parents as exemplified in black female example in the dashboard, especially since our findings 3 and 4 show that showing multiple, improper parents can mislead the participants, resulting in a higher variance across their estimations. %This enforces that every visualization shown in the dashboard has an informative reference to compare against to create a connected story.
  \item The selected $k$ visualizations are collectively most ``interesting'' in presence of their informative parents as measured by the utility in Equation \ref{edge_utility}.
\end{enumerate}
 The problem of finding a connected subgraph in the lattice that has the maximum combined edge utility is  known as the maximum-weight connected subgraph problem~\cite{ErnstAlthaus2009} and is known to be NP-Complete, via a reduction from the \textsc{Clique Problem}~\cite{Parameswaran2010}. In Section~\ref{sec:algorithms}, we discuss heuristic algorithms used for deriving a locally optimal solution for ensuring interactive runtime.

\subsection{System Architecture}
We have implemented \system\ as a Flask web application on top of a PostgreSQL database. In Figure~\ref{system_architecture}, we present the system architecture of \system, which consists of three core modules: the traversal module, the query module, and the statistics module. The interaction manager deals with the supported user interaction described in Section~\ref{sec:interaction} and sends a request to the lattice module which  contains several algorithms for generating and traversing the visualization lattice described in Section~\ref{sec:algorithms}. For generating the visualization lattice, the lattice module passes a list of data subsets corresponding to visualizations to be generated to the query module. The query module translates these visualizations into queries, and then optimizes (by grouping) and executes the queries. The statistics module is an optional module that allows the lattice module to prune low-utility visualizations without actually generating them. Specifically, it generates coarse statistics for the unexplored visualizations based on the current list of explored visualizations. Finally, the dashboard renderer takes the resulting visualizations to be included in the dashboard and perform any rendering preprocessing procedures for display and navigation of the dashboard as described in Section \ref{sec:navigation}.
\begin{figure}[ht!]
\centering
\includegraphics[width=\linewidth]{figures/system_architecture.png}
\caption{System Architecture of \system. User starts with x and y axes of interest and requests for $k$ visualizations in the dashboard. The request is processed by generating the lattice with the help of the querying module, visualization selection through the lattice traversal algorithms, and finally the dashboard is displayed at the frontend through the dashboard renderer. }%  The interaction manager translates the request to the traversal module that ???? [should we look at the offline case??]}
\label{system_architecture}
\end{figure}

\subsection{Algorithms\label{sec:algorithms}}
We give an overview of our algorithms by first discussing the approaches to generate the visualization lattice, and then presenting a high-level overview of our traversal algorithms.

\stitle{Lattice Generation.} Our system supports two variants of traversal algorithms based on the lattice generation procedure---offline variants that first generate the complete lattice and then work towards identifying the maximum utility solution, and online variants that incrementally generate the lattice and simultaneously identify the solution. The offline variants are appropriate for datasets with a small number of low-cardinality attributes, where we can generate the entire lattice in a reasonable time; whereas the online variants are appropriate for datasets with large number of high-cardinality attributes, where we incrementally generate a partial lattice.

%In most cases, the lattice contains a large number of visualizations due to the presence of many attributes or high-cardinality attributes in the dataset. In such cases finding an optimal solution is computationally challenging.

\stitle{Lattice Traversal.} Given the materialized lattice, the objective of the traversal algorithm is to find the connected subgraph in the lattice that has the maximum combined edge utility. Here, we discuss the \textit{frontier greedy} algorithm which is used for generating the dashboards for our user study and defer our discussion on the details of other algorithms that we have developed to the technical report.
% \begin{figure}[ht!]
% \centering
% \includegraphics[width=0.4\linewidth]{figures/frontier.pdf}
% \caption{Toy example demonstrating the notion of ``frontier''. Nodes that have been picked to include in the dashboard are colored green. The neighbors of the set of picked nodes are the frontier nodes, shown in pink. Grey nodes are other unpicked nodes in the lattice.}
% \end{figure}
%We devised two classes of heuristics algorithms, namely, frontier-based algorithms, and path-merging algorithms. These algorithms are guaranteed to find a solution that satisfies the constraints of our problem, except for the optimality.
\techreport{The frontier-based algorithms traverse the lattice from root to downwards, incrementally adding new nodes (visualizations) to the current solution (dashboard) till it reaches the maximum capacity $k$. To achieve this, the algorithms maintain a list of candidate nodes---called \textit{frontier} nodes---any of which can be added to the current solution since their informative parent is already present in the solution. At each step, the algorithms add a node from frontier to the current solution, and update the frontier accordingly.  The frontier based algorithms can be further categorized into three types based on their node selection strategy (from frontier), namely greedy algorithm, random walk algorithm, and probabilistic algorithm. The greedy algorithm picks the current best node from frontier (thus concentrates on exploitation), random walk algorithm picks a random node (thus concentrates on exploration), and probabilistic algorithm picks a random high-utility node (thus trades off between exploration and exploitation).}
\par As described in Algorithm \ref{algo:frontier_greedy}, our algorithm obtains a list of candidate nodes known as the \textit{frontier} nodes (pink in Figure\ref{fig:lattice} left), which encompasses all neighbors of nodes in the existing subgraph solution. Any of the nodes in the frontier can be added to the current solution since their informative parent is guaranteed to be present in the solution. The \texttt{getFrontier} function scans and adds all children of leaf nodes of the current dashboard as part of the frontier. In the online version, it additionally checks for each child whether its informative parent is present in the current dashboard. At each step, our algorithm greedily picks the node with the maximum utility from the frontier to the current solution, and updates the frontier accordingly.

\techreport{The path merging algorithm first generate the informative paths from root to every candidate node. Then, it greedily merges the paths with high-utility to create a subgraph whose size is less than or equal to maximum capacity $k$.}

% \begin{algorithm}
%     \SetKwInOut{Input}{Input}
%     \SetKwInOut{Output}{Output}
%     \Input{Precomputed Lattice of Visualizations, $G = \{V_1, \ldots, V_n\}$}
%     \Output{A Dashboard of Size $k$, $S$}
%     $S = \{ V_{root}\}$\;
%     $F = get\_child(V_{root})$\;
%     \While{$size(S) < k$}
%     {
%     	$s_{next} = pick\_next(F)$\;
%     	$S = S \cup s_{next}$\;
%       \For{$i = 0;\ i < size(S);\ i = i + 1$}
%       {
%           $F = (F \cup get\_child(S[i])) - S$\;
%       }
%     }
%     return $S$\;
%     \caption{Frontier Based Algorithm}
% \end{algorithm}

\begin{algorithm}
  \begin{algorithmic}[1]
  \Procedure{pickVisualizations}{k,lattice}
  \State dashboard $\gets$ \{ $V_{overall}$ \}
  \While{|dashboard| < k}
      \State frontier $\gets$ getFrontier(dashboard,lattice)
      \State maxNode $\gets$ getMaxUtilityNode(frontier)
      \State dashboard $\gets$ dashboard $\cup$ \{maxNode\}
  \EndWhile
  \Return dashboard
  \EndProcedure
  \end{algorithmic}
  \caption{Frontier Greedy Algorithm}\label{algo:frontier_greedy}
\end{algorithm}

%\textbf{Greedy Algorithms:} Greedy algorithms select the locally optimal node to be added to the frontier.

%A specific implementation would need to specify a scoring function to nodes in frontier that is used to pop out the next node in each iteration. One can design a scoring function based on the trade-off between performance and complexity. In the most simple case, we can use the edge weights to score nodes in the frontier. That is, at each point we add a node with the highest interestingness value. We note that this is quite a greedy approach. Specifically, we might miss visualizations with high utility that are in deeper levels of the graph. Thus, another approach would be to extent the horizon for which we calculate a nodes utility. We denote such approach as a look-ahead approach. With a free parameter $n$, we would like to assign a score to each frontier node the corresponds to the expected utility of adding this node and $n-1$ more nodes who are its descendants. For example, we can run BFP for each node in frontier treating it as a root.

\techreport{The path merging algorithm first generate the informative paths from root to every candidate node. Then, it greedily merges the paths with high-utility to create a subgraph whose size is less than or equal to maximum capacity $k$.}

\subsection{User Interaction\label{sec:interaction}}
\begin{figure}[ht!]
\centering
\includegraphics[width=\linewidth]{figures/overview.jpeg}
\caption{Overview of the \system interface for the Police Dataset~\cite{ctrp3}. Users can select x and y axes of interest, as well as a choice of an aggregation function. Default values are set for system related parameters such as the number of visualizations to show in the dashboard (k), iceberg condition for pruning ($\delta$), and informative parent criterion ($\theta$), which can be adjusted by the users via the sliders if needed.}
\label{fig:overview}
\end{figure}
\par Figure \ref{fig:overview} shows an overview of the \system interface. After the user selects the x and y axes of interest, aggregation function, and optional system parameter settings, an initial dashboard of $k$ visualizations is displayed on the canvas, such as the one seen in main canvas of Figure \ref{fig:overview}.  The system provides toolbar buttons with keyboard binding for zooming in, out, and extent, as well as moving around the canvas. Alternatively, users can zoom and pan with mouse click and scroll.

%\hdev{(1) The second sentence is in passive voice. (2) What are the optional system parameters? Clearly state them. (1) Simplify the third sentence. A non-UI person may not know the meaning of some of the terms. (1) In fourth sentence, if you are using "alternatively", there's no need for "also".}

\par After browsing through the visualizations in the dashboard, users may be interested in getting more information about a particular node. \system supports a mechanism for users to request additional summarizations based on a chosen visualization of interest. As shown in Figure \ref{fig:altroot_expansion} (left), the analyst starts with a 5-visualization dashboard on a police stop dataset~\cite{ctrp3}. The dataset contains  records of vehicle and pedestrian stops from law enforcement departments in Connecticut, dated from 2013 to 2015. The analyst learns that for the drivers who had contraband found in the vehicle, the arrest rate for drivers who are 60 and over is surprisingly higher than usual, whereas for Asian drivers the arrest rate is lower. In addition, he is also interested in learning more about the other factor that contribute to high arrest rate: duration=30+min. He clicks on the corresponding visualization and requests for 2 additional visualizations. Upon seeing the updated dashboard in Figure~\ref{fig:altroot_expansion} (right), he learns that similar to the selected visualization, any visualization that involves the duration=30+min filter results in high ticketing and arrest rates. This implies that if a police stop lasts more than 30 minutes, the outcome would more or less be the same, independent of other factors, such as driver's race or age. \system uses the same models and algorithms as before, except the root node is now set as the selected visualization, rather than the overall visualization. This node expansion capability is similarly motivated by the idea of \textit{iterative view refinement} in other visual analytics system \cite{Wongsuphasawat2016,Hoque2017}, which is essential for the users to iterate on and explore different hypotheses.

\begin{figure}[ht!]
\centering
\includegraphics[width=\linewidth]{figures/expansion_example.pdf}
\caption{Left: Original k=5 dashboard with the duration=30+min visualization clicked. A pop-up is displayed to submit the request for additional summary visualizations to be generated. Right: Resulting dashboard after requesting for 2 more visualizations based on the visualization of interest.}
\label{fig:altroot_expansion}
\end{figure}

\subsection{Assistive tools for visualizing large lattices\label{sec:navigation}}
Due to the amount of space occupied by the hierarchical layout when the number of visualizations gets large, we have developed tools to help users navigate through different parts of the dashboard interactively.
\stitle{Navigation Minimap:}  When the user zooms in on the dashboard, an overview mini-map is shown on the upper left-hand side of the canvas to help users identify which region of the dashboard they are currently exploring, as shown in Figure \ref{fig:hover_minimap}.
\stitle{Collapsed visualizations:}
One observation that we found across several datasets was that many visualizations had identical distributions, which resulted in lots of wasted space. Apart from their attribute name, these visualizations are not very informative for the users, therefore, we offer an option to collapse these visualization, as demonstrated in Figure \ref{fig:collapse_demo}. A visualization can be collapsed if it has more than one redundant sibling and does not have any children, so that there are no hidden stories due to lower-level dependencies. As shown in Figure \ref{fig:hover_minimap}, collapsed nodes can be easily identified by an orange border and the details of which visualizations are in the collapsed node are displayed when the user hovers over the visualization.
%\afterpage
\begin{figure}[ht!]
\centering
\includegraphics[width=\linewidth]{figures/minimap_zoom.png}
\caption{Zoomed-in version of Figure \ref{fig:collapse_demo} showing the labels of a collapsed visualization when user hovers over the visualization. The navigation minimap is shown in the top-left to help users navigate through the large dashboard.}
\label{fig:hover_minimap}
\end{figure}

% %!TEX root = main.tex
\section{User Study Evaluation\label{sec:userstudy}}
% We evaluate the utility of \system by performing a between-subject user study focusing 
In this section, we describe the methodology and results for a between-subject user study we have conducted for evaluating the utility of \system. To assess the efficacy of \system across various exploratory analysis goals, we focus on addressing the following research questions:
\begin{denselist}
	\item RQ1: How \textit{interesting} are the visualizations in the dashboard perceived subjectively by the users?
	\item RQ2: How well do the dashboard \textit{summarize} the relative importance of different attributes within a given dataset?
	\item RQ3: How \textit{informative} are the visualizations in the dashboard at providing users with an accurate understanding of unseen child visualizations? %effective is our tool at guiding users towards safe and informative visualization references? 
	% in providing analysts with task-specific insights? (including identifying important features for prediction tasks and estimating the distribution of an unseen visualization)
	% \item RQ3: How useful are the visualizations in the recommended dashboard to analysts?
\end{denselist}
These research questions roughly correspond to the three S's in our objective---saliency, summarization, and safety.
\subsection{Methods}
We recruited 18 participants with prior experience in working with data. Participants included undergraduate and graduate students, researchers, and data scientists, with 1 to 14 years of data analysis experience (average = 5.61).  %This can include, but are not limited to, browsing and reading data, data cleaning and wrangling, data visualization and model building. The inclusion criteria is assessed based on a self-reporting basis in the pre-study survey.
%an average of 5.61 years of experience working with data.
There were 8 female and 10 male participants. No participants reported prior experience in working with the two datasets used in the study (described below). Participants were randomly assigned two of the three types of dashboards with k=10 visualizations generated by following conditions. 
%(described in Section \ref{sec:algorithms})
\stitle{\system:} The dashboards for this condition are generated by the aforementioned frontier greedy algorithm and displayed in a hierarchical layout (as seen in Figure~\ref{fig:overview}). To ensure the informativeness of the generated dashboards, we selected a more stringent $\theta$=90\% criteria to generate the dashboards for our user study. In order to establish a fair comparison with the two other conditions, we deactivated iceberg pruning (by setting $\delta$=0) and the interactive node expansion capabilities.
\stitle{\BFS (short for breadth-first search):} Starting from the visualization of the entire population, $k$ visualizations are selected level-wise, traversing down the subset lattice, adding the visualizations at the first level with 1-filter combination one at a time, proceeding with the 2-, 3-, and so on, until $k$ visualizations have been added to the dashboard. This baseline is designed to simulate a dashboard generated by a meticulous analyst who exhaustively inspects all possible visualizations (i.e., filter combinations) from the top-down. The chosen visualizations are displayed in a 5x2 table in the traversed order.
\stitle{\cluster:} K-Means clustering is performed on the data distributions of all possible visualizations for the dataset. This results in $k$ clusters covering all visualizations of the dataset, corresponding to $k$, the number of visualizations to be shown in the dashboard. For each representative cluster, we select the visualization with the least number of filter conditions for interpretability and display them in a 5x2 table layout. Since the clusters cover all visualizations in the dataset and the overall visualization has the minimum number of filter across all visualization, the overall visualization is guaranteed to be picked as one of the displayed visualizations. This baseline is designed to showcase a diverse set of pattern distributions within the dataset.
%K-Means clustering is performed on the dataset with $k$ clusters, corresponding to $k$, the number of visualizations to be shown in the dashboard. For each representative cluster, we select the visualization with the least number of filter conditions for interpretability\footnote{Due to this requirement, the overall visualization is guaranteed to be picked as one of the displayed visualizations.} and display them in a 5x2 table layout. This baseline is designed to showcase a diverse set of pattern distributions within the dataset.
\par Each participant was assigned two different conditions on two different datasets. The ordering of each condition was randomized to prevent confounding learning effects. The study began with a 5-minute tutorial using dashboards generated from the Titanic dataset~\cite{titanic} for each condition. To prevent bias across conditions, participants were not provided an explanation of how the dashboards were generated and why the visualizations were arranged in a particular way. Then, participants proceeded onto the aforementioned Police Stop dataset. The attributes in the dataset include driver gender, age, race, stop time of day, stop outcome, whether a search was conducted, and whether contraband was found. We generated dashboards of bar chart visualizations with x-axis as the stop outcome (i.e., whether the police stop resulted in a ticket, warning, or arrest) and y-axis as the percentage of police stops that led to each outcome. %We randomize the ordering for each task combination to prevent confounding learning effects. %%, which contains visualizations of the \% of police stop that resulted in a warning, ticket, or an arrest. %, which contains a total of 312948 records of vehicle and pedestrian stops from law enforcement departments in Connecticut, dated from 2013 to 2015. 
\par The second dataset in the study is the Autism dataset~\cite{autism}, which includes the result of autism spectrum disorder screening for 704 adults. The attributes in the dataset are binary responses to 10 diagnostic questions that are part of the screening process. This dataset serves as a data-agnostic condition, since there was no descriptions of the questions or answer labels provided to the user. We generate dashboard visualizations based on whether the participant is diagnosed with autism or not. 
\par Participants were given some time to read through a worksheet containing descriptions of the data attributes. Then, they were given an attention check question where they were given a verbal description of the visualization filter and asked about the distributions for the corresponding visualization in the dashboard. After understanding the dataset and chart schema, participants were asked to accomplish the following tasks in the prescribed order below:
\stitle{Retrieval:} Participants were asked to talk aloud as they interpreted the visualizations in the dashboard and mark each visualization as either interesting, not interesting, or leave it as unselected. This task was intended to measure how interesting were the selected visualizations to participants (RQ1).%well are participants at retrieving interesting visualizations (RQ1).

\stitle{Attribute Ranking:} Participants were given a sheet of paper with all the attributes listed and asked to rank the attributes in order of importance in contributing to a particular outcome (e.g., factors leading to an arrest or autism diagnosis). Participants were allowed to assign equal ranks to more than one attribute or skip attributes that they were unable to infer importance for. Attribute ranking tasks are common in feature selection and other data science tasks. The goal of this task was to measure how well participants understood the relative importance of each attribute in contributing towards an outcome (RQ2).

\stitle{Prediction:} Participants were given a separate worksheet and asked to sketch an estimate for a visualization that is not present in the dashboard. For every condition, the visualization to be estimated contained 2 filter combinations, with exactly one parent present in the given dashboard. After making the prediction, participants were shown the actual data distribution and asked to rate on a Likert scale of 10 how surprising the result was (where 1 is not surprising and 10 is very surprising). The prediction task measured how accurate participants are at predicting an unseen visualization, estimating how well they understood key informative insights that influences other distributions from the dataset (RQ3).
% \stitle{Deep Prediction:} This task is similar to the shallow prediction, except that the visualization to be estimated is ``deep'' in the sense that it has 3 filter combinations, with only one parent in the given dashboard. Both prediction tasks measure how accurate participants are at predicting an unseen visualization (RQ3).
\par We repeated the same study procedure described above for the Autism dataset. At the end of the study, we asked two open-ended questions regarding the insights that participants have learned and what they like or dislike about each dashboard. On average, the study lasted around 48 minutes.
%The user study is composed of two phases. Phase one of the experiment focuses on comparing our tool against a set of baselines intended to simulate the natural sequence of visualizations that an analyst would encounter through various approaches during exploratory analysis. The baselines include:
%To prevent learning effects, the ordering of the baselines will be randomized across users.
% \par At the beginning of the study, participants were provided with a dashboard from an example dataset, as well as an explanation of how the dashboard is generated. For each of the visualization dashboard, participants are asked to mark visualizations as interesting/not interesting while explaining their reasoning for each annotation. Then, they are asked to summarize a list of insights that they have discovered after browsing through all visualizations in the dashboard. Participants also answer a set of task-specific questions related to causality and outliers[?], in the form as shown in the example [*]. These tasks are repeated for all baselines and our tool in randomized order on different datasets to prevent learning effects. At the end of phase one, participants are asked to comment on their experiences with each method, as well as the pros and cons of each of the tools. This phase of the experiment is designed to quantify the effects of RQ 1 and 2. In the end, we ask participants to discuss the interesting insights drawn from looking at the recommended dashboards as well as  ------.
%\par To prevent fatigue, after a 5 minute break, the participants then proceed onto phase two of the study, where they are given [10] dashboards generated by our tool and are asked to engage in a talk-aloud exercise as they browse the recommended visualizations. This is a more open-ended study intended for addressing RQ3 that can reveal our tool show unimportant results across different datasets and/or highlight larger selection of the types of insights that can be generated from the tool.
\subsection{Quantitative Results}
% In order to evaluate the efficacy of our system against the two baselines, we will first examine the quantitative results to address RQ1 and RQ2 and then discuss the qualitative findings to address RQ3.
%\dor{In general, we might have to make better connection between the RQs and the study results.}
\stitle{Retrieval (RQ1):} Using the click-stream data logged from the user study, we recorded whether a participant marked a visualization in the dashboard as interesting, not interesting, or left the visualization unselected. %Since we do not have objective ground truths of which visualization is interesting or not, we use the participant's consensus to come up with a score for each visualization. In this scoring scheme, if visualization is marked as interesting, that visualization receives a score of 1; if a visualization is marked as uninteresting, the visualization incurs a penalty of -0.5\footnote{The reason why we chose a lower penalty score for disinterested clicks than the interested click reward is that some of the participants did not chose to mark visualizations that they thought were uninteresting as disinterested explicitly and chose to simply keep them unselected.}; no score is assigned if the visualization is unselected, then the scores are summed over all users who have seen the visualization. Each user is then assigned a score based on the product of their retrieval score and the consensus score (i.e. user would receive a higher score if selected visualization was highly ranked by consensus).
% Since interestingness is a subjective measure, 
% %Since we do not have a objective ground truth on which visualization is interesting or not interesting, 
% we devise a popularity-based metric that measures how interesting a visualization is amongst all participants. We assign the selection made by user j to visualization i with score $\delta_{ij}$ of 1 if a user is interested, 0 if they leave it unselected, and -1 if they are not interested. 
% %Here i indexes the visualization and j indexes the user. 
% % As shown in Equation~\ref{weighting}, w
% % \begin{equation}\label{weighting}
% % 	\delta_{ij}= \left\{\begin{matrix}
% % 	 1& \textrm{interested}
% % 	\\ 0 & \textrm{unselected}
% % 	\\ -1 & \textrm{not interested}
% % 	\end{matrix}\right.
% % \end{equation}
% We obtain a consensus score for each visualization to measure how frequently the visualization is regarded as interesting by summing over all users' vote on that visualization.
% \begin{equation}\label{vote}
% \textrm{consensus}(V_i) =\sum_{j\in user} \delta_{ij}
% \end{equation}
% Given a consensus measure of how interesting a visualization is, we can define a rating score which measures how good a particular user's rating is, by taking the product of the consensus interestingness score and the rating value, as shown in Equation \ref{rank}. Intuitively, a rating should be rewarded more if it has retrieved interesting visualization agreed by many other users, likewise, ratings that does not retrieve such visualizations should be penalized more heavily.
% describes how the rating score (which measures how good the user's particular rating is) is the product of consensus score (how frequently is a visualization regarded as interesting?) and the rated value ($\delta_{ij}$).
% \begin{equation}\label{rank}
% \textrm{rating score}(V_{ij}) =\textrm{consensus}(V_i) \cdot \delta_{ij}
% \end{equation}
Table~\ref{table:interestingScore} summarizes counts of visualizations marked as interesting or not interesting aggregated across conditions. We also normalize the interestingness count by the total number of selected visualizations to account for variations in how some participants select more visualizations than others. The results indicate that participants who used \system\ had more visualizations that they found interesting compared to the \BFS and \cluster condition. This result indicates that \system's \textit{saliency} objective was able to select visualizations that were perceived as interesting to the users. 
\begin{table}[ht!]
	\centering
	\begin{tabular}{|l|rrr|}
	\hline
	 \small{Condition}             &   \small{\system} &   \small{\BFS} &   \small{\cluster} \\
	\hline
	 \small{Interesting}            &  \cellcolor{blue!25}       66    & 61    &      51   \\
	 \small{Not Interesting}        &  \cellcolor{blue!25}       10    & 20    &      22   \\
	 \small{Interesting (Normalized)} &   \cellcolor{blue!25}       0.87 &  0.75 &       0.7 \\
	\hline
	\end{tabular}
	\caption{Total counts of visualizations marked as interesting or not interesting across the different conditions. \system leads to more visualizations marked as interesting and fewer visualizations marked as uninteresting.}
	\label{table:interestingScore}
	\vspace{-10pt}
\end{table}
% \begin{table}[ht!]
% 	\centering
% 	\begin{tabular}{lrrr}
% 		\hline
% 		 Dataset   &   \system &   Cluster &   BFS \\
% 		\hline
% 		 Police    &      1.03 &      0.87 &  1.65 \\
% 		 Autism    &      3.55 &      3.00 &  1.90 \\
% 		\hline
% 	\end{tabular}
% 	\caption{Average consensus-agreement score for different algorithm and datasets.}%\agp{breakdown by interested and not interested}}
% 	\label{table:interestingScore}
% 	\vspace{-20pt}
% \end{table}
% \dor{We should consider doing a Chi square rather than averaging here to show significant difference between groups.}
%Even though participants were asked to --- daily experiences to answer the question, b
% \npar Due to the highly subjective nature of the retrieval task, the interestingness selection for the Police dataset was biased by participant's priors and intuition about the attributes. For example, while all participants who have seen the visualization "duration=30+min" verbally noted that stop duration is a crucial factor that leads to arrest, only 4 users marked it as interesting. 5 participants marked the visualization as not interesting and 4 left it unselected, because the visualization was not very surprising as it agreed with their intuition that ``\textit{if the police stop is taking a long time, something has probably gone wrong}''.
% \par Since the attributes in the Autism dataset are simply question numbers, participants could not associate any priors to their interestingness selection. In this prior-agnostic case, participants who used \system\ found more visualizations of interest that corresponded to the consensus, indicating that there are more interesting visualizations picked out by \system\ than compared to \BFS (p=0.003) and \cluster (p=0.09).
\stitle{Attribute Ranking (RQ2):}
To determine the attribute importance for a dataset, we computed the Cramer's V statistics between attributes to be ranked and the attributes of interest. Cramer's V is a common measure for determining the strength of association between categorical attributes~\cite{McHugh2013}. We deem an attribute as important if it has one of the top-three Cramer's V scores amongst all attributes of the dataset. This relevancy cutoff is visually-determined via the elbow method to indicate which rank the Cramer's V score drops off significantly. For the list of rankings provided by each participant, we first remove attributes where participants chose not to rank. Then we obtain the ground truth ranking based on the Cramer's V statistics for the ranked attributes. We compute the F-scores and average precision (AP) at k across a list of different k values (from 1 up to the number of ranked attributes, with k values corresponding to attributes ranked as ties deduplicated). Table \ref{table:ranking_results} summarizes the average across users in each condition, after picking the best performing k value for each user based on F-score and AP respectively. Both measures effectively capture how accurately participants were able to retrieve the three most important attributes for each dataset.
\begin{table}[ht!]
	\centering
	\begin{tabular}{|l|l|l|l|l|}
	\hline
	         & \multicolumn{2}{l|}{Police}                                   & \multicolumn{2}{l|}{Autism}                                   \\ \hline
	Metric   & F                             & AP                            & F                             & AP                            \\ \hline
	\system  & \cellcolor{blue!25}0.750 & \cellcolor{blue!25}0.867 & 0.723                         & 0.600                         \\ \hline
	\cluster & 0.739                         & 0.691                         & \cellcolor{blue!25}0.725 & \cellcolor{blue!25}0.665 \\ \hline
	\BFS     & 0.739                         & 0.592                         & 0.222                         & 0.200                         \\ \hline
	\end{tabular}
	\caption{Best AP and F-scores for the attribute ranking task.}
	\vspace{-10pt}
    \label{table:ranking_results}
\end{table}
\par Even though \BFS has inherent advantage for this task since \BFS dashboards consist of all univariate distributions, which provides more high-level information regarding each attribute, both \system and \cluster (which contained more `local' information) performed better than \BFS for both datasets. The problem with \BFS is that given the limited number of visualizations that could be shown on a dashboard, not all univariate distributions can be exhaustively shown. For the Police dataset, it happened to select several of the important attributes (related to contraband and search) to display in the first 10 visualizations. However, with a budget of k=10, only visualizations regarding binary diagnostic questions 1-4 fit in the dashboard for the Autism dataset. So the poor ranking behavior comes from the fact that the \BFS generated dashboard failed to display the three most important attributes (questions 5, 6 and 9) given the limited budget. This demonstrates \BFS's lack of providing a guarantee especially when exhaustive exploration has a limit (e.g., time or attention of analyst). 
\par We see that \system\ performs better than \cluster for the Police dataset and closely follows \cluster for the Autism dataset. It is not entirely surprising that \cluster did well, since it is a well-established method for summarizing high-dimensional data~\cite{Han2005}. For the Autism dataset, \cluster happened to pick the majority of visualizations (8/10) as univariate distributions that exhibited high-skew and diversity, leading to more informed inference on attribute importance. Since clustering seeks visualizations that exhibit diversity in the shape of the data distributions, it could potentially result in visualizations with many filter combinations. For the police dataset, 6 out of 10 visualizations had 2-4 filters, making it difficult for analysts to interpret without appropriate context to compare against. 
\par Both \BFS and \cluster do not provide consistent guarantees for highlighting important visualizations across different datasets. In general, our results indicate that users gain a better understanding of attribute importance using \system, with only a few targeted visualizations that tells the `whole story'. Note that this is without \system being explicitly optimized for this ranking purpose.
%\BFS performs better than \system in the Police dataset, but not in the Autism dataset. \BFS may have performed better than \system\ in the Police dataset for a combination of two reasons: 1) since \BFS exhaustively displays all attributes sequentially, for the Police dataset it happened to select several of the important attributes (related to contraband and search) to display in the first 10 visualizations and 2) some participants had priors on the data attribute which influenced their ranking. 
% In order to evaluate how accurately analysts understood the distributions present in various drill down paths, we use the prediction tasks as a proxy for how informative the selected visualizations in the dashboard are\agp{confusing sentence}. Informativeness is measured by how accurate participants are at predicting an unseen visualization.
\stitle{Prediction (RQ3):} The prediction task serves as a proxy for evaluating how accurately analysts understood the distributions present in various drill-down paths. In particular, we can get a sense of how \emph{informative} the dashboards were by examining how accurately participants could use visualizations present in the dashboard to predict an unseen visualization. The accuracy of participants' predictions was measured by the Euclidean distance between the predicted distributions and ground truth data distributions. As shown in Figure~\ref{fig:distance} (left), predictions made using \system\ (highlighted in red) were closer to the actual distribution than compared to the baselines, as indicated by the smaller Euclidean distances. Figure~\ref{fig:distance} (right) also shows that \system participants found the resulting visualizations to be less surprising, since they were able to more accurately reason about the expected properties of unseen data subsets. \cluster may have performed better in the Police dataset than it did in the Autism dataset due to the same reason as in the attribute ranking task, where more univariate visualizations happened to be selected.
\begin{figure}[h!]
\centering
\includegraphics[width=0.95\linewidth]{figures/prediction_surprisingness_distance.pdf}
\caption{Left: Euclidean distance between predicted and ground truth. In general, predictions made using \system are closer to ground truth. Right: Surprisingness rating reported by users after seeing the actual visualizations on a Likert scale of 10. \system participants had a more accurate mental model of the unseen visualization and therefore reported less surprise than compared to the baseline.}
\label{fig:distance}
\vspace{-10pt}
\end{figure}
% \par \system\ did not perform as well compared to the baselines for the Autism deep prediction task. One possible reason for this is due to the fact that the shallow and deep prediction tasks for the Autism dataset were correlated. Therefore, after learning about the insights that answering 1 on question 9 results in a very high probability for an autism diagnosis, some participants made use of that information when tackling the subsequent deep prediction task. By discussing with the baseline participants on how they have obtained the prediction estimates, they described how surprised they were by the finding in the shallow prediction and therefore adjusted the autism diagnosed values to be higher to compensate for their mistake in the subsequent deep prediction task.
\par We also compute the variance of participants' predictions across the same condition. In this case, low variance implies that any participant who reads the dashboard is able to provide consistent predictions, whereas high variance implies that the dashboard did not convey a clear data-driven story that could guide participants' predictions. So instead, participants relied on different priors or guessing to form their prediction. These trends can be observed in both Figure~\ref{fig:distance} and in more detail in Figure \ref{fig:actual_predictions}, where the prediction variance amongst participants who used \system\ is generally lower than the variance from the baselines.
\begin{figure}[h!]
\centering
\includegraphics[width=0.85\linewidth]{figures/prediction.pdf}
\caption{Mean and variance of predicted values. Predictions based on \system exhibit lower variance (as indicated by the error bars) and closer proximity to the ground truth values (dotted).}
\label{fig:actual_predictions}
\vspace{-15pt}
\end{figure}
% %!TEX root = main.tex
\section{Discussion}
To understand the usefulness of our recommended visualizations, we analyzed the user study transcriptions through an open coding process by two of the authors. For each task in our study, we assigned a binary-valued code to indicate whether or not a participant engaged in that particular task (action or thought process). Table~\ref{table:thematic_summary} highlights results from thematic coding, discussed in this section. We will use the notation [Participant.DatasetAlgorithm] to refer to a participant engaging with a dashboard created by an algorithm=\{1,2,3\}=\{\system, \cluster, \textsc{BFS}\} on a dataset =\{A,B\}=\{Police, Autism\}.
% \stitle{\system promotes distribution-awareness by provoking comparisons against more informative contextual references.}We first studied the thematic codes to understand how participants select contextual references for visualizations. 
\subsection{The Choice of Contextual References}
\par As discussed earlier, analysts often make use of related visualizations as \emph{contextual references}\agp{define if not here, define here.} to form their expectations for unseen visualization. The choices of a proper informative parent is essential for ensuring the \emph{safety} of insights derived through drill-downs. To understand how `safe' the dashboards generated from each condition were, we examined the types of visualizations that participants utilized and compared against to form their expectations regarding how other unseen visualizations should look like. In particular, we thematically encoded participant's use of contextual references based on the verbal explanations that they provided to justify their prediction task responses. Participants can (and often do) make comparisons against more than one type of contextual references to obtain their prediction. We uncovered four main classes of contextual references, described below using the example visualization \texttt{gender=F, race=White, age=21-30} (in the order of most to least similar) and illustrated graphically in Figure~\ref{fig:reference}:
\begin{enumerate}
	\item Parent : Comparison against a visualization with one filter criterion removed (e.g., \texttt{gender=F, race=White})
	\item Siblings : Comparison against a visualization that shares the same parent. In other words, the filter types are the same, but with one criterion changed to inherit a different value. (e.g., \texttt{gender=M, race=White, age=21-30})
	\item Relatives : Comparison against a visualization that shares some common ancestor (excluding overall), but not necessarily the same parent. In other words, these visualizations share at least one common filter type, but with more than one criterion that inherits a different value. (e.g., \texttt{gender=F, race=White, age=60+, search conducted=T})
	\item Overall : Comparison against the distribution that describes the overall population (no filters applied).
\end{enumerate}
\begin{figure}[h!]
\centering
\includegraphics[width=\linewidth]{figures/contextual_reference.pdf}
\caption{Illustrative example of the different types of contextual reference for a given visualization of interest.}
\label{fig:reference}
\end{figure}
As shown in Table \ref{table:contextualReferenceCount}, in general, we find that participants make more comparisons in total using \system than compared to \cluster and \BFS. Studying participants' use of contextual references reveals inherent challenges that arise from using the dashboards generated by \BFS and \cluster. For \cluster , participants mainly compared against relatives and the overall visualizations. Since \cluster optimizes the diversity of shape (distributions) amongst the visualizations, the selected visualizations had up to 4 filters and were disconnected from each other. For this reason, in many cases participants could only rely on relatives and the overall one as contextual references. For example, P4.A2 pointed at a 4-filter visualization with extreme values (100\% for warning; 0\% for arrest and ticket) and indicated how ``\textit{a lot of [the visualizations] are far too specific. This is not very helpful. You can't really hypothesize that all people are going to be warned, because it is such a specific category, it might just be one person}''. %, you need to have a bigger dataset. And that category will not really give you such extremes to make it more credible''.
He further explained how he ``\textit{would not want to see the intersections [(visualizations with many filters)] at first and would want to see all the bases [(univariate summaries)] then dig in from there.}'' The lack of informative contextual references in the \cluster dashboard is also reflected in how analysts exhibited high variance and deviation in their prediction responses.
%despite our modification to KMeans which picks the visualizations with the least number of filters to show in the dashboard for improving interpretability
%These themes are drawn from user's explanations of how they obtained certain insights or ---- that --different tasks or while interpreting the dashboards. 4 categories :
\begin{table}[h!]
\hspace{-10pt}
\centering
	\begin{tabular}{|l|rrrr|r|}
	\hline
	 \small{Algorithm}   &    \small{Parent} &   \small{Sibling} &   \small{Relative} & \small{Overall} &   \small{Total} \\
	\hline
	 \small{\system}     &    \cellcolor{blue!25} 12 &       8 &     0 &  11 &      \cellcolor{blue!25} 31 \\
	 \small{\cluster}     &         4 &        0 &         7 &          8 &      19 \\
	 \small{\BFS}         &         0 &        5 &         1 &          8 &      14 \\
	\hline
	\end{tabular}
\caption{Out of 12 participants, the number of participants who made use of each contextual reference across the two datasets. Participant behavior shows a similar trend in individual datasets. \system participants made more comparisons in general and against parents compared to the baseline.}
\label{table:contextualReferenceCount}
\end{table}
\par For \BFS, most comparisons were based on the overall and siblings. Due to the sequential level-wise picking approach, for the \BFS dashboards, the overall visualization corresponded to the immediate parent, so they are not explicitly recorded as a parent. While the overall and sibling comparisons can be informative, due to the limited number of visualizations ($k$), not all first-level visualizations were displayed in the dashboard. These incomplete comparisons can result in flawed reasoning, as observed in the Autism shallow prediction task described earlier. In contrast, for \system, almost all users compared against the overall and parents, while some also exploited sibling comparison information to make weaker guesses for less-frequently observed attributes (e.g., using a 2-filter sibling visualization involving \texttt{driver\_age} to infer another 2-filter visualization \texttt{driver\_age} with a different parent.)
% \subsection{Improper contextual reference can lead to misleading insights.}
\begin{table*}[ht!]
	\centering
	\begin{tabular}{|l|l|l|l|}
	\hline & \system & \cluster & \BFS \\ \hline
	Difficulty with Interpretating Visualizations & 0 & \cellcolor[HTML]{FD6864}3 & 1 \\ \hline
	Misjudged Significance of Potential Small-Size Population & 0 & \cellcolor[HTML]{FD6864}4 & 1 \\ \hline
	Interpretable ``Human-like" Dashboard & \cellcolor[HTML]{9AFF99}5 & 1 & 0 \\ \hline
	Number of Insights (Police) & \cellcolor[HTML]{9AFF99}11 & 8 & 9 \\ \hline
	Number of Insights (Autism) & \cellcolor[HTML]{9AFF99}16 & 6 & 11 \\\hline
	\end{tabular}
\caption{Summary of qualitative insights from thematic coding. We record the number of insights based on overall findings regarding the dataset or information regarding one or more attributes that are discovered by more than two different participants. There were a total of 7 such insights that we coded for the Police dataset and 6 for the Autism dataset.}
\label{table:thematic_summary}
\end{table*}
\subsection{The Danger of Improper References}
\par While comparisons are essential for data understanding, choosing the wrong contextual reference for comparison could lead to misleading insights. In particular, when a visualization composed of multiple filter conditions is shown in a dashboard created using \cluster, 25\% of participants had trouble interpreting the meaning of the filter for at least one of the datasets. In contrast, as shown in Table~\ref{table:thematic_summary}, this confusion only happened once for \BFS and none for \system. This is due to the fact that \cluster dashboards are seemingly random to the users, whereas \BFS and \system both have a more natural, interpretable ordering. In addition, when examining visualizations with many filters along with extremely-skewed values in one or more bars (bars with 100\% or 0\%), 4 \cluster participants did not realize that charts with multiple filters may have a smaller subpopulation size, echoing our previous concern regarding the danger of small subpopulation sizes. This issue stems from the fact that the contextual reference used for comparison was the overall population, however the unseen parent subpopulation may have behaved very differently. This subpopulation-size fallacy was observed to be more severe for the Autism dataset, where participants had less intuition on the expected attribute behavior. In contrast, 6 of the participants using \system explicitly noted that while these extreme-valued visualizations may be interesting, they were less certain due to the unknown subpopulation size and should be investigated further. For example, P1.A1 noted that a visualization with warning=100\% caught her eye, ``\textit{but I don't know what the N is, maybe it's one person, this makes me a little skeptical, that makes me want to go back to the raw data and look at what is the N and what drives something so drastic?}'' Since \BFS dashboards only displayed first-level visualizations, participants for \BFS did not see such visualizations during the study session, so none of the \BFS participants exhibited signs of this fallacy.

% \subsection{Hierarchical layout leads to more natural contextual comparisons compared to table layout.}
\subsection{Interpretability of Hierarchical Layouts}
\par In the post-study interviews,  participants cited hierarchical layout as one of the key reasons why it was easier to follow contextual references in \system. Users were able to easily interpret the meaning of the dashboard through \system's hierarchical layout, even though they were never explicitly told what the edge connections between the visualizations meant. For example, P1.A1 stated that ``\textit{the hierarchical nature [is] a very natural flow...so when you are comparing, you don't have to be making those comparisons in your head, visually that is very pleasing and easy to follow.}'' %Likewise, [P8.A1] also stated that ``I like the different levels, it makes it very visually easy to figure out what you want to look at, if you want to look at the overall data, it's right there at the top for you, if you want to get more specific, you just follow a branch downwards, which I think is very intuitive.''
Likewise, P9 described how the hierarchical layout she saw for the Autism dataset was a lot easier to follow than the Police dataset shown in the table layout for \cluster:
\begin{quote}
\textit{If I had to look at this dataset in the format of the other one, this would be much more difficult. It was pretty hard for me to tell in the other one how to organize the tree, if there was even a tree to be organized. I like this layout much better, I think this layout allows me to approach it in a more meaningful way. I can decide, what do I think matters more: the overall trend? or the super detailed trends? and I know where to look to start, in the other one, every time I go back to it, I would say, where's the top level, where's the second level? I mentally did this. Like when you asked me that first question, it took much longer to find it, because I literally have to put every chart in a space in my head and that took a lot longer than knowing how to look at it.}
\end{quote}
At the end of the study, some participants who saw table layouts sketched and explained how they would like the layout of the visualizations to be done. Participants expressed that they wanted ``groupings'' or layouts that arranged visualizations with the same attribute together. Other participants advocated for isolating the overall visualization outside of the dashboard table for facilitating easier comparisons. Both of these provides further motivation for our hierarchical layout and the idea of the collapsed visualizations as described in the System section.%in Section \ref{sec:interaction}.
\par Since we did not inform participants about how the dashboards were generated, it was also interesting to note that some participants thought that the dashboards were hand-picked by a human analyst and described what this person's intentions were (e.g., ``\textit{It seems like the researcher who created this dashboard was specifically looking at people of Asian descent and people who are 60 or older.}'' [P7.A1]). We encoded this phenomenon by looking at instances where a participant either explicitly referring to a person who picked out the dashboard or implicitly described their intentions through personal pronouns. As summarized in Table~\ref{table:thematic_summary}, a total of 5 different participants referred to the \system dashboards as if they were generated by a human, whereas there was only 1 participant for \cluster and none for \BFS made such remarks. At the end of the study, many were surprised to learn that the \system dashboard was actually picked out by an algorithm, indicating that \system could automatically generate convincing dashboard stories similar to a dashboard that was authored with human intention.
\stitle{Limitations of \system}
% Interestingness task is highly subjective, so not conclusive whether interesting or not , despite the positive result
% Due to the highly subjective nature of the retrieval task, the interestingness selection for the Police dataset was biased by participant's priors and intuition about the attributes. For example, while all participants who have seen the visualization "duration=30+min" verbally noted that stop duration is a crucial factor that leads to arrest, only 4 users marked it as interesting. 5 participants marked the visualization as not interesting and 4 left it unselected, because the visualization was not very surprising as it agreed with their intuition that ``\textit{if the police stop is taking a long time, something has probably gone wrong}''.
\par As described earlier, since the details of how the dashboard was obtained was not explained to the users during the study, some users expressed that they were initially confused by \system since not all variables were present in the dashboard. Others also found it confusing that the addition of filters did not always correspond to the same variables. For example, P2.A1 criticized how the dashboard was intentionally selected to be biased:
\begin{quote}
\textit{I feel like this one, not all the data is here, so we are already telling a story, you are trying to steer the viewer to look at certain things. And the focus seems to be on where the arrest rate is high. You probably could have found other things that led to ticket being high, but you didn't pull those out. You are trying to see if there are other factors that leads to more arrests.}
\end{quote}
\npar This sentiment is related to participants' desire to perform their own ad-hoc querying alongside the dashboard to inspect other related visualizations for verifying their hypothesis. For example, P7.A1 wanted to inspect all other first-level visualizations for driver's race to assess its influence. P7.A1 expressed that while he had learned many insights from the dashboard, ``\textit{the only thing I don't like is I cannot control the types of filter, which is fixed.}'' Outside the context of the user study, it is essential to explain how \system are picking the visualizations in a easy and interpretable manner to establish a sense of summarization guarantee for the users and help them make better inferences with the dashboard.
\par As discussed earlier, the subpopulation size is important in establishing the significance of a trend observed in a visualization. While subpopulation size is taken into account implicitly in our objective, we should design interfaces that can convey the notion of subpopulation size in our dashboard. Examples include Sankey-like flow diagrams indicating the percentage of the parent population broken down into individual subpopulations or subpopulation size explicitly specified via edge labels.%, either explicitly displayed as text when hovering over the visualization or changing the size or background color of the visualizations to encode subpopulation size.
%“I actually found it really confused at first because such a low arrest rate at the top, and then at the bottom the arrest rate was much higher, so I was like is this data wrong. Then I realized we’re not looking at all the data here, you’ve pulled out some of it. It took me a minute to realize that. And once I read the title of the charts I realized that makes sense.” [P2.A1]
% - Reference of Comparison
% - Layout naturally lends itself for comparison:
% 	- describe ordering layout, how participants naturally follow the flow
% 	- emph that we did not tell them what the edge connections mean and how they were computed but the users naturally figured it out, that it means adding an additional filter.
% 	- hierarchical interpretable nature (quotes)
% 	- compared to other baselines
% 	- describe dashboard by human (count)
% - Misleading insights v.s. True insight discovery rates
% 	- Interpretability:
% 	- misled understanding subpopulation size
% 		- for autism, it is important to see if they compare to overall because if not they would think high skew to NO is important whereas its actually pretty close to overall.
% 	- trouble interpreting filter combination
%%%%%%%%%%%%%%%%%%%%%%%%%%%%%%%%%%%%%%%%%%%%%%%%%%%%%%%%%%%%%%%%%%%%%%%%%%%%%%%%%%%%%%%%%%%%%%%%%%%%%%%%%%%%%%%%%%%%%%%%
%%%%%%%%%%%%%%%%%%%%%%%%%%%%%%%%%%%%%%%%%%%%%%%%%%%%%%%%%%%%%%%%%%%%%%%%%%%%%%%%%%%%%%%%%%%%%%%%%%%%%%%%%%%%%%%%%%%%%%%%
%%%%%%%%%%%%%%%%%%%%%%%%%%%%%%%%%%%%%%%%%%%%%%%%%%%%%%%%%%%%%%%%%%%%%%%%%%%%%%%%%%%%%%%%%%%%%%%%%%%%%%%%%%%%%%%%%%%%%%%%
% \subsection{Statistical Paradoxes}\dor{make title full sentences}
% Visualizations are powerful representations for studying different distributions or patterns in a dataset, but our human intuition could often mislead us when it comes to interpreting those patterns\cite{Binnig2017,Wall2017}. Several statistical paradoxes can lead analysts to draw incorrect conclusions from observed visualizations, including Simpson's paradox as discussed in the introduction. The key reason why many of these paradoxes emerge is the \emph{incompleteness} of the observed data or lack of focus on relevant informative subsets of the data. For example, Simpson's paradox arises in the presence of an unseen confounding variable. %likewise, the absence of  base rate information causes base rate fallacy.
%  We assert \dor{too strong of a sentence} that distributional awareness can be useful in avoiding such statistical paradoxes. If an analyst is aware of all distributions in a given dataset, he/she is less prone to many statistical paradoxes. However, given the large number of dimensions and high cardinality of these dimension in modern datasets, it is not possible for an analyst to explore and memorize all distributions. Therefore, a more evolved approach is to be aware of the exceptional distributions. In this work, we propose a first step towards this goal, where we identify the exceptional distributions in terms of their informative references. The remaining (unseen) distributions in the dataset are rather unsurprising and can be inferred from the visualizations in the dashboard. \dor{I would recommend first talk about issue with large dimension + danger of multiple hypothesis testing + incomplete testing, point out problem, then talk about how our system resolves this.}
% \subsection{Structural Insight}
% Our proposed dashboard consists of a hierarchy of visualizations, where each visualization is linked to its most informative parent. The shape or structure of the hierarchy contains useful information that augments the information learned from the visualizations and aid distribution awareness and understanding. \dor{what's interesting here is that while many work have looked at visualization presentation, layout of presentation never considered, we find in Sec 5 that this is actually important and can encode info.} For example, the depth and branching factor of the hierarchy could inform a user regarding the configuration of insights. Deep hierarchies contain long paths, i.e., insights are present at lower level visualizations with multiple constraints. In contrast, bushy hierarchies (with high branching factor) contain cases where multiple visualizations have the same informative parent and they differ from that parent. \dor{do we have examples from the study that support this?} We assert that the depth and branching factor could be a meaningful constraint in our problem formulation \dor{too strong of a sentence}. Some applications for example, funnel exploration require studying deep hierarchies, whereas others for example, building decision trees require studying bushy hierarchies. A natural extension of our current problem formulation is to allow users to select the depth and branching factor for the hierarchy. 
% \subsection{Other Visualization Lattices}
% In this work, we explore the space of data subsets to generate our visualization lattice. Note that it is possible to explore the space of dimension attributes in x-axis to generate a different visualization lattice. In particular, given a combination of dimension attributes $X = \{X_1, \ldots, X_n\}$, adding one or more new dimensions in $X$ will generate a new combination. An ancestor-descendant relationship exists between these dimension combinations, following the same principles of Section 3.1. These relationships lead to a new lattice, which we call the dimension combination lattice. Our informative deviation based approach could be used for traversing the dimension combination lattice. However, we observe that most users do not visualize more than two attributes in x-axis. Therefore, traversing the dimension combination lattice is not very useful for most applications.
% \dor{I think 6.2,6.3 don't tie well with the rest of the paper. It sounds like stretching our own ideas rather than being motivated by the work done in this paper. Other potentially more relevant discussion: distribution awareness and how it might be useful in other contexts? Decision trees?}
% %\subsection{Utility Metrics} 
% %!TEX root = main.tex
\section{Related Work}

Our work draws from, and improves upon, past research in multidimensional data exploration, fallacies in visual analytics, decision tree visualization, and visualization storytelling.

\subsection{Guided Exploration of Multidimensional Data}
Given a dataset, tools such as Tableau support automatic generation of visualizations based on perceptual graphical presentation rules~\cite{Mackinlay2007,Wongsuphasawat2016}. A more recent body of work automatically selects visualizations based on statistical measures, such as scagnostics and deviation. Given a scatterplot, Anand et al. \cite{Anand2015} applies randomized permutation tests to select partitioning variables that reveals interesting small multiples using scagnostics. Given a bar chart, Vartak et al. \cite{Vartak2015} finds other interesting bar charts that deviate form the input chart using a deviation-based measure. Our work extends the deviation-based measure to formulate user expectation. However, unlike existing works, we concentrate on informativeness, which enables our system to avoid drill-down fallacies.
%\cite{Elmqvist2008Rolling} presents an interactive tool to explore multidimensional data using a matrix of scatterplots that shows the relationship between all pairs of attributes.

\subsection{Preventing Biases and Statistical Fallacies}
Visualizations are powerful representations for discovering trends and patterns in a dataset; however, cognitive biases and statistical fallacies could mislead analysts' interpretation of those patterns~\cite{Alipourfard2018WSDM,Wall2017,Zgraggen2018CHI}. Wall et al.~\cite{Wall2017} presents six metrics to systematically detect and quantify bias from user interactions in visual analytic systems. These metrics are based on coverage and distribution, which focus on the assessment of the process by which users sample the data space. Alipourfard et al.~\cite{Alipourfard2018WSDM} presents a statistical method to automatically identify Simpson's paradox by comparing statistical trends in the aggregate data to those in the disaggregated subgroups. Zgraggen et al.~\cite{Zgraggen2018CHI} presents a method to detect the presence of the multiple comparisons problem in visual analysis. In this paper, we concentrate on a novel type of fallacy during drill-down exploration that has not been addressed by past work. %drill-down fallacy, a fallacy that has not been addressed before in visual analytics literature.

\subsection{Decision Tree Visualization}
The popularity of decision trees in a variety of classification tasks have led to the development of visualizations that make these models more interpretable~\cite{Ankerst1999,Hermann2017,Terence2018}. These visualizations often contain a visual representation of the rules as paths connecting the decision nodes, illustrating the proportion of sample along different paths, as well as statistics regarding the prediction accuracy at every node. Though our dashboards visually look similar to decision trees, the underlying objectives are different for the two methods. During tree construction, a decision tree algorithm aims to improve the classification accuracy of a target variable, typically by minimizing the entropy of distribution from parent node to child node~\cite{Quinlan1986}. In contrast, our method aims to deliver informative insights, by maximizing the informative deviation between parent and child nodes. Consequently, the generated outcomes are different for the two methods---a decision tree well explains the general rules (e.g., if stop duration is more than 30 minutes, the driver has 60\% probability of being arrested), whereas our method well explains the exceptions (e.g., if a stop duration is more than 30 minutes and the driver's race is Asian, the probability of arrest goes down to 35\%). Note that the general rule is useful for predicting the stop outcome for an unlabeled test datapoint (classification), whereas the exception is useful for realizing when the general rule no longer holds (insight). The latter insight may not be discovered by a decision tree as it does not directly improve classification accuracy. Another key difference between the two methods is \emph{coverage}---a decision tree covers the entire dataset (consistent with its classification goal), whereas our method highlights only the interesting regions of a dataset (consistent with its insight goal).

%While both methods are useful in understanding the distributions in different regions of data, and further in model interpretability. In particular, decision trees have been very successful in learning where and how a model works; our method could be useful in understanding where it might break.

\subsection{Storytelling with Visualization Sequences}
Visualizations are often arranged in a sequence to narrate a data-driven story. Existing work on visualization sequences and storytelling has studied the structures of narrative visualizations~\cite{Hullman2017,Segel2010}, effects of augmenting exploratory information visualizations with narration~\cite{Boy2015} and, more recently, ways to automate the creation of visualization sequences~\cite{Hullman2013,Kim2017}. Most of these work have adopted a linear layout (motivated by slidedecks) to present the visualization sequences. Hullman et al.~\cite{Hullman2017} found that most people prefer visualization sequences structured hierarchically based on shared data properties such as levels of aggregation. Kim et al.~\cite{Kim2017} modeled relationships between charts by empirically estimating transition (edge) cost between moving from one visualization (node) to another. They found that participants preferred ``\textit{starting from the entire data and introducing increasing levels of summarization}''. Our work is the first to automatically organize visualizations in a hierarchical layout for summarizing data distributions across the space of data subsets.

%Both \cite{Hullman2013,Kim2017} use a graph model to formalize the visualization design space.
%In addition, we present a novel problem formulation that recommends a connected visualization sequence in a hierarchical layout summarizing the space of data subsets.

% \iffalse
% \section{Related Works}
% \npar \stitle{Storytelling with visualization sequences:}
% Visualizations are often arranged in sequence to narrate a data story. Existing work on visualization sequences and storytelling have studied the structures of narrative visualizations\cite{Segel2010,Hullman2017}, effects of augmenting exploratory information visualizations with narration\cite{Boy2015} and, more recently, ways to automate the creation of visualization sequences\cite{Hullman2013,Kim2017}. Most of these work have adopted a linear layout (motivated by slidedecks) to present the visualization sequences. Hullman et al. \cite{Hullman2017} found that most people prefer visualization sequences structured hierarchically based on shared data properties such as levels of aggregation. %Both \cite{Hullman2013,Kim2017} use a graph model to formalize the visualization design space.
% Kim et al. \cite{Kim2017} models relationships between charts by empirically estimating transition (edge) cost between moving from one visualization (node) to another. They find that participants preferred ``\textit{starting from the entire data and introducing increasing levels of summarization}''. Our work is the first to automatically sequence visualizations in a hierarchical layout for summarizing the space of data subsets. %In addition, we present a novel problem formulation that recommends a connected visualization sequence in a hierarchical layout summarizing the space of data subsets.

% \subsection{Visualization recommendation}
% \par%Despite the large body of work that recommends informative visualizations given pre-selected data attributes and aggregations, the data selection problem is a more important problem in exploratory data analysis, since the analysts have to figure out which groups of data attributes would be of interest in order avoid manual exploration of the data.
% Visualization recommendation systems select appropriate visualizations to show based on an objective function. The metrics considered by these systems can largely be divided into two categories: perceptual or data-driven. The first type of recommendation system selects visualizations based on its visual effectiveness and expressiveness~\cite{Wongsuphasawat2016,Mackinlay2007}. Our work is more related to the latter category of systems which uses statistical measures computed based on the underlying data subset, such as cognostics or deviation. Anand et al. \cite{Anand2015} used randomized permutation tests to automatically select partitioning variables to display visualizations exhibiting patterns that are different from the input visualization as determined by its cognostic score. Vartak et al. \cite{Vartak2015} finds interesting visualizations by a deviation-based measure between the user's query view and reference view, given a query of interest. While both existing systems require the analyst to input a visualization of interest as a query, our paper extends the deviation-based idea to establish user's expectation using informative parent enabling \system to traverse the visualization lattice in search of a connected, maximally informative and interesting story without the need for an input query.  %Wongsuphasawat et al. \cite{Wongsuphasawat2016} presents a mixed-initiative system where the users direct the variables of interest and the system suggests other variables that may be potentially interesting to the user. Since this is a mixed-initiative system rather than an automatic recommendation engine, the system only ``looks ahead" one variable at a time. Their goal is to promote breadth-oriented data exploration rather than helping users find interesting stories or visualizations.

% %- the issue of surprisingness metric and ---- have been examined before for viz, but none have looked at data subset lattice specifically for viz

% \subsection{Data Exploration of OLAP Data Cubes} %\dor{This may be a bit long and need to be cut, I didn't include related works on surprisingness metrics (e.g. Bayesian Cognition paper, Surprise Map ,etc.). Can add if necessary.}
% The challenge of manual, unguided search in online analytical processing (OLAP) applications have been well studied in the context of data cube exploration by Sarawagi et al. ~\cite{Sarawagi1999,Sarawagi2000,Sarawagi1998}. To address this challenge, they simplify the search by identifying ``interesting'' regions of a data cube. These techniques includes precomputed statistics accounting for the surprisingness attributed to neighboring paths to cell and amount of deviation from constrained maximum entropy-based expectations. While these interesting sub-cubes correspond to finding filter combinations for constructing the aggregate visualization in \system, our lattice search space enforces fixed x, y and aggregation as well as connectedness during traversal to discover more interpretable stories.

% %Sarawagi et al.\cite{Sarawagi1998} introduced the problem that manual, unguided search for seeking interesting patterns in a datacube is inefficient and requires large numbers of online analytical processing (OLAP) operations, such as roll-up, drill-down, slicing, and dicing. They proposed a discovery-driven approach to data exploration to simplify the search for \textit{exceptions} in the data, based on precomputed statistics regarding how surprising a data cube cell is, relative to neighboring cells at the same level of aggregation, at levels of aggregation below the cell, and along the drill-down path. Sarawagi \cite{Sarawagi1999} presents an OLAP operator that summarizes the reasons for variation in a data view, by computing the information theoretic distance between the immediate parent and its child nodes.
% % \par While Sarawagi et al. \cite{Sarawagi1998} takes a more data-driven approach of finding exceptions intrinsic to a given datacube, Sarawagi \cite{Sarawagi2000} envisions a more user-centric application where the comparisons are based only on parents of seen visualization. The user's expectation regarding an unseen visualization is based on the maximum entropy principle where the relationships between the attributes should be maximally uniform across all dimensions, while being consistent with constraints from seen visualizations. The visualization that deviates the most from the user expectation is regarded as an ``interesting" visualization. They take an iterative approach to find the unique solution for the expected values for each attribute from the constrained maximum entropy problem and employ several optimization strategies (reusing computed values, sharing storage of contiguous regions, pruning constraints that subsumes one another or with little influence). Our coverage-based models improve on the iterative approach in providing a tighter constraint to the variable regions of the bars.
% \fi

% %!TEX root = main.tex
\section{Conclusion}

\system compared to baselines
- perform better in a wide range of analytic task such as attribute ranking, prediction, and interestingness. 
- interpretable, more insights

\bibliographystyle{ACM-Reference-Format}
\bibliography{reference}

\end{document}
