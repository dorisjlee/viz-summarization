%!TEX root = main.tex
\section{Discussion}
To understand the usefulness of our recommended visualizations, we analyzed the user study transcriptions through an open coding process by two of the authors. For each task in our study, we assigned a binary-valued code to indicate whether or not a participant engaged in that particular task (action or thought process). We will use the notation [Participant.DatasetAlgorithm] to refer to a participant engaging with a dashboard created by an algorithm=\{1,2,3\}=\{\system, \cluster, \textsc{BFS}\} on a dataset =\{A,B\}=\{Police, Autism\}.
% \stitle{\system promotes distribution-awareness by provoking comparisons against more informative contextual references.}
\subsection{The Choice of Informative References}
\par \dor{Connect to Safety} We first studied the thematic codes to understand how participants select contextual references for visualizations. In particular, we examined the types of visualizations that participants utilized (compared against) to form their expectations regarding how other visualizations should look like. %We define this property of visualization understanding as \textit{distribution-awareness} and the visualizations that are compared against as the \textit{contextual reference}. 
Via thematic coding, we uncovered four main classes of contextual references, described below using the example visualization \texttt{gender=F, race=White, age=21-30} (in order of most to least similar):
\begin{enumerate}
	\item Parent : Comparison against a visualization with one filter criterion removed (e.g. \texttt{gender=F, race=White})
	\item Siblings : Comparison against a visualization that share the same parent. In other words, the filter types are the same, but with one criterion that inherit a different value. (e.g. \texttt{gender=M, race=White, age=21-30})
	\item Relatives : Comparison against a visualization that share some common ancestor (excluding overall), but not necessarily the same parent. In other words, these visualizations share at least one common filter type, but with more than one criterion that inherit a different value. (e.g. \texttt{gender=F, race=White, age=60+, search conducted=T})
	\item Overall : Comparison against the distribution that describes the overall population (no filters applied).
\end{enumerate}
Studying participants' use of contextual references reveals inherent challenges that arise from using the dashboards generated by \BFS and \cluster. As shown in Table \ref{table:contextualReferenceCount}, for \cluster , participants mainly compared against relatives and the overall. Since \cluster optimizes the diversity of shape (distributions) amongst the visualizations, the selected visualizations had up to 4 filters and were disconnected from each other. For this reason, in many cases participants could only rely on relatives and the overall as contextual references. For example, [P4.A2] disliked how ``a lot of [the visualizations] are far too specific. [Pointing at visualization consisting of 4 filters with a 100\% bar for warning] This is not very helpful. You can't really hypothesize that all people are going to be warned, because it is such a specific category, it might just be one person''. %, you need to have a bigger dataset. And that category will not really give you such extremes to make it more credible''.
He further explained how he ``would not want to see the intersections (visualizations with many filters) at first and would want to see all the bases (univariate summaries) then dig in from there.'' In addition, the lack of informative contextual references in the \cluster dashboard is reflected in the high variance and deviation of the predicted visualization results.
%despite our modification to KMeans which picks the visualizations with the least number of filters to show in the dashboard for improving interpretability
%These themes are drawn from user's explanations of how they obtained certain insights or ---- that --different tasks or while interpreting the dashboards. 4 categories :
\begin{table}[h!]
\centering
	\begin{tabular}{l|rrrr|r}
	\hline
	 Algorithm   &   Overall &   Parent &   Sibling &   Relative &   Total \\
	\hline
	 \system     &        11 &      \cellcolor{blue!25} 12 &         8 &          0 &      \cellcolor{blue!25} 31 \\
	 \cluster     &         8 &        4 &         0 &          7 &      19 \\
	 \BFS         &         8 &        0 &         5 &          1 &      14 \\
	\hline
	\end{tabular}
\caption{Number of participants who made use of each contextual parents, summed across the two datasets. Participant behavior shows a similar trend in individual datasets. \system participants made more comparisons in general and against parents compared to the baseline.}
\label{table:contextualReferenceCount}
\end{table}
\par For \BFS, most comparisons were based on the overall and siblings. Due to the sequential level-wise picking approach, for the \BFS dashboards, the overall corresponded to the immediate parent, so they are not explicitly recorded as a parent. While the overall and sibling comparisons can be informative, due to the limited number of visualizations ($k$), not all first-level visualizations were displayed in the dashboard. These incomplete comparisons can result in flawed reasoning, as observed in the Autism shallow prediction task described earlier. In contrast, for \system, users mainly compared against the overall and parents, while some also exploited sibling comparison information to make a weaker guess for deep predictions. We also find that more participants make comparisons in total using \system than compared to \cluster and BFS.

% \subsection{Improper contextual reference can lead to misleading insights.}
\subsection{The Danger of Improper References}
\par While comparisons are essential for data understanding, choosing the wrong contextual reference for comparison could lead to misleading insights. In particular, when a visualization composed of multiple filter conditions is shown in a dashboard created using \cluster, 5 out of 12 participants for both datasets had a hard time interpreting the meaning of the filter, whereas there was only 1 for \system and BFS. This is due to the fact that \cluster dashboards are seemingly random to the users, whereas \BFS and \system both have some natural ordering. In addition, when examining visualizations with many filters along with extreme values in one or more bars (bars with 100\% or 0\%), 4 \cluster participants did not realize that charts with multiple filters may have a smaller subpopulation size. This issue stems from the fact that the contextual reference used for comparison was the overall population, however the unseen parent subpopulation may have behaved very differently. The fallacy was observed to be more severe for the Autism dataset, where participants had less intuition on the expected attribute behavior. In contrast, 6 of the participants using \system recognized that while these extreme-valued visualizations may be interesting, they were less certain due to the unknown subpopulation size and should be investigated further. For example, [P1.A1] noted that a visualization with warning=100\% caught her eye, ``\textit{but I don't know what the N is, maybe it's one person, this makes me a little skeptical, that makes me want to go back to the raw data and look at what is the N and what drives something so drastic?}'' Since \BFS dashboards only displayed first-level visualizations, participants for \BFS did not see such visualizations during the study session, so we did not see signs of this fallacy for \BFS participants.

% \subsection{Hierarchical layout leads to more natural contextual comparisons compared to table layout.}
\subsection{Interpretability of Hierarchical Layouts}
\par Participants cited hierarchical layout as one of the key reasons why it was easier to follow contextual reference in \system. Based on the hierarchical layout in \system, users were able to easily interpret the meaning of the dashboard, even though they were never explicitly told what the edge connections between the visualizations meant. For example, [P1.A1] stated that ``the hierarchical nature [is] a very natural flow...so when you are comparing, you don't have to be making those comparisons in your head, visually that is very pleasing and easy to follow.'' %Likewise, [P8.A1] also stated that ``I like the different levels, it makes it very visually easy to figure out what you want to look at, if you want to look at the overall data, it's right there at the top for you, if you want to get more specific, you just follow a branch downwards, which I think is very intuitive.''
Likewise, P9 described how the hierarchical layout she saw for the Autism dataset was a lot easier to follow than the Police dataset shown in the table layout for \cluster:
\begin{quote}
If I had to look at this dataset in the format of the other one, this would be much more difficult. It was pretty hard for me to tell in the other one how to organize the tree, if there was even a tree to be organized. I like this layout much better, I think this layout allows me to approach it in a more meaningful way. I can decide, what do I think matters more: the overall trend? or the super detailed trends? and I know where to look to start, in the other one, every time I go back to it, I would say, where's the top level, where's the second level? I mentally did this. Like when you asked me that first question, it took much longer to find it, because I literally have to put every chart in a space in my head and that took a lot longer than knowing how to look at it.
\end{quote}
At the end of the study, some participants who saw table layouts sketched and explained how they would like the layout of the visualizations to be done. Participants expressed that they wanted ``groupings'' or layouts that arranged visualizations with the same attribute together. Other participants advocate for isolating the overall visualization outside of the dashboard table for facilitating easier comparisons. Both of these provides further motivation for our hierarchical layout and the idea of the collapsed visualizations as described in Section \ref{sec:interaction}.
\par Since we did not inform participants how the dashboards were generated, it was also interesting to note that some participants thought that the dashboards were hand-picked by a human analyst and described what this person's intentions were (e.g. ``It seems like the researcher who created this dashboard was specifically looking at people of Asian descent and people who are 60 or older.'' [P7.A1]). We encoded this phenomenon by looking at instances where a participant either explicitly referring to a person who picked out the dashboard or implicitly described their intentions through personal pronouns. A total of 5 different participants referred to the dashboard generated by \system as generated by a human, whereas there was only 1 participant for \cluster and none for \BFS made such remarks. At the end of the study, many were surprised to learn that the \system dashboard was actually picked out by an algorithm, indicating that \system could automatically generate convincing dashboard stories similar to a dashboard that was authored with human intention.

\stitle{Limitations of \system}
% Interestingness task is highly subjective, so not conclusive whether interesting or not , despite the positive result
% Due to the highly subjective nature of the retrieval task, the interestingness selection for the Police dataset was biased by participant's priors and intuition about the attributes. For example, while all participants who have seen the visualization "duration=30+min" verbally noted that stop duration is a crucial factor that leads to arrest, only 4 users marked it as interesting. 5 participants marked the visualization as not interesting and 4 left it unselected, because the visualization was not very surprising as it agreed with their intuition that ``\textit{if the police stop is taking a long time, something has probably gone wrong}''.
\par As described earlier, since the details of how the dashboard was obtained was not explained to the users during the study, some users expressed that they were initially confused by \system since not all variables were present in the dashboard and others found it confusing that the addition of filters did not always correspond to the same variables. For example, [P2.A1] criticized how the dashboard was intentionally selected to be biased:
\begin{quote}
I feel like this one, not all the data is here, so we are already telling a story, you are trying to steer the viewer to look at certain things. And the focus seems to be on where the arrest rate is high. You probably could have found other things that led to ticket being high, but you didn't pull those out. You are trying to see if there are other factors that leads to more arrests.
\end{quote}
\npar This sentiment is related to participants' desire to perform their own ad-hoc querying alongside the dashboard to inspect other related visualizations for verifying their hypothesis. For example, [P7.A1] wanted to inspect all other first-level visualizations for driver's race to assess its influence. [P7.A1] expressed that while he had learned many insights from the dashboard, ``the only thing I don't like is I can not control the types of filter, which is fixed.'' Outside the context of the user study, it is essential to explain how \system are picking the visualizations in a easy and interpretable manner to establish a sense of summarization guarantee for the users and help them make better inferences with the dashboard.
\par As discussed earlier, subpopulation size is important in establishing the `credibility' of a visualization. While subpopulation size is taken into account implicitly in the ``significance'' portion of our objective, we should design interfaces that can convey the notion of subpopulation size in our dashboard, either explicitly displayed as text when hovering over the visualization or changing the size or background color of the visualizations to encode subpopulation size.



%“I actually found it really confused at first because such a low arrest rate at the top, and then at the bottom the arrest rate was much higher, so I was like is this data wrong. Then I realized we’re not looking at all the data here, you’ve pulled out some of it. It took me a minute to realize that. And once I read the title of the charts I realized that makes sense.” [P2.A1]
% - Reference of Comparison
% - Layout naturally lends itself for comparison:
% 	- describe ordering layout, how participants naturally follow the flow
% 	- emph that we did not tell them what the edge connections mean and how they were computed but the users naturally figured it out, that it means adding an additional filter.
% 	- hierarchical interpretable nature (quotes)
% 	- compared to other baselines
% 	- describe dashboard by human (count)
% - Misleading insights v.s. True insight discovery rates
% 	- Interpretability:
% 	- misled understanding subpopulation size
% 		- for autism, it is important to see if they compare to overall because if not they would think high skew to NO is important whereas its actually pretty close to overall.
% 	- trouble interpreting filter combination
%%%%%%%%%%%%%%%%%%%%%%%%%%%%%%%%%%%%%%%%%%%%%%%%%%%%%%%%%%%%%%%%%%%%%%%%%%%%%%%%%%%%%%%%%%%%%%%%%%%%%%%%%%%%%%%%%%%%%%%%
%%%%%%%%%%%%%%%%%%%%%%%%%%%%%%%%%%%%%%%%%%%%%%%%%%%%%%%%%%%%%%%%%%%%%%%%%%%%%%%%%%%%%%%%%%%%%%%%%%%%%%%%%%%%%%%%%%%%%%%%
%%%%%%%%%%%%%%%%%%%%%%%%%%%%%%%%%%%%%%%%%%%%%%%%%%%%%%%%%%%%%%%%%%%%%%%%%%%%%%%%%%%%%%%%%%%%%%%%%%%%%%%%%%%%%%%%%%%%%%%%
% \subsection{Statistical Paradoxes}\dor{make title full sentences}
% Visualizations are powerful representations for studying different distributions or patterns in a dataset, but our human intuition could often mislead us when it comes to interpreting those patterns\cite{Binnig2017,Wall2017}. Several statistical paradoxes can lead analysts to draw incorrect conclusions from observed visualizations, including Simpson's paradox as discussed in the introduction. The key reason why many of these paradoxes emerge is the \emph{incompleteness} of the observed data or lack of focus on relevant informative subsets of the data. For example, Simpson's paradox arises in the presence of an unseen confounding variable. %likewise, the absence of  base rate information causes base rate fallacy.
%  We assert \dor{too strong of a sentence} that distributional awareness can be useful in avoiding such statistical paradoxes. If an analyst is aware of all distributions in a given dataset, he/she is less prone to many statistical paradoxes. However, given the large number of dimensions and high cardinality of these dimension in modern datasets, it is not possible for an analyst to explore and memorize all distributions. Therefore, a more evolved approach is to be aware of the exceptional distributions. In this work, we propose a first step towards this goal, where we identify the exceptional distributions in terms of their informative references. The remaining (unseen) distributions in the dataset are rather unsurprising and can be inferred from the visualizations in the dashboard. \dor{I would recommend first talk about issue with large dimension + danger of multiple hypothesis testing + incomplete testing, point out problem, then talk about how our system resolves this.}

% \subsection{Structural Insight}
% Our proposed dashboard consists of a hierarchy of visualizations, where each visualization is linked to its most informative parent. The shape or structure of the hierarchy contains useful information that augments the information learned from the visualizations and aid distribution awareness and understanding. \dor{what's interesting here is that while many work have looked at visualization presentation, layout of presentation never considered, we find in Sec 5 that this is actually important and can encode info.} For example, the depth and branching factor of the hierarchy could inform a user regarding the configuration of insights. Deep hierarchies contain long paths, i.e., insights are present at lower level visualizations with multiple constraints. In contrast, bushy hierarchies (with high branching factor) contain cases where multiple visualizations have the same informative parent and they differ from that parent. \dor{do we have examples from the study that support this?} We assert that the depth and branching factor could be a meaningful constraint in our problem formulation \dor{too strong of a sentence}. Some applications for example, funnel exploration require studying deep hierarchies, whereas others for example, building decision trees require studying bushy hierarchies. A natural extension of our current problem formulation is to allow users to select the depth and branching factor for the hierarchy. 

% \subsection{Other Visualization Lattices}
% In this work, we explore the space of data subsets to generate our visualization lattice. Note that it is possible to explore the space of dimension attributes in x-axis to generate a different visualization lattice. In particular, given a combination of dimension attributes $X = \{X_1, \ldots, X_n\}$, adding one or more new dimensions in $X$ will generate a new combination. An ancestor-descendant relationship exists between these dimension combinations, following the same principles of Section 3.1. These relationships lead to a new lattice, which we call the dimension combination lattice. Our informative deviation based approach could be used for traversing the dimension combination lattice. However, we observe that most users do not visualize more than two attributes in x-axis. Therefore, traversing the dimension combination lattice is not very useful for most applications.
% \dor{I think 6.2,6.3 don't tie well with the rest of the paper. It sounds like stretching our own ideas rather than being motivated by the work done in this paper. Other potentially more relevant discussion: distribution awareness and how it might be useful in other contexts? Decision trees?}
% %\subsection{Utility Metrics} 



