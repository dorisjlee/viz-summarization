%!TEX root = main.tex
\section{Problem Formulation\label{sec:datamodel}}
\par In this section, we first describe how analysts explore the space of visualizations through drill-downs and introduce a common fallacy that arises when analysts have limited time and attention to examine all possible factors that contribute to the observed visualization. Then, we discuss how to resolve the problem of finding informative references along a drill-down path.
% How users explore visualizations
% 	- Drill down
% 	- Expectation formation
% 		- focussing on bar chart 
% to explore the space of possible data subset
\par Research in visualization storytelling shows that people prefer hierarchically structured visualizations with increasing levels of aggregation~\cite{Kim2017,Hullman2017,Hullman2013}. In order to find meaningful insights, analysts often drill-down to explore data at different levels of granularity by adding one filter at a time. For each data subset that they encounter, they may want to visualize the data distributions through a bar chart. When analysts perform a drill-down by adding one additional filter, they naturally look towards the last visualization that they have seen (known as the `parent') to establish what they expect to see in the current visualization (known as the `child'). In this case, a parent is any visualization that can be obtained by removing one filter constraint from the child. For example in Figure~\ref{fig:elections_example}, the visualizations Female and Black are the parents of the Black Female visualization. %By extending this concept of parent-child relationships, we can organize the space of visualization from different data subsets to form a lattice as shown in Figure~\ref{fig:elections_example}.

% Fallacies of Forming Expectations:
% 	- two extremes: 
% 		- random parent v.s. exhaustive parent browsing
% 		- limitation of analyst 
\par As analysts perform drill-downs, they may be misguided by child visualizations that highly deviate from one of its parents, if one of the other potential factors that explain seemingly-anomalous behavior is overlooked (i.e. not along the chosen drill-down path). As exemplified by the exploration along the purple path in Figure~\ref{fig:elections_example}, we refer to this phenomena as \emph{drill-down fallacy}, since this type of fallacy arises from the inductive nature of the drill-down operation. %We demonstrate this fallacy with an example from the 2016 US Elections exit polls dataset. 
While such fallacies can be prevented if the analyst exhaustively browses through all possible parents of any visualization that he observes in the dataset, the prohibitively large number of visualizations and limited memory and attention of analysts make this task impractical.
% Problem definition 
% 	- picking right parent
\par Due to these challenges, our goal is to develop a mechanism that would  \emph{provide safe guarantee by picking the proper informative parent} as a reference when analysts navigate through the space of data subsets.  To model the informativeness of an observed parent in the context of an unseen visualization, we characterize the capability of the parent in predicting the unseen visualization. An observed parent is \emph{informative} if its data distribution closely follows the data distribution of the unseen child visualization, since the visualization helps the analyst form an accurate mental picture of what to expect from the unseen visualization. Specifically, we formulate the informativeness of an observed parent $V_i^j$ of an unseen visualization $V_i$ as the similarity between their data distributions measured using a distance function $D(V_i, V_i^j)$. The most informative parents $V_i^*$ of an unseen visualization $V_i$ are the ones whose data distributions are most similar to the unseen.
\begin{equation}
    V_i^*=\underset{V_i^j}{argmin}\ D(V_i, V_i^j)
\end{equation}
We regard a visualization as informative if its distance falls within a user-defined threshold $\theta\%$ close to its most informative parent:
\begin{equation}
    V_i^{*, \theta} = \{V_i^j : \frac{D(V_i, V_i^*)}{D(V_i, V_i^j)} \geq \theta\}
\end{equation}
For example in Figure~\ref{fig:elections_example}, while both visualization Black and Female visualizations are considered parents of the Black Female visualization, only the Black visualization are considered the informative parent of the black female population, for any values of $\theta \geq 11\%$ via the Euclidean distance metric. Note that, our proposed system can work with different distance metrics such as cosine similarity and earth mover's distance. Without loss of generality, we chose to use Euclidean distance metric for the remainder of our paper.
