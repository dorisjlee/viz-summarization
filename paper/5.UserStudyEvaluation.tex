\section{User Study Evaluation}
\subsection{Procedure}
Given that our preliminary study motivated our desiderata for the metrics and constraints used in the problem formulation, we further evaluate the utility of our tool by performing a user study focusing on addressing the research questions: 
\begin{itemize}
	\item RQ1: How effective is our tool at summarizing key insights within a given dataset?
	\item RQ2: How effective is our tool in providing analysts with task-specific insights? (including understanding causal explanations, identification of interesting visualization, and finding outliers[?])
	\item RQ3: How useful are the visualizations in the recommended dashboard to analysts?
\end{itemize}
\par We recruited [NUMBER] participants who have experience working with data. This can include, but are not limited to, browsing and reading data, data cleaning and wrangling, data visualization and model building. The inclusion criteria is assessed based on a self-reporting basis in the pre-study survey. [REPORT BASIC DEMOGRAPHICS ABOUT USERS.]
\par The user study is composed of two phases. Phase one of the experiment focuses on comparing our tool against a set of baselines intended to simulate the natural sequence of visualizations that an analyst would encounter through various approaches during exploratory analysis. The baselines include: 
\begin{enumerate}
	\item Clustering (\textit{Clust}): K-Means clustering is performed on the dataset with k clusters. For each representative cluster, pick the visualization that is closest to the cluster center and display the visualization in table layout in the order of increasing number of conditions in the filter combination.
	\item Parent-Child Pairs (\textit{PC}) : Pick top k/2 parent-child pairs without enforcing informative criterion. Display as pairs of disconnected graph.
	\item Bread-First (\textit{BFS}) : k visualizations is selected in level-wise order and displayed in a graph layout. 
	\item Random (\textit{Rand}) [OPTIONAL]: k visualizations is randomly selected and display in table layout. This is intended to simulate user behavior on free-form visualization generation tool, where users add arbitrary combinations of filters to explore an unfamiliar dataset.
	%\item Direct manipulation: Allow users to add arbitrary combinations of filters and facets. Record insights until k visualizations seen. Display in table layout.
	\item Direct manipulation (\textit{DM}): Through crowdsourcing, we asked a group of expert users to add arbitrary combinations of filters and facets to create a dashboard of k visualization while given a time constraint of 30 minutes. These expert-generated dashboards are used as the DM baselines viewed by all other users in the user study. \footnote{We chose not to perform a comparison with a Tableau-like interface, since activities involved in free-form DM and our tool is fundamentally different. Direct manipulation engages users in the act of visualization construction and browsing through an iterative workflow, while our displayed dashboard only require the users to browse a set of recommendation. The confounding variable related to the cost-benefit tradeoff between visualization construction versus browsing is an unaddressed but interesting direction of research for visualization recommendation.}
\end{enumerate}
%To prevent learning effects, the ordering of the baselines will be randomized across users. 
\par At the beginning of the study, participants were provided with a dashboard from an example dataset, as well as an explanation of how the dashboard is generated. For each of the visualization dashboard generated by our tool or one of the baselines, participants are asked to mark visualizations as interesting/not interesting while explaining their reasoning for each annotation. Then, they are asked to summarize a list of insights that they have discovered after browsing through all visualizations in the dashboard. Participants also answer a set of task-specific questions related to causality and outliers[?], in the form as shown in the example [*]. These tasks are repeated for all baselines and our tool in randomized order on different datasets to prevent learning effects. At the end of phase one, participants are asked to comment on their experiences with each method, as well as the pros and cons of each of the tools. This phase of the experiment is designed to quantify the effects of RQ 1 and 2. 
\par To prevent fatigue, after a 5 minute break, the participants then proceed onto phase two of the study, where they are given [10] dashboards generated by our tool and are asked to engage in a talk-aloud exercise as they browse the recommended visualizations. This is a more open-ended study intended for addressing RQ3 that can reveal our tool show unimportant results across different datasets and/or highlight larger selection of the types of insights that can be generated from the tool. 
\subsection{Result}