\newpage
\section{Introduction}
\par The task of navigating through a large, multidimensional dataset is a common challenge in data analytics. For example, a data scientist may want to examine the bar charts summarizing \% of sales for different user populations from different demographic groups such as states, gender, and age group\dor{we need to prolonged example in the intro and throughout}, but is not sure about \textit{what} subset of the data would be interesting to visualize. His struggle is not unjustified: a simple dataset with 10 attributes (with an average of 4 possible values per attributes) can yield up to ---- possible combinations. Not only is manual drill-down and roll-ups tedious and inefficient for the analyst, but even with all the information given, currently there is also no systematic and effective way for an analyst to make sense of and navigate through the large space of possible visualizations. 
\par Our work was motivated by storyboards commonly used in the development of movies or animations. Storyboards are often rough, drafting, summaries of the whole ----  ----discussion among diretors, storywriters and film crews, iterate design and --- fill in the details --. Similarly, in the context of exploratory data analysis, we -----whether it was possible to algorithmically generate a dashboard ---- provides the ---, guidance ----,  starting point to provoke further exploration?
\par 

The contribution of this paper include: 
\begin{itemize}
\item Proposing a novel problem of identifying a set of informatively connected interesting visualizations,
\item Presenting a novel visualization dashboard design that adopts a simple and intuitive graph layout,
\item Designing efficient algorithms and optimizations to solve this problem,  
\item Evaluating the efficacy of our system through experiments and user study
\end{itemize}