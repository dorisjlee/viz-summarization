%!TEX root = main.tex
\section{Related Works}
\subsection{Data Exploration of OLAP Data Cubes}
Sarawagi et al.\cite{Sarawagi1998} introduced the problem that manual, unguided search for seeking interesting patterns in a datacube is inefficient and requires large numbers of online analytical processing (OLAP) operations, such as roll-up, drill-down, slicing, and dicing. They proposed a discovery-driven approach to data exploration to simplify the search for \textit{exceptions} in the data, based on precomputed statistics regarding how surprising a data cube cell is, relative to neighboring cells at the same level of aggregation, at levels of aggregation below the cell, and along the drill-down path. Sarawagi \cite{Sarawagi1999} presents an OLAP operator that summarizes the reasons for variation in a data view, by computing the information theoretic distance between the immediate parent and its child nodes.
\par While Sarawagi et al. \cite{Sarawagi1998} takes a more data-driven approach of finding exceptions intrinsic to a given datacube, Sarawagi \cite{Sarawagi2000} envisions a more user-centric application where the comparisons are based only on parents of seen visualization. The user's expectation regarding an unseen visualization is based on the maximum entropy principle where the relationships between the attributes should be maximally uniform across all dimensions, while being consistent with constraints from seen visualizations. The visualization that deviates the most from the user expectation is regarded as an ``interesting" visualization. They take an iterative approach to find the unique solution for the expected values for each attribute from the constrained maximum entropy problem and employ several optimization strategies (reusing computed values, sharing storage of contiguous regions, pruning constraints that subsumes one another or with little influence). Our coverage-based models improve on the iterative approach in providing a tighter constraint to the variable regions of the bars.
\subsection{Visualization recommendation}
\par Despite the large body of work that recommends informative visualizations given pre-selected data attributes and aggregations, the data selection problem is a more important problem in exploratory data analysis, since the analysts have to figure out which groups of data attributes would be of interest in order avoid manual exploration of the data. Anand et al. \cite{Anand2015} used randomized permutation tests to automatically select categorical variables to partition upon to determine which displaying multiple views of a multidimensional dataset. Vartak et al. \cite{Vartak2015} developed a system that finds interesting visualizations by a distance-based measure between the user's query view and reference view,  given a query of interest. Wongsuphasawat et al. \cite{Wongsuphasawat2016} presents a mixed-initiative system where the users direct the variables of interest and the system suggests other variables that may be potentially interesting to the user. Since this is a mixed-initiative system rather than an automatic recommendation engine, the system only ``looks ahead"  by only one variable at a time. Their goal is to promote breadth-oriented data exploration rather than helping users find interesting stories or visualizations.
\subsection{Storytelling with visualization sequences}
- the issue of surprisingness metric and ---- have been examined before for viz, but none have looked at data subset lattice specifically for viz

While most papers on visualization sequences and storytelling have focussed on using a linear layout to present the visualizations (motivated by slidedecks), ours is the first to present visualization in a graph-layout shown to be more interpretable for drill-down, roll-up type operations.