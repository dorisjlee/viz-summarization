\section{User Interaction\label{sec:interaction}}
\begin{figure}[ht!]
\centering
\includegraphics[width=\linewidth]{figures/overview.jpeg}
\caption{Overview of the \system interface for the Police Dataset. Users can select x and y axes of interest, as well as a choice of an aggregation function. Default values are set for system related parameters such as the number of visualizations to show in the dashboard (k), iceberg condition for pruning ($\delta$), and informative parent criterion ($\theta$), which can be adjusted by the users via the sliders if needed.}
\label{fig:overview}
\end{figure}
\par Figure \ref{fig:overview} shows an overview of the \system interface. After the user selects the x and y axes of interest, aggregation function, and optional system parameter settings, an initial dashboard of $k$ visualizations is displayed on the canvas, such as the one seen in main canvas of Figure \ref{fig:overview}. The system provides toolbar buttons with keyboard binding for zooming in, out, and extent, as well as moving around the canvas. Alternatively, users can also zoom and pan with mouse click and scroll.
\par After browsing through the visualizations the dashboard, users may be interested in getting more information about a particular node. \system supports a mechanism for users to request additional summarizations based on a chosen node of interest. As shown in Figure \ref{fig:altroot_expansion}, the analyst starts a 5-visualization dashboard. He learns that for the drivers who had contraband found in the vehicle, the arrest rate for drivers who are 60 and over is surprisingly higher than usual, whereas for Asian drivers the arrest rate is lower. Apart from the contraband found stories, he is also interested in learning more about the other factor that contribute to high arrest rate: duration=30+min. He clicks on the corresponding visualization and requests for 2 additional visualizations. Upon seeing the updated dashboard, he learns that unlike the contraband found story, the child distributions of the duration=30+min distribution largely follows its parent distribution, which implies that if a police stop last more than 30 minutes, the outcome would more or less be the same, independent of other factors, such as driver's race or age. The system uses the same models and algorithms as discussed earlier to generate the expansion dashboard, with the only difference that the starting node is now the selected visualization, rather than the overall visualization. This node expansion capability is similarly motivated by the idea of \textit{iterative view refinement} in other visual analytics system\cite{Wongsuphasawat2016,Hoque2017}, which is essential for the users to iterate on and explore different hypotheses. 
\begin{figure}[ht!]
\centering
\includegraphics[width=\linewidth]{figures/expansion_example.pdf}
\caption{Left: Original k=5 dashboard with the duration=30+min visualization clicked. A pop-up is displayed to submit the request for additional summary visualizations to be generated. Right: Resulting dashboard after requesting for 2 more visualizations based on the visualization of interest.}
\label{fig:altroot_expansion}
\end{figure}

\subsection{Assistive tools for visualizing large lattices\label{sec:navigation}}
Due to the amount of space occupied by the hierarchical layout when the number of visualizations gets large, we have developed tools to help users navigate through different parts of the dashboard interactively. 
\stitle{Navigation Minimap:}  When the user zooms in on the dashboard, an overview mini-map is shown on the upper right-hand side of the canvas to help users identify which region of the dashboard they are currently exploring, as shown in Figure \ref{fig:hover_minimap}. 
\stitle{Collapsed visualizations:} 
One observation that we found across several datasets was that many visualizations had identical distributions, which resulted in lots of wasted space. Apart from their attribute name, these visualizations are not very informative for the users, therefore, we offer an option to collapse these visualization, as demonstrated in Figure \ref{fig:collapse_demo}. A visualization can be collapsed if it has more than one redundant sibling and does not have any children, so that there are no hidden stories due to lower-level dependencies. As shown in Figure \ref{fig:hover_minimap}, collapsed nodes can be easily identified by an orange border and the details of which visualizations are in the collapsed node are displayed when the user hovers over the visualization.
%\afterpage
\begin{figure}[ht!]
\centering
\includegraphics[width=\linewidth]{figures/minimap_zoom.png}
\caption{Zoomed-in version of Figure \ref{fig:collapse_demo} showing the labels of a collapsed visualization when user hovers over the visualization. The navigation minimap is shown in the top-left to help users navigate through the large dashboard.}
\label{fig:hover_minimap}
\end{figure}

\subsection{Algorithms\label{sec:algorithms}}
We give an overview of our algorithms by first discussing the approaches to generate the visualization lattice, and then presenting a high-level overview of our traversal algorithms.

\stitle{Lattice Generation.} Our system supports two variants of traversal algorithms based on the lattice generation procedure---offline variants that first generate the complete lattice and then work towards identifying the maximum utility solution, and online variants that incrementally generate the lattice and simultaneously identify the solution. The offline variants are appropriate for datasets with a small number of low-cardinality attributes, where we can generate the entire lattice in a reasonable time; whereas the online variants are appropriate for datasets with large number of high-cardinality attributes, where we incrementally generate a partial lattice. 

%In most cases, the lattice contains a large number of visualizations due to the presence of many attributes or high-cardinality attributes in the dataset. In such cases finding an optimal solution is computationally challenging.

\stitle{Lattice Traversal.} Given the materialized lattice, the objective of the traversal algorithm is to find the connected subgraph in the lattice that has the maximum combined edge utility. Here, we discuss the \textit{frontier greedy} algorithm which is used for generating the dashboards for our user study and defer our discussion on the details of other algorithms that we have developed to the technical report.
% \begin{figure}[ht!]
% \centering
% \includegraphics[width=0.4\linewidth]{figures/frontier.pdf}
% \caption{Toy example demonstrating the notion of ``frontier''. Nodes that have been picked to include in the dashboard are colored green. The neighbors of the set of picked nodes are the frontier nodes, shown in pink. Grey nodes are other unpicked nodes in the lattice.}
% \end{figure}
%We devised two classes of heuristics algorithms, namely, frontier-based algorithms, and path-merging algorithms. These algorithms are guaranteed to find a solution that satisfies the constraints of our problem, except for the optimality. 
\techreport{The frontier-based algorithms traverse the lattice from root to downwards, incrementally adding new nodes (visualizations) to the current solution (dashboard) till it reaches the maximum capacity $k$. To achieve this, the algorithms maintain a list of candidate nodes---called \textit{frontier} nodes---any of which can be added to the current solution since their informative parent is already present in the solution. At each step, the algorithms add a node from frontier to the current solution, and update the frontier accordingly.  The frontier based algorithms can be further categorized into three types based on their node selection strategy (from frontier), namely greedy algorithm, random walk algorithm, and probabilistic algorithm. The greedy algorithm picks the current best node from frontier (thus concentrates on exploitation), random walk algorithm picks a random node (thus concentrates on exploration), and probabilistic algorithm picks a random high-utility node (thus trades off between exploration and exploitation).}
\par As described in Algorithm \ref{algo:frontier_greedy}, our algorithm obtains a list of candidate nodes known as the \textit{frontier} nodes (pink in Figure\ref{fig:lattice} left), which encompasses all neighbors of nodes in the existing subgraph solution. Any of the nodes in the frontier can be added to the current solution since their informative parent is guaranteed to be present in the solution. The \texttt{getFrontier} function scans and adds all children of leaf nodes of the current dashboard as part of the frontier. In the online version, it additionally checks for each child whether its informative parent is present in the current dashboard. At each step, our algorithm greedily picks the node with the maximum utility from the frontier to the current solution, and updates the frontier accordingly. 
