\section{Introduction}
Visual data exploration are accessible and intuitive means for helping analysts understand multidimensional datasets. Analysts often study data distributions at different levels of data granularity~\cite{Anand2015,Wu2013,Heer2012} to discover trends and patterns, generate or verify hypotheses, and understand complex causal relationships. While visualizations exploits analysts's perceptual capabilities for visual pattern recognition, the effectiveness of human perceptual reasoning breaks down as datasets increases in size and complexity. 
\par Manual exploration of multidimensional dataset is a time-consuming and error-prone process. To explore different subsets of data in a multidimensional dataset, an analysts would need to manually perform \emph{drill-downs} on the space of possible data subsets by adding one filter at a time, without knowing what data subset would lead to an insight. Even if visualizations for all possible data subsets has been constructed, there is no systematic way for analysts to make sense of or even navigate through this large space of possible visualizations. More alarmingly, without being able to contextualize the relationships between data subsets, analysts may mistaken the general cause of an observed deviation in trend for a specific cause---a fallacy coined as ``drill-down fallacy'' in our paper~\cite{}.
\par To this end, we present \vispilot, a visual analytic tool that addresses the challenges of manual drill-down exploration. \vispilot traverses the space of possible data subsets (hereafter known as the \emph{lattice}) to \emph{automatically} identify a \emph{compact} network of  \emph{informative} and  \emph{interesting} visualizations that collectively convey the key insights in a dataset. Demo attendees will have the opportunity to interact with \vispilot to better understand and gain insights regarding of multiple provided datasets.